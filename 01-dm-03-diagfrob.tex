% Chapter:  Frobenius on the diagonal fibre  %%%%%%%%%%%%%%%%%%%%%%%%%%%%%%%%%%

In this chapter we discuss the action of Frobenius on the diagonal fibre. 
We first describe the general method based on an explicit formula by 
Dwork~\citep[\S 4]{Dwork64}, following the expositions in the work of 
Lauder~\citep[\S 6]{Lau04} and Gerkmann~\citep[\S 4.4]{Gerkmann2007}.  Then, 
we present a new improvement in Theorem~\ref{thm:alpha}, showing that a 
certain expression is defined already over $\mathbf{Q}_p$ instead of a ramified
extension as observed by Gerkmann.  Furthermore, Theorems~\ref{thm:alpha2} 
and \ref{thm:alphap} state the expression lies in $\mathbf{Z}_p$ and 
provide computationally suitable reformulations.  To conclude this chapter, 
we illustrate the practical relevance of these improvements with three example 
computations.

% Introduction %%%%%%%%%%%%%%%%%%%%%%%%%%%%%%%%%%%%%%%%%%%%%%%%%%%%%%%%%%%%%%%%

\section{Introduction}

In order to simplify our notation, we temporarily suppress the earlier 
setup of a family of hypersurfaces.
Thus, we consider a smooth, projective, diagonal hypersurface~$\mathfrak{X}$ 
over $\mathbf{Z}_p$ given by a polynomial $P \in \mathbf{Z}[x_0, \dotsc, x_n]$ 
of homogeneous degree~$d$ and its reduction~$X$ over $\mathbf{F}_p$ given by 
\begin{equation*}
\tilde{P}(x_0, x_1, \dotsc, x_n) = 
    a_0 x_0^d + a_1 x_1^d + \dotsb + a_n x_n^d = 0
\end{equation*}
where $a_0, a_1, \dotsc, a_n \in \mathbf{F}_p^{\times}$ and $p \nmid d$.  
We also define affine complements $\mathfrak{U}$ and $U$ of $\mathfrak{X}$ 
and $X$, respectively, and choose the basis 
$\mathcal{B}_1 \cup \dotsb \cup \mathcal{B}_n$ for $H_{dR}^n(\mathfrak{U})$ 
where, for $k = 1, \dotsc, n$, 
\begin{align*}
B_k & = \{ x^i : \text{$\abs{i} = kd - (n+1)$ and $0 \leq i_j < d-1$, for all $j$} \}, \\
\mathcal{B}_k & = \{x^i \Omega / P^k : x^i \in B_k \}, 
\end{align*}
recalling the condition that $d \geq 2$ whenever $n$ is odd and $d \geq 3$ 
whenever $n$ is even.  Our goal is to compute the matrix~$F$ representing 
the action of $p^{-1} F_p$ on $H_{rig}^n(U) \cong H_{dR}^n(\mathfrak{U})$ 
to $p$-adic precision~$N$.

We now describe an explicit formula for the $b = \dim H_{dR}^n(\mathfrak{U})$ 
non-zero coefficients of the matrix~$F$ due to Dwork~\citep{Dwork64}, see 
e.g.\ Lauder~\citep[\S 6.1]{Lau04}.   We work over the ramified 
extension~$\mathbf{Q}_p(\pi)$ where $\pi^{p-1} = -p$, where we normalise 
the valuation such that \mbox{$\ord_p(\pi) = (p-1)^{-1}$}.

Let $u = (u_0, \dotsc, u_n)$ and $v = (v_0, \dotsc, v_n)$ be tuples such 
that $x^u, x^v \in B_1 \cup \dotsb \cup B_n$,  and let $u'$ and $v'$ 
denote integers such that $d u' = \sum_{i=0}^n (u_i + 1)$ and similarly 
for $v'$.  Finally, for $m \geq 0$, let $\lambda_m$ denote the coefficient 
of $z^m$ in the expansion of $\exp \pi (z - z^p)$ and define products 
$(w)_r = \prod_{j=0}^{r-1} (w + j)$ for $w \in \mathbf{Q}$ and $r \geq 0$. 
We introduce terms $\alpha_{u,v}$, 
\begin{equation*}
\alpha_{u,v} = \pi^{v' - u'} \prod_{i = 0}^n \sum_{m, r} \lambda_m (u_i / d)_r (-1)^r \pi^{-r} {\hat{a}_i}^{m-r}
\end{equation*}
where $\hat{a}_i$ is the Teichm\"uller lift of $a_i \in \mathbf{F}_p$ 
to $\mathbf{Q}_p$ and the summation indices $m, r \geq 0$ satisfy 
$p u_i - v_i = d (m - pr)$.

\begin{thm} \label{thm:01-03-diagfrob}
Continuing with the notation set out above, let $\omega_1$ and $\omega_2$ 
denote the forms in $\mathcal{B}_1 \cup \dotsb \cup \mathcal{B}_n$ 
corresponding to $u$ and $v$, respectively.  
Then $p^{-1} F_p (\omega_1) = 0$ unless, for all $i = 0, \dotsc, n$, 
$p (u_i + 1) \equiv v_i + 1 \pmod{d}$.  In this case, 
\begin{equation*}
p^{-1} F_p (\omega_1) = 
    (-1)^{u' + v'} \frac{(v' - 1)!}{(u' - 1)!} p^n \alpha_{u+1,v+1}^{-1} \omega_2
\end{equation*}
where $u + 1 = (u_0 + 1, \dotsc, u_n + 1)$ and similarly for $v + 1$.
\end{thm}

\begin{proof}
See Dwork~\citep[\S 4]{Dwork64} for Dwork's original work, or 
Lauder~\citep[\S 6.1]{Lau04}.  Both references also treat the 
more general case when $\mathfrak{X}$ is a smooth, projective, 
diagonal hypersurface over $\mathbf{Z}_q$, for $q$ a prime power.
\end{proof}

% Improvements %%%%%%%%%%%%%%%%%%%%%%%%%%%%%%%%%%%%%%%%%%%%%%%%%%%%%%%%%%%%%%%%

\section{Improvements}

From the previous description, it appears that this computation 
genuinely has to take place over the extension field 
$\mathbf{Q}_p(\pi)$.  This is, however, not the case as we 
will show in this section.  We are able to show that the terms 
$\alpha_{u+1,v+1}$ are $p$-adic integers and provide expressions 
for these that are more suitable from a computational perspective.  

\begin{lem} \label{lem:lambdam}
Let $\pi^{p-1} = -p$ and, for $m \geq 0$, let $\lambda_m$ 
be the coefficient of $z^m$ in the power series expansion 
of $\exp \pi (z - z^p)$ in $\mathbf{Q}_p[[z]]$.  Then 
\begin{equation*}
\pi^{- (m \bmod {p-1})} \lambda_m = (-1)^{\floor{m/(p-1)}} \sum_{k=0}^{\floor{m/p}} p^{\floor{m/(p-1)} - k} \frac{1}{(m-pk)! k!},
\end{equation*}
where $m \bmod{p-1}$ denotes the remainder of $m$ upon Eulidean 
division by $p-1$.
\end{lem}

\begin{proof}
From
\begin{equation*}
\exp \pi (z - z^p) = \sum_{n=0}^{\infty} \frac{\pi^n}{n!} (z - z^p) ^n
                   = \sum_{n=0}^{\infty} \frac{\pi^n}{n!} \sum_{k=0}^n \binom{n}{k} (-1)^k z^{n + k(p-1)}
\end{equation*}
we find that 
\begin{align*}
\lambda_m & = \sum_{n = 0}^{\infty} \frac{\pi^n}{n!} \sum_{\substack{0 \leq k \leq n\\n + k(p-1) = m}} \binom{n}{k} (-1)^k \\
          & = \sum_{\substack{\ceil{m/p} \leq n \leq m\\ n \equiv m \pmod{p-1}}} \frac{\pi^n}{n!} \binom{n}{(m-n)/(p-1)} (-1)^{(m-n)/(p-1)} \\
\intertext{To simplify this further, let $m = q(p-1) + r$ and 
$n = q'(p-1) + r'$, noting that $r = r'$ and also setting 
$k = (m-n)/(p-1) = q - q'$, }
\lambda_m & = \sum_{\substack{\ceil{m/p} \leq n \leq m\\ n \equiv m \pmod{p-1}}} \pi^r p^{q'} (-1)^{q' + k} \frac{1}{(n-k)! k!} \\
          & = (-1)^q \pi^r \sum_{k=0}^{\floor{m/p}} p^{q - k} \frac{1}{(m - pk)!k!}
\end{align*}
The expression for $\pi^{-(m \bmod{p-1})} \lambda_m$ now follows.
\end{proof}

\begin{thm} \label{thm:alpha}
Continuing with the notation set up in this chapter, let 
$u, v \in \mathbf{Z}^{n+1}$ be such that 
$x^u, x^v \in B_1 \cup \dotsb \cup B_n$ and satisfy, 
for all $i$, $p (u_i + 1) \equiv v_i + 1 \pmod{d}$. 
Then 
\begin{equation*}
\alpha_{u+1,v+1} = (-p)^{u'} \prod_{i=0}^n 
    \hat{a}_i^{(p (u_i + 1) - (v_i + 1))/d} \sum_{m,r} 
    \Bigl(\frac{u_i+1}{d}\Bigr)_r 
    \sum_{k=0}^{\floor{m/p}} \frac{p^{r-k}}{(m-pk)! k!}.
\end{equation*}
where $m, r \geq 0$ satisfy $p (u_i + 1) - (v_i + 1) = d (m - pr)$. 
\end{thm}

\begin{proof}
We begin from the definition of $\alpha_{u+1,v+1}$ 
and use Lemma~\ref{lem:lambdam},
\begin{align*}
\alpha_{u+1,v+1} 
    & = \pi^{v'-u'} \prod_{i=0}^n \sum_{m,r} \lambda_m \Bigl( \frac{u_i+1}{d} \Bigr)_r (-1)^r \pi^{-r} \hat{a}_i^{m-r} \\
    & = \pi^{v'-u'} \prod_{i=0}^n \sum_{m,r} (-1)^{\floor{m/(p-1)}} \pi^{m \bmod{p-1}} \biggl( \sum_{k=0}^{\floor{m/p}} \frac{p^{\floor{m/(p-1)}-k}}{(m-pk)!k!} \biggr) \Bigl( \frac{u_i+1}{d} \Bigr)_r (-1)^r \pi^{-r} \hat{a}_i^{m-r} \\
\intertext{We now write $m = \floor{m/(p-1)} (p-1) + (m \bmod{p-1})$ and simplify by $\pi^{p-1} = -p$,}
\alpha_{u+1,v+1}
    & = \pi^{v'-u'} \prod_{i=0}^n \sum_{m,r} \pi^{m - r} \biggl( \sum_{k=0}^{\floor{m/p}} \frac{p^{-k}}{(m-pk)!k!} \biggr) \Bigl( \frac{u_i+1}{d} \Bigr)_r (-1)^r \hat{a}_i^{m-r} \\
\intertext{and now using that $m-r = (p-1)r + (p(u_i+1) - (v_i+1))/d$,}
\alpha_{u+1,v+1}
    & = \pi^{v'-u'} \prod_{i=0}^n \pi^{(p (u_i + 1) - (v_i + 1))/d} \hat{a}_i^{(p (u_i + 1) - (v_i + 1))/d} \sum_{m,r} \biggl( \sum_{k=0}^{\floor{m/p}} \frac{p^{r-k}}{(m-pk)!k!} \biggr) \Bigl( \frac{u_i+1}{d} \Bigr)_r \\
    & = (-p)^{u'} \prod_{i=0}^n \hat{a}_i^{(p (u_i + 1) - (v_i + 1))/d} \sum_{m,r} \Bigl( \frac{u_i+1}{d} \Bigr)_r \sum_{k=0}^{\floor{m/p}} \frac{p^{r-k}}{(m-pk)!k!}
\end{align*}
as required.
\end{proof}

In particular, Theorem~\ref{thm:alpha} implies that 
$\alpha_{u+1, v+1} \in \mathbf{Q}_p$.  Our next aim to obtain 
expressions for $\alpha_{u+1,v+1}$ which are defined over 
$\mathbf{Z}_p$.  But before progressing, we collect a few 
intermediate results which we will refer to later.

\begin{prop} \label{prop:mpr1}
Let $u, v \in B_1 \cup \dotsb \cup B_n$, $i \in \{0,\dotsc,n\}$ and 
let $m, r \geq 0$ satisfy $d(m-pr) = p(u_i + 1) - (v_i + 1)$.  Then 
\begin{equation*}
0 \leq m - p r \leq \frac{p(d-1)-1}{d}.
\end{equation*}
In particular, $m = m(r) = pr + d^{-1}\bigl(p(u_i+1)-(v_i+1)\bigr) \geq 0$.
\end{prop}

\begin{proof}
This can be easily verified using that $0 \leq u_i, v_i \leq d - 2$ 
and $m - pr \in \mathbf{Z}$.
\end{proof}

\begin{prop} \label{prop:mpr2}
Let $u, v \in B_1 \cup \dotsb \cup B_n$, $i \in \{0,\dotsc,n\}$ and 
let $m, r \geq 0$ satisfy $d(m-pr) = p(u_i + 1) - (v_i + 1)$.  Then 
\begin{equation*}
r - \floor{\frac{m}{p}} \geq 0.
\end{equation*}
\end{prop}

\begin{proof}
Using the previous proposition,
\begin{equation*}
r - \floor{\frac{m}{p}} 
= - \floor{\frac{m-pr}{p}} 
\geq - \floor{\frac{p(d-1)-1}{pd}} 
= -1 + \ceil{\frac{p + 1}{pd}} 
= 0 
\end{equation*}
as $p \geq 2$, $d \geq 2$.
\end{proof}

\begin{prop} \label{prop:rfac}
For all integers $u, d \geq 1$ and $r \geq 0$ with $p \nmid d$, 
\begin{equation*}
\ord_p\Bigl(\frac{u}{d}\Bigr)_r \geq \frac{r}{p-1} - \floor{\log_p(r) + 1}.
\end{equation*}
\end{prop}

\begin{proof}
Let $s_p(r)$ denote the sum of digits in the $p$-adic expansion of~$r$ 
and observe that $s_p(r) \leq \floor{\log_p(r) + 1}$.  Using the fact that 
$\ord_p((u/d)_r) \geq \ord_p(r!)$ from Clark~\citep[Page~265, Case~3]{Clark66} 
it follows that 
\begin{equation*}
\ord_p\Bigl(\frac{u}{d}\Bigr)_r \geq \ord_p(r!) = \frac{r - s_p(r)}{p-1} \geq \frac{r}{p-1} - \floor{\log_p(r) + 1}
\end{equation*}
as required.
\end{proof}

% The case $p = 2$ %%%%%%%%%%%%%%%%%%%%%%%%%%%%%%%%%%%%%%%%%%%%%%%%%%%%%%%%%%%%

\subsection{The case $p = 2$}

We first note that in the case when $p = 2$, the expression for 
$\lambda_m$ in Lemma~\ref{lem:lambdam} can be simplified.

\begin{cor} \label{cor:lambda2}
Let $p = 2$, let $\pi^{p-1} = -p$ and, for $m \geq 0$, let $\lambda_m$ 
be the coefficient of $z^m$ in the power series expansion of 
$\exp \pi (z - z^p)$ in $\mathbf{Q}_p[[z]]$.  Then 
\begin{equation*}
\lambda_m = (-1)^m \sum_{k=0}^{\floor{m/p}} \frac{p^{m - k}}{(m - pk)! k!}.
\end{equation*}
\end{cor}

\begin{lem} \label{lem:mu2}
Let $p = 2$ and define a sequence $\bigl(\mu_m^{(2)}\bigr)$ by 
\begin{equation*}
\mu_m^{(2)} = 
    \sum_{k=0}^{\floor{m/2}} \frac{2^{\floor{3m/4} - \nu_m - k}}{(m-2k)! k!}
\end{equation*}
where $\nu_m$ is equal to one whenever $m = 3, 7$ and zero otherwise. 
Then $\mu_m \in \mathbf{Z}_p$ for all $m \geq 0$.
\end{lem}

\begin{proof}
In the two cases $m = 3, 7$ we explicitly compute the values of 
$\mu_m$ as $4/3$ and $232/315$.  Now suppose that $m \neq 3, 7$. 
From Corollary~\ref{cor:lambda2} we obtain that 
\begin{equation*}
\ord_2 \bigl(\mu_m\bigr) 
    = \floor{3m/4} - m + \ord_2(\lambda_m).
\end{equation*}
Using the bound $\ord_p(\lambda_m) \geq \bigl((p-1)/p^2\bigr) m$ from 
Dwork~\citep[Pages~55--57]{Dwork62}, we obtain the lower bound 
\begin{equation*}
\ord_2 \bigl(\mu_m\bigr) 
    \geq \floor{3m/4} - m + \ceil{\frac{m}{4}} = 0. \qedhere
\end{equation*}
\end{proof}

\begin{rem}
The expression $\mu_m$, $m \neq 3, 7$, can be computed efficiently 
modulo~$p^{\tilde{N}}$ via
\begin{equation*}
\mu_m = \frac{2^{\floor{3m/4}}}{m!} 
    \sum_{k=0}^{\floor{m/2}} \frac{m!}{2^k (m-2k)! k!}
\end{equation*}
using only one $p$-adic inversion.  We observe that the summands are 
integers of size $\mathcal{O}(m \log m)$ bits, which allows us to 
avoid performing intermediate reductions modulo~$p^{\tilde{N}}$.
\end{rem}

\begin{thm} \label{thm:alpha2}
Let $p = 2$. 
Continuing with the notation set up in this chapter, let 
$u, v \in \mathbf{Z}^{n+1}$ be such that 
$x^u, x^v \in B_1 \cup \dotsb \cup B_n$ and satisfy, 
for all $i$, $p (u_i + 1) \equiv v_i + 1 \pmod{d}$. 
Then $\alpha_{u+1,v+1}$ can be expressed as 
\begin{equation} \label{eq:alpha2.0}
\alpha_{u+1,v+1} = (-2)^{u'} \prod_{i=0}^n \sum_{m,r} 
    \Bigl(\frac{u_i+1}{d}\Bigr)_r 2^{-\floor{(m+1)/4}+\nu_m} \mu_m, 
\end{equation}
and $\alpha_{u+1,v+1}$ is a $p$-adic integer.
\end{thm}

\begin{proof}
The expression for $\alpha_{u+1,v+1}$ is an immediate consequence 
of Theorem~\ref{thm:alpha} and Lemma~\ref{lem:mu2}, together with 
Proposition~\ref{prop:mpr1} implying $0 \leq m - 2r \leq 1$ and 
hence $r = \floor{m/2}$.  It remains to prove that 
$\alpha_{u+1,v+1}$ is a $p$-adic integer.  Following Lemma~\ref{lem:mu2}, 
it suffices to show that the valuation of the factor 
\begin{equation*}
\Bigl(\frac{u_i+1}{d}\Bigr)_r 2^{- \floor{(m+1)/4} + \nu_m}
\end{equation*}
in each summand is non-negative.  From the proof of 
Proposition~\ref{prop:rfac} we obtain that 
\begin{equation} \label{eq:alpha2.1}
\ord_p \biggl( \Bigl(\frac{u_i+1}{d}\Bigr)_r 2^{- \floor{(m+1)/4} + \nu_m} \biggr)
\geq \ord_p\Bigl(\floor{\frac{m}{2}}!\Bigr) - \floor{\frac{m+1}{4}} + \nu_m.
\end{equation}
Applying Proposition~\ref{prop:rfac}, we see that this is bounded 
below by 
\begin{equation*}
\floor{\frac{m}{2}} - \floor{\frac{m+1}{4}} - \floor{\log_2 m}
\end{equation*}
which is non-negative whenever $m \geq 12$.  In the remaining 
cases $m = 0, \dotsc, 11$, we explicitly verify that the 
lower bound in~\eqref{eq:alpha2.1} is non-negative.
\end{proof}

\begin{rem}
We observe that in Equation~\eqref{eq:alpha2.0} the exponent 
$-\floor{(m+1)/4}+\nu_m$ is non-positive for each value $m \geq 0$, 
which suggests the computational question of how to determine the 
sequence of factors 
\begin{equation*}
\Bigl(\frac{u_i+1}{d}\Bigr)_r 2^{- \floor{(m+1)/4} + \nu_m}
\end{equation*}
to $p$-adic precision $\tilde{N}$ without intermediate precision loss.
To simplify our notation slightly, suppose we wish to compute the sequence
\begin{equation*}
f_r = 2^{- \floor{(m+1)/4} + \nu_m} \prod_{j=0}^{r-1} (u + j d)
\end{equation*}
for all $r \geq 0$, where $m(r) = m(0) + 2r$, $m(0) \in \{0,1\}$, 
$u$ is a positive integer and $d$ is a positive odd integer.
Explicitly, we find that this sequence satisfies 
\begin{align*}
f_0 & = 1, \\
f_1 & = u, \\
f_5 & = \begin{cases}
        \displaystyle \tfrac{1}{4} \bigl( u (u + d) (u + 2d) (u + 3d) (u + 4d) \bigr)
            & \text{if $m(0)=0$,} \\
        \displaystyle \tfrac{1}{8} \bigl( u (u + d) (u + 2d) (u + 3d) (u + 4d) \bigr)
            & \text{otherwise,}
        \end{cases} \\
f_r & = f_{r-2} \frac{(u + (r - 2)d)(u + (r - 1)d)}{2}
\end{align*}
for all remaining $r \geq 0$.  Since the product in the numerator is 
an even integer, it is clear that this computation can be carried 
out without precision loss despite reducing previous values of~$f_r$ 
modulo~$p^{\tilde{N}}$.
\end{rem}

% The case of odd primes %%%%%%%%%%%%%%%%%%%%%%%%%%%%%%%%%%%%%%%%%%%%%%%%%%%%%%

\subsection{The case of odd primes}

\begin{lem} \label{lem:mup}
Let $p \geq 3$ be an odd prime and define a sequence 
$\bigl(\mu_m^{(p)}\bigr)$ by 
\begin{equation*}
\mu_m^{(p)} = \sum_{k=0}^{\floor{m/p}} \frac{p^{\floor{m/p} - k}}{(m-pk)! k!}, 
\end{equation*}
where we write $\mu_m = \mu_m^{(p)}$ when the prime can be identified 
from the context.  Then $\mu_m \in \mathbf{Z}_p$ for all $m \geq 0$.
\end{lem}

\begin{proof}
It is clear that $\mu_m \in \mathbf{Q}$.  From Lemma~\ref{lem:lambdam} 
we observe that 
\begin{equation*}
\lambda_m = \pi^m p^{- \floor{m/p}} \mu_m.
\end{equation*}
Using the bound $\ord_p(\lambda_m) \geq \bigl((p-1)/p^2\bigr) m$ 
from Dwork~\citep[Pages~55--57]{Dwork62}, it follows that 
\begin{equation*}
\ord_p (\mu_m) \geq \frac{p-1}{p^2} m + \floor{\frac{m}{p}} - \frac{m}{p-1}
\end{equation*}
Let us write $m = q p + r$ with $0 \leq r \leq p-1$.  As the valuation 
of $\mu_m$ is an integer, it suffices to show that, for $q \geq 0$, 
\begin{equation*}
\frac{p-1}{p} q + q - \frac{q p + p - 1}{p - 1} > -1
\end{equation*}
which is equivalent to 
\begin{equation*}
\Bigl( \frac{p-1}{p} + 1 - \frac{p}{p - 1} \Bigr) q > 0
\end{equation*}
Dividing by $q$ and multiplying by $p (p-1)$, we obtain the equivalent 
condition
\begin{equation*}
p^2 - 3p + 1 > 0
\end{equation*}
which holds true provided that $p \geq 3$.
\end{proof}

\begin{rem} \label{rem:01-03-mup}
The expression $\mu_m$ can be computed efficiently 
modulo~$p^{\tilde{N}}$ via
\begin{equation*}
\mu_m = \frac{p^{\floor{m/p}}}{m!} 
    \sum_{k=0}^{\floor{m/p}} \frac{m!}{p^k (m-pk)! k!}
\end{equation*}
using only one $p$-adic inversion.  We observe that the summands are 
integers of size $\mathcal{O}(m \log m)$ bits, which allows us to 
avoid performing intermediate reductions modulo~$p^{\tilde{N}}$.
\end{rem}

\begin{thm} \label{thm:alphap}
Let $p \geq 3$.  Continuing with the notation set up in this chapter, 
let $u, v \in \mathbf{Z}^{n+1}$ be such that 
$x^u, x^v \in B_1 \cup \dotsb \cup B_n$ and satisfy, 
for all $i$, $p (u_i + 1) \equiv v_i + 1 \pmod{d}$. 
Then 
\begin{equation} \label{eq:alphap.0}
\alpha_{u+1,v+1} = (-p)^{u'} \prod_{i=0}^n 
    \hat{a}_i^{(p (u_i + 1) - (v_i + 1))/d} \sum_{m,r} 
    \Bigl(\frac{u_i+1}{d}\Bigr)_r p^{r - \floor{m/p}} \mu_m
\end{equation}
where $m, r \geq 0$ satisfy $p (u_i + 1) - (v_i + 1) = d (m - pr)$. 
In particular, $\alpha_{u+1, v+1} \in \mathbf{Z}_p$. 
\end{thm}

\begin{proof}
This follows from Theorem~\ref{thm:alpha}, Proposition~\ref{prop:mpr2} 
and Lemma~\ref{lem:mup}.
\end{proof}

% Description of the algorithm %%%%%%%%%%%%%%%%%%%%%%%%%%%%%%%%%%%%%%%%%%%%%%%%

\section{Description of the algorithm}

Up to this point, the double sum over $m,r$ in our expressions for 
$\alpha_{u+1,v+1}$ has been an infinite sum.  We now present a convergence 
result, which will allow us to derive a finite expression to determine 
$\alpha_{u+1,v+1}$ modulo $p^{\tilde{N}}$.

\begin{lem} \label{lem:log}
Given integers $b,c \geq 2$ and defining $x = c + \log_b c + 1$ 
we have that, for all real numbers $y \geq x$, 
\begin{equation*}
y - \log_b y \geq c.
\end{equation*}
\end{lem}

\begin{proof}
We first note that the function $y \mapsto y - \log_b y$ is increasing 
for $y \geq 2$ because it has derivative $1 - \log_b(e)/y > 0$.  Thus, it 
suffices to verify the result for $x$.  Indeed, as $c \geq 2$ we have 
that 
\begin{gather*}
\log_b c + 1 \leq c
\intertext{and hence}
c + \log_b c + 1 \leq b^{\log_b c + 1}
\intertext{Taking logarithms to base $b$ and rearranging, we obtain}
c + \log_b c + 1 - \log_b \bigl(c + \log_b c + 1\bigr) - c \geq 0
\end{gather*}
as required.
\end{proof}

\begin{prop}
Continuing with the notation set up in this chapter, in order to compute 
$\alpha_{u+1,v+1} \bmod p^{\tilde{N}}$ it suffices to restrict the inner 
sum to pairs $m,r \geq 0$ such that $m \leq M = M(\tilde{N}, p)$ where 
\begin{equation*}
M = \floor{ p^2 \biggl( \frac{\tilde{N}}{p-1} 
            + \log_p\Bigl(\frac{\tilde{N}}{p-1} + 2\Bigr) + 3 \biggr) }.
\end{equation*}
\end{prop}

\begin{proof}
This follows from~\citep[Section~6.2]{Lau04} and Lemma~\ref{lem:log}.
\end{proof}

Finally, we describe how to compute the matrix~$F$ representing 
the action of $p^{-1} F_p$ using our previous results.

\begin{prop}
The valuation of $(u'-1)! \alpha_{u+1,v+1}$ is bounded from above by 
\begin{equation*}
\ord_p((u'-1)! \alpha_{u+1,v+1}) \leq \ord_p((n-1)!) + n + (r + s)
\end{equation*}
where $r = \ord_p((n-1)!)$ and $s = (n+1) \floor{\log_p(n-1)}$. 
\end{prop}

\begin{proof}
Recall from Gerkmann~\citep[Lemma~3.3]{Gerkmann2007} that the valuations 
of the entries of the matrix~$F$ are bounded from below by $-(r+s)$. 
Thus, by Theorem~\ref{thm:01-03-diagfrob}, 
\begin{equation*}
-(r+s) \leq \ord_p((v'-1)!) + n - \ord_p((u'-1)! \alpha_{u+1,v+1})
\end{equation*}
Noting that $d v' = \sum_{i=0}^n (v_i + 1) \leq n d$ and 
hence $v' \leq n$, the result follows.
\end{proof}

This allows us to formalise the procedure for computing the 
entries of~$F$ modulo~$p^N$ in Algorithm~\ref{alg:01-03-diagfrob}.

\begin{algorithm}
\caption{Compute the matrix for $p^{-1} F_p$ on $H_{dR}^n(\mathfrak{U})$}
\label{alg:01-03-diagfrob}
\begin{algorithmic}
\vspace{1mm}
\Require Polynomial $a_0 x_0^d + \dotsb + a_n x_n^d$, 
         prime~$p$, precision~$N \geq 0$.
\Ensure  Matrix for $p^{-1} F_p$ on $H_{dR}^n(\mathfrak{U})$ modulo $p^N$.
\Procedure{DiagFrob}{$n, d, p, N, a_0, \dotsc, a_n$}
\begin{enumerate}
\item Determine the numbers 
      $C = C(n,p) = n + 2 \ord_p((n-1)!) + (n+1) \floor{\log_p(n-1)}$, 
      $\tilde{N} = N - n + 2 C$, $M = M(\tilde{N}, p)$, and 
      $R = R(\tilde{N}, p) = \floor{M/p}$.
\item Precompute the Teichm\"uller lifts $\hat{a}_0, \dotsc, \hat{a}_n$, 
      and the sequences $(d^{-r})_{r=0}^R$, and $(\mu_m)_{m=0}^{M}$ 
      to precision~$\tilde{N}$.
\item For each monomial $u = (u_0, \dotsc, u_n) \in B_1 \cup \dotsb \cup B_n$, 
      determine the unique monomial $v = (v_0, \dotsc, v_n)$ such that 
      $v_i = p (u_i + 1) - 1 \bmod{d}$.  Use the following steps to 
      compute the entry at position $(u,v)$ in the matrix $F$.
\item Compute the expression $x_1 = (-1)^{u'+v'} (v'-1)! p^n$ as an 
      exact integer and observe that \mbox{$\ord_p(x_1) \geq n$}.
\item Compute the expression $x_2 = (u' - 1)! \alpha_{u+1,v+1}$ to 
      precision~$\tilde{N}$ using Equation~\eqref{eq:alpha2.0} or 
      \eqref{eq:alphap.0} depending on whether $p$ is even or odd. 
\item Compute the inverse $x_2^{-1}$ to precision $N - n$.
\item Finally, compute the product $x_1 x_2^{-1}$ to precision~$N$.
\end{enumerate}
\EndProcedure
\end{algorithmic}
\end{algorithm}

\begin{thm} \label{thm:01-dm-03-complexity}
The time complexity of Algorithm~\ref{alg:01-03-diagfrob} is given 
by 
\begin{equation*}
p \tilde{N}^2 M\bigl(p \tilde{N} \log (p \tilde{N})\bigr)
    + d^n n \bigl( M(\log d) + (\tilde{N} + \log p) M(\tilde{N} \log p) \bigr)
\end{equation*}
where $M(-)$ denotes the complexity of integer multiplication and 
$\tilde{N}$ is $\mathcal{O}(N + n \log_p n)$.
\end{thm}

\begin{proof}
We consider each of the steps in Algorithm~\ref{alg:01-03-diagfrob}. 
We can ignore the computational effort in {Step~(i)}, but observe 
that $\tilde{N}$ is $\mathcal{O}(N + n \log_p n)$, $M$ is 
$\mathcal{O}(p \tilde{N})$, and $R$ is $\mathcal{O}(\tilde{N})$.  

In {Step~(ii)}, we compute $n+1$ Teichm\"uller lifts with an 
overall complexity of $\mathcal{O}(n (\log p) M(\tilde{N} \log p))$. 
The computation of the sequence $(d^{-r})_{r=0}^{R}$ requires 
one reduction of $d$ modulo $p^{\tilde{N}}$, one $p$-adic inversion 
and $R-1$~products to precision~$\tilde{N}$, yielding 
$\mathcal{O}\bigl(M(\log d) + (\log \log p) M(\log p) + \tilde{N} M(\tilde{N} \log p)\bigr)$. 
Finally, we carry out the computation of $(\mu_m)_{m=0}^{M}$ following 
Remark~\ref{rem:01-03-mup}, which requires time 
$\mathcal{O}\bigl(p \tilde{N}^2 M(p \tilde{N} \log(p \tilde{N}))\bigr)$.

The following {Steps~(iii) through (vii)} are executed 
$\dim H_{dR}^{n}(\mathfrak{U})$ times, where this dimension 
is $\mathcal{O}(d^n)$.
The time complexity of {Step~(iii)} is 
$\mathcal{O}\bigl(n M(\log \max\{p,d\})\bigr)$.
We observe that we can ignore {Step~(iv)}.
Step~(v) involves an $(n+1)$-fold product of series 
with~$\mathcal{O}(R)$ terms modulo~$p^{\tilde{N}}$, where each summand 
requires an absolutely bounded number of products, as well as 
$n+1$~exponentiations of Teichm\"uller lifts with exponents given 
by $d^{-1} \bigl(p (u_i+1) - (v_i+1)\bigr) < p$.  Computing the 
exponents has complexity $\mathcal{O}(M(\log \max\{p,d\}))$ and 
we note that we may ignore the update of the term 
\mbox{$\bigl((u_i+1)/d\bigr)_r$} throughout the summation 
assuming we have computed the reduction of $d \bmod{p^{\tilde{N}}}$ 
once and for all earlier.  Therefore, the complexity of each 
invocation of {Step~(v)} is 
$\mathcal{O}\bigl( n M(\log d) + n (R + \log p) M(\tilde{N} \log p)\bigr)$.  
The $p$-adic inverse in {Step~(vi)} requires time 
$\mathcal{O}\bigl((\log \log p) M(\log p) + M(\tilde{N} \log p)\bigr)$.
Finally, we can ignore the product in {Step~(vii)}.  
Thus, the aggregate time complexity of {Steps~(iii)} through {(vii)} 
is given by 
$\mathcal{O}\bigl(d^n n \bigl( M(\log d) + (\tilde{N} + \log p) M(\tilde{N} \log p) \bigr)\bigr)$.
\end{proof}

\begin{rem}
Although the current implementation of Algorithm~\ref{alg:01-03-diagfrob} 
performs very well in practical applications, its time complexity is 
quasi-cubic in the $p$-adic precision~$N$, which is \emph{not} optimal.  
Both of these aspects are due to the use of the sequence $(\mu_m)_{m=0}^{M}$ 
defined over $\mathbf{Z}_p$ instead of 
the coefficients $(\lambda_m)_{m=0}^{M}$ defined over $\mathbf{Q}_p(\pi)$. 
However, we can achieve a better time complexity by utilising fast 
composition of power series as described in~\citep[\S 9.3]{Bernstein2008}. 
This allows us to compute the composition $\exp \pi (z - z^p)$ 
modulo~$z^{M+1}$ in a number of operations in~$\mathbf{Q}_p(\pi)$ which 
is quasi-linear in~$M$.  As the bit complexity of an operation in this 
ring is quasi-linear in~$N$, this shows that the sequence 
$(\lambda_m)_{m=0}^{M}$ can be computed in time quasi-quadratic in~$N$.
\end{rem}

% Computational examples %%%%%%%%%%%%%%%%%%%%%%%%%%%%%%%%%%%%%%%%%%%%%%%%%%%%%%

\section{Computational examples} 

We conclude this chapter by presenting three examples, demonstrating the 
relevance of the improvements in practical computations.  The first two 
allow us to compare our implementations with timings report by 
Gerkmann~\citep{Gerkmann2007}, and the third example features an increase 
in various parameters.

Firstly, let us consider the surface given by 
\begin{equation*}
x_0^4 + x_1^4 + x_2^4 + x_3^4 = 0
\end{equation*}
over $\mathbf{F}_3$ and aim to compute the action of $p^{-1} F_p$ 
to precision $N = 374$.

We find that we need to compute the expression $\alpha_{u+1,v+1}$ to 
precision $\tilde{N} = 377$ and are led to compute the terms $\mu_m$ for 
$m \leq 1768$.

In~\citep[Section~7.4]{Gerkmann2007}, Gerkmann considers this example 
and reports a runtime of $45.26$~minutes.  In contrast, we are able 
to compute this matrix in $0.55$~seconds, of which $0.39$~seconds are 
spent in the precomputation of the sequence~$(\mu_m)$, an improvement 
of a factor of nearly~$5,000$.

Secondly, we consider a quintic surface over $\mathbf{F}_2$, 
\begin{equation*}
x_0^5 + x_1^5 + x_2^5 + x_3^5 = 0
\end{equation*}
and we aim to compute the matrix~$F$ to precision~$N = 355$.  This leads 
us to compute $\alpha_{u+1,v+1}$ to precision~$\tilde{N} = 370$, which 
requires us to compute $\mu_{m}$ up to $M = 1528$.

Gerkmann~\citep[Section~7.6]{Gerkmann2007} reports a runtime of 
$64.13$~minutes.  In contrast, we are able to compute this matrix 
in $0.32$~seconds, of which $0.27$~seconds are spent in the precomputation 
of the sequence~$(\mu_m)$, an improvement by a factor of over~$12,000$.

For the third and final example, we consider the sextic threefold 
over~$\mathbf{F}_{97}$ given by 
\begin{equation*}
x_0^6 + 2 x_1^6 + 3 x_2^6 + 4 x_3^6 + 5 x_4^6
\end{equation*}
and compute the matrix~$F$ to precision~$N = 500$.  This leads to the 
other parameters $\tilde{N} = 504$ and $M = 87033$.  We also observe 
that $\dim H_{dR}^{n}(\mathfrak{U}) = 520$.  The precomputation requires 
$0.01$~seconds for the sequence $d^{-r}$ and $9.67$~hours for the 
sequence~$\mu_m$.  After this, we can determine the entries of the 
Frobenius matrix in another $30.86$~seconds.

