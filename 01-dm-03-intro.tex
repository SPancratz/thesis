% Chapter:  Introduction to the deformation method  %%%%%%%%%%%%%%%%%%%%%%%%%%%

% Definitions and notation %%%%%%%%%%%%%%%%%%%%%%%%%%%%%%%%%%%%%%%%%%%%%%%%%%%%

\section{Definitions and notation}

In this section we present the problem that we are concerned 
with in this part of the thesis and carefully set up the notation 
that we are using throughout.  We begin by giving a brief overview.

The aim of the deformation method as introduced by 
Lauder~\citep{Lau04} following a paper by Dwork~\citep{Dwork62b} 
is to compute the zeta function of a smooth 
hypersurface, which in hindsight we call $X_{t_1}$, in projective 
$n$-space over a finite field of characteristic~$p$.  This is 
facilitated by computing 
the action of the Frobenius morphism on the associated rigid cohomology 
spaces $H_{rig}^{\bullet}(X_{t_1})$.  The main idea is to embed this 
hypersurface in a family of smooth projective hypersurfaces~$X/S$ 
defined over a finite field~$k$, where $S \subset A_k^1(t)$, containing 
a fibre~$X_{t_0}$ for which it is easier to compute the Frobenius action. 
Lifting the family~$X/S$ to a family $\mathfrak{X}/\mathfrak{S}$ in 
characteristic zero, we can utilise a certain $p$-adic differential 
equation involving the Frobenius matrix and the algebraic Gauss--Manin 
connection.  Moreover, due to a comparison theorem of Baldassarri and 
Chiarellotto~\citep{BalChi94}, in this 
part of the algorithm we can work with the algebraic de~Rham cohomology 
spaces~$H_{dR}^{\bullet}(\mathfrak{X}/\mathfrak{S})$, which are 
computationally more accessible.  Solving this differential equation 
and using the Frobenius action on~$\mathfrak{X}_{t_0}$ as the initial 
condition, by an appropriate evaluation we can compute the Frobenius matrix 
for~$X_{t_1}$ and hence recover its zeta function.

For the remaining part of this thesis, we restrict ourselves to the 
case where the family $\mathfrak{X}/\mathfrak{S}$ is defined over a 
prime field, $t_0 = 0$, and the initial fibre $X_0$ is diagonal.  
Specifically, we establish the following set-up.

\begin{notation} \label{not:01-03-main}
Let $p$ be a prime number, $q$ a power of $p$, and let $\mathbf{Q}_q$ 
denote the unique unramified extension of $\mathbf{Q}_p$ with residue 
field~$\mathbf{F}_q$ and $\mathbf{Z}_q$ its ring of integers.  
We consider two smooth $\mathbf{Z}_q$-schemes $\mathfrak{X}$ and $\mathfrak{S}$ 
together with a smooth projective morphism $\mathfrak{X} \to \mathfrak{S}$, 
where $\mathfrak{X}$ is a family of hypersurfaces 
and $\mathfrak{S} \subset \mathbf{A}_{\mathbf{Z}_q}^1(t)$ is an affine, open 
subset containing the point~$0$ and the Teichm\"uller lift~$\hat{t}_1$ of 
$t_1 \in \mathbf{F}_q$.  
We assume that this family is defined by a homogenous polynomial 
$P \in \mathbf{Z}[t][x_0,\dotsc,x_n]$ over the integers of degree~$d$, 
where $n \geq 1$ and $d \geq 1$.  
Moreover, we require that at $t = 0$, the polynomial $P_0(x_0,\dotsc,x_n)$ 
is diagonal.  
We also define complements $\mathfrak{U}$ and $U$ by 
$\mathbf{P}_{\mathbf{Z}_q}^n \times \mathfrak{S} - \mathfrak{X}$ 
and $\mathbf{P}_{\mathbf{F}_p}^n \times S - X$, respectively, 
where $S \subset \mathbf{A}_{\mathbf{F}_q}^1(t)$.  
\end{notation}

\begin{notation}
Furthermore, as an artefact of our particular choice of basis for 
$H_{dR}^n(\mathfrak{U}/\mathfrak{S})$ in Definition~\ref{defn:01-05-basis}, 
we assume that $d \geq 2$ whenever $n$ is odd and $d \geq 3$ whenever $n$ 
is even.
\end{notation}

\begin{rem}
It follows that all fibres $X_t$ are non-singular varieties, and by 
considering the fibre~$X_0$ we see that our assumptions imply $p \nmid d$.
\end{rem}

% Outline of the algorithm %%%%%%%%%%%%%%%%%%%%%%%%%%%%%%%%%%%%%%%%%%%%%%%%%%%%

\section{Outline of the algorithm}

The deformation algorithm is outlined in Figure~\ref{fig:01-03-deformation}.

\begin{figure}[h]
\centering
\begin{tikzpicture}[node distance = 24mm, auto]

    % Define block styles
    \tikzstyle{block} = [rectangle, draw, 
        text centered, rounded corners, minimum height = 6mm]
    \tikzstyle{empty} = [rectangle]
    \tikzstyle{line} = [draw, -latex']

    % Place nodes
    \node [block, text width = 100mm] (input) 
        {Polynomial $P \in \mathbf{Z}[t][x_0, \dotsc, x_n]$; 
        point $t_1 \in \mathbf{F}_q$ with $q = p^a$};

    \node [block, below left=14.6mm and -30mm of input, text width = 48mm] (diag) 
        {Matrix $F_0$ over $\mathbf{Q}_p \bmod p^{N_3}$ for $p^{-1} F_p$ 
        on $H_{dR}^n(\mathfrak{U}_0)$};

    \node [block, below right=6mm and -50mm of input, text width = 58mm] (gm) 
        {Connection matrix $M$ over $\mathbf{Q}(t)$ 
        with denominator $r \in \mathbf{Z}[t]$};

    \node [block, below=6mm of gm, text width = 82mm] (local) 
        {Solution $C(t)$ in $\mathbf{Q}_p[[t]] \bmod {(p^{N_2}, t^{K})}$ of the 
        differential equation \mbox{$\bigl(\tfrac{d}{dt} + M\bigr) C = 0$}, 
        and $C(t^p)^{-1}$};

    \node [block, below=42mm of input, text width = 126mm] (series)
        {Matrix $F_t = C(t) F_0 C(t^p)^{-1}$ in 
        $\mathbf{Q}_p[[t]] \bmod {(p^{N_1}, t^{K})}$ for $p^{-1} F_p$ 
        on $H_{dR}^n(\mathfrak{U}_t)$};

    \node [block, below=6mm of series, text width = 76mm] (poly) 
        {Matrix $G = r^m F_t$ over $\mathbf{Q}_p[t] \bmod {(p^{N_1})}$};

    \node [block, below=6mm of poly, text width = 114mm] (eval) 
        {Matrix $F_1 = r(\hat{t}_1)^{-m} G(\hat{t}_1)$ over $\mathbf{Q}_q \bmod p^{N_1}$
        for $p^{-1} F_p$ on $H_{dR}^n(\mathfrak{U}_1)$}; 

    \node [block, below=6mm of eval, text width = 118mm] (qpower) 
        {Matrix $F_{q,1} = F_1 F_1^{\sigma} \dotsm F_1^{\sigma^{a-1}}$ 
        over $\mathbf{Q}_q \bmod {p^{N_1}}$ for $q^{-1} F_q$ on $H_{dR}^n(\mathfrak{U}_1)$};

    \node [block, below=6mm of qpower, text width = 72mm] (zeta) 
        {Polynomial $p(T) = \det(1 - T F_{q,1})$ over $\mathbf{Z}$}; 

    % Draw edges
    \path [line] (input) -- (diag);
    \path [line] (input) -- (gm);
    \path [line] (gm) -- (local);
    \path [line] (diag) -- (series);
    \path [line] (local) -- (series);
    \path [line] (series) -- (poly);
    \path [line] (poly) -- (eval);
    \path [line] (eval) -- (qpower);
    \path [line] (qpower) -- (zeta);

\end{tikzpicture}
\caption{Flowchart describing the deformation algorithm}
\label{fig:01-03-deformation}
\end{figure}
