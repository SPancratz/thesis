%%%%%%%%%%%%%%%%%%%%%%%%%%%%%%%%%%%%%%%%%%%%%%%%%%%%%%%%%%%%%%%%%%%%%%%%%%%%%%%
% Introduction                                                                %
%%%%%%%%%%%%%%%%%%%%%%%%%%%%%%%%%%%%%%%%%%%%%%%%%%%%%%%%%%%%%%%%%%%%%%%%%%%%%%%

\chapter{Introduction}
\label{ch:02-Intro}

In this part of the thesis, we describe various strategies to solve the 
problem of polynomial radix conversion.  Two approaches, the naive 
repeated division by the radix and the divide and conquer strategies, are 
well known, see e.g.~\citep[\S 9.2]{GathenGerhard2003}.  We present a 
practical improvement to the latter approach based on the use of 
precomputed power series inverses.  A number theoretical application is 
the fibration method for point counting, in current implementations of 
which the runtime in practice is dominated by radix conversions.

In its general form, the problem of polynomial radix conversion 
can be stated as follows:

\begin{prob} \label{prob:exact}
Let $R$ be a ring\footnotemark and suppose that a polynomial $f \in R[X]$ is 
presented in the usual monomial basis as $f(X) = \sum_{i=0}^{n} a_i X^i$.
Suppose that $r \in R[X]$ is another polynomial, called the \emph{radix}, 
such that $m = \deg(r) \geq 1$ and the leading coefficient of~$r$ is invertible 
in $R$.  Compute polynomials $b_0, \dotsc, b_{\ell} \in R[X]$ in terms of 
the monomial basis such that 
\begin{equation*}
f = b_0 + b_1 r + \dotsb b_{\ell} r^{\ell}
\end{equation*}
where necessarily $\ell = \floor{n / m}$.
\end{prob}

\footnotetext{The term \emph{ring} here means commutative ring with 
multiplicative identity.}

The existence and uniqueness of the sequence of polynomials $b_0, \dotsc, b_{\ell}$ 
follows easily from the properties of Euclidean division;  an explicit algorithm is 
presented in Section~\ref{sec:EuclideanDivision}.

To motivate the work in this context, we point out that Lauder's 
fibration algorithm for counting points on smooth projective varieties 
in practice heavily benefits from a fast implementation for this problem 
of radix conversions, see Lauder~\citep[\S 6.5.2]{Lauder2006} or 
Walker~\citep[\S 3.2.2]{Walker2009}.  In both these cases, the authors 
utilise the divide and conquer approach in their computations.  In the 
fibration algorithm, one has to convert the power series of each 
entry in a local expansion of a Frobenius matrix.  That is to say, 
the number of instances with the same radix polynomial is quadratic 
in the dimension of the middle-dimensional rigid cohomology space, 
which motivates our approach using precomputed power series inverses.

In this context, we consider our ring $R$ to be the field $\mathbf{Q}_p$, 
although in practice computations will only be carried out to finite 
$p$-adic precision.

\begin{prob} \label{prob:padic}
Given two polynomials $f, r \in \mathbf{Q}_p[X]$ such that 
$r \neq 0$ and the valuation of the leading coefficient of 
$r$ is equal to $\ord_p(r)$, find a representation of $f$ 
in the form
\begin{equation*}
f = b_0 + b_1 r + \dotsb + b_{\ell} r^{\ell}.
\end{equation*}
\end{prob}

In the remaining part of this section, we describe how to 
reduce Problem~\ref{prob:padic} to the original 
Problem~\ref{prob:exact} and carry out the $p$-adic precision 
analysis.

Write $f = p^{\ord_p(f)} g$ and $r = p^{\ord_p(r)} s$ 
with two polynomials $g$ and $s$ in $\mathbf{Z}_p[X]$ 
each containing at least one coefficient which is a unit 
in $\mathbf{Z}_p$.  The above equation then becomes
\begin{equation*}
g = p^{-\ord_p(f)} b_0 + p^{-\ord_p(f) + \ord_p(r)} b_1 s + \dotsb + p^{-\ord_p(f) + \ell \ord_p(r)} b_{\ell} s^{\ell}
\end{equation*}
and we also note that the leading coefficient of $s$ 
is a unit in $\mathbf{Z}_p$.
In order to solve Problem~\ref{prob:padic}, we first solve the above 
for the polynomials $g$ and $s$ over $\mathbf{Z}_p$, obtaining a 
sequence of polynomials $c_0, \dotsc, c_{\ell}$ in $\mathbf{Z}_p[X]$ 
such that 
\begin{equation*}
g = c_0 + c_1 s + \dotsb + c_{\ell} s^{\ell}.
\end{equation*}
We can then recover the sequence $b_0, \dotsc, b_{\ell}$ via 
\begin{equation*}
b_i = p^{\ord_p(f) - i \ord_p(r)} c_i.
\end{equation*}
We observe that, in order to determine the polynomial $b_i$ 
modulo $p^N$ we need to determine the polynomial $c_i$ 
modulo $p^{N + i \ord_p(r) - \ord_p(f)}$.  Maximising this 
over the whole sequence of polynomials, we see that we need 
to determine the polynomials $c_0, \dotsc, c_{\ell}$ modulo 
$p^{N + \eta}$ where
\begin{equation*}
\eta = - \ord_p(f) + \max\{\ell \ord_p(r), 0\}.
\end{equation*}
Thus, we require that the polynomial $f \in \mathbf{Q}_p[X]$ 
is provided to precision 
\begin{equation*}
N + \eta + \ord_p(f) = N + \max\{\ell \ord_p(r), 0\}.
\end{equation*}
and that $r \in \mathbf{Q}_p[X]$ is provided to precision 
\begin{equation*}
N + \eta + \ord_p(r) = N - \ord_p(f) + \ord_p(r) + \max\{\ell \ord_p(r), 0\}.
\end{equation*}

%%%%%%%%%%%%%%%%%%%%%%%%%%%%%%%%%%%%%%%%%%%%%%%%%%%%%%%%%%%%%%%%%%%%%%%%%%%%%%%
% Algorithms                                                                  %
%%%%%%%%%%%%%%%%%%%%%%%%%%%%%%%%%%%%%%%%%%%%%%%%%%%%%%%%%%%%%%%%%%%%%%%%%%%%%%%

\chapter{Algorithms}
\label{ch:02-Algorithms}

% Euclidean division and complexity results  %%%%%%%%%%%%%%%%%%%%%%%%%%%%%%%%%%

\section{Euclidean division and complexity results}
\label{sec:EuclideanDivision}

In this section we collect a few well known results for later 
reference.

\begin{prop}
Let $R$ be a ring and suppose that $f, r \in R[X]$ with the 
leading coefficient of $r$ a unit in $R$.  Then there exist 
unique polynomials $q, s \in R[X]$ such that $f = q r + s$ 
with $s = 0$ or $\deg(s) < \deg(r)$.
\end{prop}

This result is well known in the context of polynomial rings 
over fields.  The usual proof of this result in $\mathbf{Q}[X]$ 
applies upon observing that only the inverse of the leading 
coefficient is required in the division process.

\begin{prop} \label{prop:mul}
There exists an algorithm returning the degree~$2n$ product of two 
polynomials of degree~$n$ in $\mathcal{O}(n \log n \log \log n)$ 
ring operations.
\end{prop}

Such an algorithm can be constructed based on Fast Fourier Transformation 
techniques, see for example~\citep[\S Theorem~8.23]{GathenGerhard2003} or 
\citep[\S 4]{Bernstein2008}.

\begin{prop} \label{prop:div1}
There exists an algorithm returning the degree~$n$ quotient 
in a Euclidean division with a dividend of degree~$2n$ 
and a divisor of degree~$n$ in $\mathcal{O}(n \log n \log \log n)$ 
ring operations.
\end{prop}

This result can be derived from the previous one by exhibiting an algorithm 
for Euclidean division with the same asymptotic time requirements as 
multiplication, see for example~\citep[\S 17]{Bernstein2008}.

\begin{cor} \label{cor:div2}
In the unbalanced case for a dividend of degree~$n$ and 
a divisor of degree~$m$, where $m \leq n$, there exists an 
algorithm that returns the degree~\mbox{$n-m$} quotient in 
$\mathcal{O}(n \log m \log \log m)$ ring operations.
\end{cor}

This can be achieved by splitting the dividend into parts of degree~$2m$ 
and applying the previous algorithm in these balanced divisions, carefully 
combining the results.

% Repeated division by radix  %%%%%%%%%%%%%%%%%%%%%%%%%%%%%%%%%%%%%%%%%%%%%%%%%

\section{Repeated division by the radix}

The naive solution to the radix conversion problem consists of 
obtaining the sequence of polynomials $b_0, \dotsc, b_{\ell}$ 
by performing $\ell$ Euclidean divisions with remainder.  Letting 
{\sc DivRem} denote a routine that performs Euclidean division, 
this approach can be formalised as follows:

\begin{algorithm}[H]
\caption{Repeated division by the radix}
\label{alg:RepeatedDivision}
\begin{algorithmic}
\vspace{1mm}
\Procedure{RadixConversion}{$b$, $f$, $r$}
\State $f_0 \gets f$
\For{$i \gets 0, \dotsc, \ell-1$}
\State $f_{i+1}, b_i \gets \Call{Divrem}{f_i, r}$
\EndFor
\State $b_{\ell} \gets f_{\ell}$
\EndProcedure
\end{algorithmic}
\end{algorithm}

We can repeatedly apply the previous result on the complexity 
of unbalanced divisions, Corollary~\ref{cor:div2}, with fixed 
divisor~$r$.  The conclusion is that the complexity of this 
algorithm is $\mathcal{O} \bigl( n^2 m^{-1} \log m \log \log m \bigr)$.

% Divide and conquer  %%%%%%%%%%%%%%%%%%%%%%%%%%%%%%%%%%%%%%%%%%%%%%%%%%%%%%%%%

\section{Divide and conquer}

In this section we present another algorithm, which achieves 
a better time complexity.  The driving idea is that we divide 
the polynomial $f$ into two halves and then deal with each half 
separately.  This recursive idea is formalised in 
Algorithm~\ref{alg:DivConquer}.

\begin{algorithm}[H]
\caption{Recursive divide and conquer}
\label{alg:DivConquer}
\begin{algorithmic}
\vspace{1mm}
\Procedure{RCRecursive}{$b$, $f$, $r$, $k$}
\State $N \gets \floor{\deg(f) / \deg(r)}$
\If{$N = 0$}
\State $b_k \gets f$
\Else
\State $e \gets \ceil{N/2}$
\State $Q, S \gets \Call{Divrem}{f, r^{e}}$
\State $\Call{RCRecursive}{b, q, r, k + e}$
\State $\Call{RCRecursive}{b, s, r, k}$
\EndIf
\EndProcedure
\Procedure{RadixConversion}{$b$, $f$, $r$}
\State $\Call{RCRecursive}{b, f, r, 0}$
\EndProcedure
\end{algorithmic}
\end{algorithm}

Before we proceed with the analysis of this algorithm, we make the 
following assumption, which will allow for a cleaner presentation, 
analysis and implementation of the algorithm:
\begin{notation}
Assume that $n = 2^k m - 1$ for some $k \geq 1$.
\end{notation}

\begin{rem}
Since all Euclidean divisions then involve a dividend of 
length $2^{i} m$ and a divisor of length $2^{i-1} m + 1$, 
where $i = k, \dotsc, 1$, we can more efficiently re-use 
precomputed data such as powers and Newton inverses of 
the radix~$r$.

Moreover, this assumption can easily be incorporated in an 
implementation by choosing the least integer $k$ such that 
$n \leq 2^k m - 1$ and then zero-padding the top coefficients 
of~$f$.
\end{rem}

\begin{rem}
The runtime of an implementation which enforces the above 
assumption by zero-padding as a function of $n$ is 
not smooth, featuring jumps as $n$ changes from 
$2^k m - 1$ to $2^k m$.  Although this does not 
affect the asymptotic behaviour, it is undesirable in 
practice.  However, it is possible to combine the two 
approaches, for example, by following Algorithm~\ref{alg:DivConquer} 
at the first level and only pad the lengths to powers of~$2$ 
as necessary at the subsequent levels.
\end{rem}

In order to determine the complexity of Algorithm~\ref{alg:DivConquer} 
we count the number of multiplications as well as the number of 
Euclidean divisions of each size.

First, we note that we can precompute $r^{2^i}$ for $i = 1, \dotsc, k-1$ 
before proceeding with the Euclidean divisions, by computing $k-1$ squares 
involving input polynomials of length $2^{i} m + 1$ for 
$i = 0, \dotsc, k-2$.  This part of the algorithm has complexity 
\begin{equation*}
\mathcal{O} \biggl( \sum_{i=0}^{k-2} \bigl(2^{i} m\bigr) 
    \log \bigl(2^{i} m\bigr) \log \log \bigl(2^{i} m\bigr) \biggr)
\end{equation*}
which can be simplified to $\mathcal{O}(n \log n \log \log n)$.
By inspection, there are $2^i$ divisions with dividend of length 
$2^{k-i} m$ and divisor of length $2^{k-1-i} m + 1$
for $i = 0, \dotsc, k-1$.  Thus, the complexity is 
\begin{equation*}
\mathcal{O} \biggl( \sum_{i=0}^{k-1} \bigl(2^{k-1-i} m + 1\bigr) 
    \log \bigl(2^{k-1-i} m + 1\bigr) 
    \log \log \bigl(2^{k-1-i} m + 1\bigr) \biggr)
\end{equation*}
which can be simplified to $\mathcal{O} \bigl( n \log n \log \log n \bigr)$.

Thus, the overall complexity of Algorithm~\ref{alg:DivConquer} is 
$\mathcal{O}(n \log n \log \log n)$.

% Euclidean division with precomputed inverses  %%%%%%%%%%%%%%%%%%%%%%%%%%%%%%%

\section{Euclidean division with precomputed inverses}

Since the first part of the divide and conquer algorithm, the 
precomputation of the powers of the radix, already forces a 
time complexity of at least $\mathcal{O} (n \log n \log \log n)$ 
ring operations we cannot hope for an asymptotic improvement.

However, by precomputing further data the cost of the repeated 
Euclidean divisions can be reduced significantly.  This is 
particularly valuable in the context of the fibration algorithm 
where the operation of radix conversion is called repeatedly 
for varying polynomials $f$ of similar degree but with an invariant 
radix~$r$.  The approach presented in this section precomputes 
power series inverses, allowing us to reduce the cost of a 
Euclidean division to two short products, that is, routines
that only require the lower half of the coefficients of a degree~$2n$ 
product of two degree~$n$ factors.

We now temporarily abandon the earlier notation to explain 
a procedure for Euclidean division with precomputed inverses. 
The algorithm, which is presented as Algorithm~\ref{alg:NewtonDivision}, 
assumes that we have two subroutines at our disposal.  Firstly, 
a routine $\Call{Rev}{f,n}$ that returns $X^{\deg(f)} f(X^{-1})$ 
modulo~$X^n$, that is, a routine that simply reverses the top~$n$ 
coefficients of $f$, possibly including zero padding if $f$ has degree 
less than $n - 1$.  Secondly, a routine $\Call{Inv}{f,n}$ that returns 
the power series inverse of~$f$ modulo~$X^n$.

\begin{algorithm}[H]
\caption{Newton division}
\label{alg:NewtonDivision}
\begin{algorithmic}
\vspace{1mm}
\Require Polynomials $a, b \in R[X]$ with $\deg(a) \geq \deg(b) > 0$ 
         and the leading coefficient of $b$ invertible.
\Ensure  Polynomials $q, r \in R[X]$ such that $a = qb + r$ with $r = 0$ 
         or $\deg(r) < \deg(b)$.
\Procedure{DivRem}{$q$, $r$, $a$, $b$}
\State  $n \gets \deg(a) - \deg(b) + 1$
\State  $a_r \gets \Call{Rev}{a,n}$
\State  $b_r \gets \Call{Rev}{b,n}$
\State  ${b_r}^{-1} \gets \Call{Inv}{b_r,n}$
\State  $q_r \gets a_r {b_r}^{-1} \bmod X^n$
\State  $q \gets \Call{Rev}{q_r, n}$
\State{$r \gets a - q b$}
\State \textbf{return} $q, r$
\EndProcedure
\end{algorithmic}
\end{algorithm}

\begin{rem}
The properties of Euclidean division imply that $\deg(r) < \deg(b)$, 
whenever $r$ is non-zero, which in turn implies that the top 
$\deg(a) - \deg(b) + 1$ coefficients of $a$ and $q b$ 
coincide.  Thus, in order to compute $r$ we do \emph{not} 
need to compute the full product $q b$.  It suffices to 
compute the product modulo $X^{\deg(b)}$.
\end{rem}

\begin{thm}
Algorithm~\ref{alg:NewtonDivision} correctly determines 
the quotient and remainder of the Euclidean division of 
$a$ upon $b$.
\end{thm}

\begin{proof}
Let $v_0$ and $v_{\infty}$ denote the valuations on $R(X)$ at 
$0$ and $\infty$, respectively.  Note that, for $f \in R[X]$, 
\begin{equation*}
v_{\infty}(f) = - \deg(f).
\end{equation*}
We observe that $q, r \in R[X]$ such that $a = qb + r$ are 
the output of the Euclidean division if and only if 
\begin{equation*}
v_{\infty} (a - qb) > v_{\infty} (b).
\end{equation*}
Let $\tilde{b}^{-1}$ denote an approximation to $b^{-1}$ 
for $v_{\infty}$ and set $q = a \tilde{b}^{-1}$.  Then 
\begin{equation*}
a - qb = ab \bigl(b^{-1} - \tilde{b}^{-1}\bigr),
\end{equation*}
and it suffices to have an approximation such that 
\mbox{$v_{\infty} \bigl( b^{-1} - \tilde{b}^{-1} \bigr) > - v_{\infty}(a)$}.

We now relate this to the reverse polynomials.  Under the map 
$X \mapsto X^{-1}$ we have that $v_{\infty}$ maps to $v_0$, 
and $b$ maps to $X^{v_{\infty}(b)} b_r$, where $b_r$ denotes 
the reverse polynomial to $b$.  Thus, in order to compute 
$\tilde{b}^{-1}$ such that 
\mbox{$v_{\infty} \bigl( b^{-1} - \tilde{b}^{-1} \bigr) > - v_{\infty}(a)$} 
we have to compute an approximation $\tilde{b}_r^{-1}$ to 
$b_r^{-1}$ for $v_0$ such that 
\begin{gather*}
v_0 \bigl( X^{-v_{\infty}(b)} \bigl(b_r^{-1} - \tilde{b}_r^{-1} \bigr) \bigr) > - v_{\infty} (a) \\
\intertext{or, equivalently, }
v_0 \bigl(b_r^{-1} - \tilde{b}_r^{-1}\bigr) > v_{\infty}(b) - v_{\infty}(a) = \deg(a) - \deg(b), 
\end{gather*}
as required.
\end{proof}

\begin{rem}
In the notation of Algorithm~\ref{alg:NewtonDivision}, we observe 
that if the power series inverse~$b_r^{-1}$ is part of the input, 
the procedure reduces the Euclidean division to two short products, 
one subtraction, and one reversal.
\end{rem}

\begin{rem}
In Algorithm~\ref{alg:DivConquer}, all short products involve either 
a power~$r^{2^i}$ of $r$, where $i = 0, \dotsc, k-1$ or the power series 
inverse of such a power.  Thus, a further practical improvement might 
be possible if the underlying arithmetic software packages of an 
implementation allow for re-using data relevant to only one of the 
factors in a polynomial multiplication.
\end{rem}

%%%%%%%%%%%%%%%%%%%%%%%%%%%%%%%%%%%%%%%%%%%%%%%%%%%%%%%%%%%%%%%%%%%%%%%%%%%%%%%
% Examples                                                                    %
%%%%%%%%%%%%%%%%%%%%%%%%%%%%%%%%%%%%%%%%%%%%%%%%%%%%%%%%%%%%%%%%%%%%%%%%%%%%%%%

\chapter{Examples}
\label{ch:02-Exmp}

We have implemented a routine for radix conversion as described 
in the previous chapter in the C~language as part of the open-source 
library FLINT~\citep{FLINT}.  Specifically, the routine is contained 
in the module for polynomial arithmetic over $\mathbf{Z}/n\mathbf{Z}$ 
as the function {\tt{fmpz\_mod\_poly\_radix()}}.  The timings that we 
present for our routines were obtained on the departmental machine 
{\tt{wolverine.maths.ox.ac.uk}}, featuring \mbox{Intel Xeon} processors 
running at~2.93GHz.

We consider an example due to Lauder~\citep[Example~9.1]{Lauder2006}.  
Specifically, we consider the toric compactification of the smooth 
surface given by 
\begin{equation*}
y^2 = x^3 + (4 t^4 + 5 t^3) x + 
    (t^{13} + 6 t^{12} + 5 t^{10} + 
    8 t^9 + 8 t^8 + 5 t^5 + t^4 + 5 t^3 + t^2 + 1)
\end{equation*}
over $\mathbf{F}_{17}$.  Lauder reports an overall runtime 
of $23$~hours and $13$~minutes, pointing out that over half the 
time was spent in the radix conversion routine.

In the notation used by Lauder, we reestablish the parameters used 
in the computation as $g = 1$, $N_3 = 18$, $x_{fin} = 476$ and 
$N_{2,fin} = 26$.  Moreover, writing the equation defining the surface 
as $y^2 = x^3 + A(t) x + B(t)$ we compute the resultant as 
$4 A(t)^3 + 27 B(t)^2$, which is a polynomial of degree~$26$.  As the 
$r(t)$-adic precision is given by $x_{fin}$, we find that the $t$-adic 
precision is~$12,376$.

In order to establish a comparison of Lauder's computation with the 
routine we implemented, we perform a radix reduction of a random, 
dense polynomial of degree~$12,376$ with coefficients 
in~$\mathbf{Z}/17^{26} \mathbf{Z}$.  The precomputation step of our 
routine, which computes powers of $r(t)$ and their power series 
inverses, requires $0.06$~seconds.  The actual conversion step for a 
single polynomial requires $0.14$~seconds.  As the relative Frobenius 
matrix in Lauder's example is a $2 \times 2$~matrix of power series, 
the relevant figure amounts to $0.06 + 4 \times 0.14 = 0.62$ seconds. 
This is an improvement over the timings reported by Lauder by a 
factor of at least~$67,403$.

