% Chapter:  Constructing the Gauss--Manin connection  %%%%%%%%%%%%%%%%%%%%%%%%%

In this chapter we return to the relative situation of a family of 
smooth projective hypersurfaces.  Specifically, recall that given 
a family $X$ of smooth projective hypersurfaces over a finite field~$k$ 
we consider a lift $\mathfrak{X}$ in $\mathbf{P}^n(K) \times \mathbf{A}^1(K)$ 
defined by $P \in K[t][x_0, \dotsc, x_n]$ where $K$ is a field of 
characteristic zero.  The goal of this chapter is to 
describe a practical routine for computing the action of the Gauss--Manin 
connection, viewed as a linear map on $H_{dR}^n(\mathfrak{U})$ where 
$\mathfrak{U}$ is the complement of $\mathfrak{X}$.

The material in this chapter is arranged as follows.  We begin by introducing 
the concept of the Gauss--Manin connection in a scheme-theoretic set-up 
following Katz and Oda~\citep{KatzOda1968}.  Specialising to the case of 
hypersurfaces, we then follow Abbott, Kedlaya, and 
Roe~\citep{AbbottKedlayaRoe2006} in the description of arithmetic 
in algebraic de~Rham cohomology.  This leads us a complete description of the 
algorithm, including critical details on sparse matrix techniques used to 
accelerate the computations.  Finally, we present computional examples, 
comparing the runtimes achieved by our approach to previous work where 
possible.

% Scheme-theoretic introduction  %%%%%%%%%%%%%%%%%%%%%%%%%%%%%%%%%%%%%%%%%%%%%%

\section{Scheme-theoretic introduction}

In this section, we give a brief introduction to the Gauss--Manin connection 
in the algebraic sense, following Katz and Oda~\citep{KatzOda1968}.  
We consider two smooth schemes $\mathfrak{X}$ and $\mathfrak{S}$ over a 
field~$K$ of characteristic zero together with a smooth $K$-morphism 
$\pi \colon \mathfrak{X} \to \mathfrak{S}$.

\begin{defn}
A \emph{connection} on a quasi-coherent $\mathcal{O}_{\mathfrak{S}}$-module 
$\mathcal{E}$ is defined to be a homomorphism~$\rho$ of abelian sheaves
\begin{equation*}
\rho \colon \mathcal{E} \to 
    \Omega_{\mathfrak{S}/K}^1 \otimes_{\mathcal{O}_{\mathfrak{S}}} \mathcal{E}
\end{equation*}
such that
\begin{equation*}
\rho(s e) = s \rho(e) + ds \otimes e
\end{equation*}
for sections $s$ and $e$ of $\mathcal{O}_{\mathfrak{S}}$ and $\mathcal{E}$, 
respectively, over an open subset of~$\mathfrak{S}$.
\end{defn}

\begin{rem}
We observe that this can be extended to the whole de~Rham complex as follows. 
For every $i \in \N$, given $\omega \in \Omega_{\mathfrak{S}/K}^i$ and 
$e \in \mathcal{E}$ let $\omega \wedge \rho(e)$ denote the image of 
$\omega \otimes \rho(e)$ under the map
\begin{equation*}
\Omega_{\mathfrak{S}/K}^i \otimes_{\mathcal{O}_{\mathfrak{S}}} 
    \bigl(\Omega_{\mathfrak{S}/K}^1 \otimes_{\mathcal{O}_{\mathfrak{S}}} \mathcal{E}\bigr) 
    \to \Omega_{\mathfrak{S}/K}^{i+1} \otimes_{\mathcal{O}_{\mathfrak{S}}} \mathcal{E}, \quad 
\omega \otimes (\tau \otimes e) \mapsto (\omega \wedge \tau) \otimes e.
\end{equation*}
We then obtain a homomorphism of abelian sheaves $\rho_i$ given by 
\begin{equation*}
\rho_i \colon \Omega_{\mathfrak{S}/K}^i \otimes_{\mathcal{O}_{\mathfrak{S}}} \mathcal{E} 
    \to \Omega_{\mathfrak{S}/K}^{i+1} \otimes_{\mathcal{O}_{\mathfrak{S}}} \mathcal{E}, \quad 
    \omega \otimes e \mapsto d \omega \otimes e + (-1)^i \omega \wedge \rho(e).
\end{equation*}
\end{rem}

\begin{defn}
The complex $\Omega_{\mathfrak{X}/K}^{\bullet}$ admits a decreasing filtration given by 
\begin{equation*}
F^j = 
    \fIm\Bigl( \Omega_{\mathfrak{X}/K}^{\bullet - j} \otimes_{\mathcal{O}_{\mathfrak{X}}} 
    \pi^* \bigl(\Omega_{\mathfrak{S}/K}^j\bigr) \to \Omega_{\mathfrak{X}/K}^{\bullet} \Bigr).
\end{equation*}
Forming a spectral sequence $\{E_r, d_r\}_{r \geq 0}$ as in Griffiths and 
Harris~\citep[\S 3.5, p.\ 440]{GriffithsHarris1978}, we find that 
\begin{equation*}
E_1^{i,j} = 
    \Omega_{\mathfrak{S}/K}^i \otimes_{\mathcal{O}_{\mathfrak{S}}} 
    \mathcal{H}_{dR}^j(\mathfrak{X}/\mathfrak{S}).
\end{equation*}
The \emph{(algebraic) Gauss--Manin connection} $\nabla_{GM}$ is now defined as 
the differential $d_1^{0,j}$, i.e., 
\begin{equation*}
\nabla_{GM} = 
    d_1^{0,j} \colon \mathcal{H}_{dR}^j(\mathfrak{X}/\mathfrak{S}) \to 
    \Omega_{\mathfrak{S}/K}^1 \otimes_{\mathcal{O}_{\mathfrak{S}}} 
    \mathcal{H}_{dR}^j(\mathfrak{X}/\mathfrak{S}).
\end{equation*}
\end{defn}

\begin{rem}
From the description by Katz and Oda~\citep{KatzOda1968} and following the 
practical description by Kedlaya~\citep{Kedlaya2008}, we can describe the 
action of the Gauss--Manin connection further in the case when $\mathfrak{S}$ 
is affine, say $\mathfrak{S} = \Spec(A)$.  In this case, we may apply the 
global section functor $\Gamma(\mathfrak{S}, -)$ throughout and consider
\begin{equation*}
\nabla_{GM} \colon H_{dR}^j(\mathfrak{X}/\mathfrak{S}) \to 
    \Omega_{A/K}^1 \otimes_{A} H_{dR}^j(\mathfrak{X}/\mathfrak{S}).
\end{equation*}
The action of $\nabla_{GM}$ can be computed as follows.  Given 
$\omega \in H_{dR}^j(\mathfrak{X}/\mathfrak{S})$, arbitrarily lift this to 
$\tilde{\omega} \in \Omega_{\mathfrak{X}/K}^j$.  Computing the 
exterior derivative, which decomposes as 
$d_{\mathfrak{X}/K} = d_{\mathfrak{S}} + d_{\mathfrak{X}/\mathfrak{S}}$, 
the image of $\omega$ under the Gauss--Manin connection is then given as 
the projection of $d_{\mathfrak{X}/K}(\tilde{\omega})$ onto 
$\Omega_{A/K}^1 \otimes_{A} H_{dR}^j(\mathfrak{X}/\mathfrak{S})$.
\end{rem}

% Hypersurfaces in projective space  %%%%%%%%%%%%%%%%%%%%%%%%%%%%%%%%%%%%%%%%%%

\section{Hypersurfaces in projective space}
\label{sec:01-05-hypersurfaces}

In this section we make the results from the previous sections concrete, 
specialising to the case of hypersurfaces in projective space.  In the 
development of a computationally feasible description of the Gauss--Manin 
connection in this situation, we follow Abbott, Kedlaya and 
Roe~\citep[\S~3.2]{AbbottKedlayaRoe2006} when computing in 
de~Rham cohomology and Kedlaya~\citep[\S~3.2]{Kedlaya2008} when 
expressing the action of the Gauss--Manin connection.  We begin by recalling 
the notation relevant in this chapter.

\begin{notation} \label{not:01-05-main}
We consider a non-singular hypersurface~$\mathfrak{X}$ in 
$\mathbf{P}^n(K) \times \A^1(K)$, where $K$ is a field of characteristic 
zero and $n \geq 2$, defined by a homogeneous polynomial 
$P \in K[t][x_0, \dotsc, x_n]$ of degree~$d$.  We let $\mathfrak{U}$ denote 
the open complement $\mathbf{P}^n(K) \times \A^1(K) - \mathfrak{X}$.

Moreover, we assume that we may also view this as two 
smooth $K$-schemes $\mathfrak{X}$ and $\mathfrak{S}$ 
together with a smooth proper $K$-morphism 
$\pi \colon \mathfrak{X} \to \mathfrak{S}$, where we consider an 
affine subspace $\mathfrak{S} = \Spec(A)$ of the $t$-line.  We 
denote the fibre of $\mathfrak{X}$ above $t$ in $\mathfrak{S}$ 
by $\mathfrak{X}_t$.
\end{notation}

Let us consider a smooth fibre~$\mathfrak{X}_t$.  The embedding 
$\mathfrak{X}_t \into \mathbf{P}^n(K)$ induces maps 
$H_{dR}^i(\mathbf{P}^n(K)/K) \to H_{dR}^i(\mathfrak{X}_t/K)$, which 
by the Lefschetz hyperplane theorem~\citep[\S~1.2, p.\ 156]{GriffithsHarris1978} 
are bijective for $0 \leq i \leq n-2$ and injective for $i = n-1$.  A direct 
computation~\citep[Corollary~3.1.4]{AbbottKedlayaRoe2006} shows that 
$H_{dR}^i(\mathbf{P}^n(K))$ has, as a $K$-vector space, dimension~$1$ if 
$i = 0, 2, \dotsc, 2n$ and $0$ otherwise.  Using Poincar\'e duality, it then 
follows that, for all $0 \leq i \leq 2n-2$ with $i \neq n-1$, 
\begin{equation}
\dim_K H_{dR}^i(\mathfrak{X}_t/K) = \begin{cases} 1 & \text{if $i$ is even,} \\ 
                                    0 & \text{otherwise.} \end{cases}
\end{equation}
In particular, $H_{dR}^{n-1}(\mathfrak{X}_t/K)$ is the only 
cohomology group that remains to be determined.  Defining the complement 
$\mathfrak{U}_t = \mathbf{P}^n(K) - \mathfrak{X}_t$, 
from~\citep[(10.16)]{Griffiths1969}, we have one of the following two exact 
sequences
\begin{gather}
0 \to H_{dR}^n(\mathfrak{U}_t/K) \to H_{dR}^{n-1}(\mathfrak{X}_t/K) \to 0, \\
0 \to H_{dR}^n(\mathfrak{U}_t/K) \to H_{dR}^{n-1}(\mathfrak{X}_t/K) \to H_{dR}^{n+1}(\mathbf{P}^n(K)/K) \to 0,
\end{gather}
as $n$ is even or odd, respectively.

Now define the $n$-form 
\begin{equation}
\Omega = \sum_{i=0}^n (-1)^i x_i d x_0 \wedge \dotsb \wedge \widehat{d x_i} \wedge \dotsb \wedge d x_n.
\end{equation}
A calculation as in~\citep[\S~4]{Griffiths1969} then shows that the 
above cohomology group $H_{dR}^n(\mathfrak{U}_t/K)$ is isomorphic as a 
$K$-vector space to the quotient of the group of $n$-forms $Q \Omega / P^k$ 
with $k \in \N$ and $Q \in K[x_0, x_1, \dotsc, x_n]$ homogeneous of degree 
$k d - (n + 1)$ by the subgroup generated by 
\begin{equation} \label{eq:PoleReductionQuotient}
\frac{(\partial_i Q) \Omega}{P^k} - k \frac{Q (\partial_i P) \Omega}{P^{k+1}}
\end{equation}
for all $0 \leq i \leq n$, where here and in the following $\partial_i$ 
denotes the partial derivative operator with respect to~$x_i$.

Since the fibres of the relative de~Rham cohomology 
$H_{dR}^i(\mathfrak{X}/\mathfrak{S})$ can be identified with the de~Rham 
cohomology $H_{dR}^i(\mathfrak{X}_t/K)$ of the fibres, it 
follows that we can calculate the action of the Gauss--Manin connection 
$\nabla_{GM} \colon H_{dR}^{n-1}(\mathfrak{X}/\mathfrak{S}) \to 
\Omega_{A/K}^1 \otimes_{A} H_{dR}^{n-1}(\mathfrak{X}/\mathfrak{S})$
via the induced map 
$H_{dR}^n(\mathfrak{U}/\mathfrak{S}) \to 
\Omega_{A/K}^1 \otimes_{A} H_{dR}^n(\mathfrak{U}/\mathfrak{S})$, 
which by abuse of notation we shall refer to as $\nabla_{GM}$, too.  
Let $\omega \in H_{dR}^{n}(\mathfrak{U}/\mathfrak{S})$, where we may 
assume it is of the form $Q \Omega / P^k$ as described above.  Since 
$\mathfrak{S}$ is an affine curve, we have that $\Omega_{A/K}^1$ is 
free of rank one, generated by the symbol~$dt$.  Then the action of 
$\nabla_{GM}$ is given by 
\begin{equation}
\nabla_{GM} \colon \omega \mapsto d_{\mathfrak{U}}(\omega) = 
    d_{\mathfrak{S}}(\omega) + d_{\mathfrak{U}/\mathfrak{S}}(\omega) = 
    \frac{\partial}{\partial t} \omega \wedge dt
\end{equation}
where the term $d_{\mathfrak{U}/\mathfrak{S}}(\omega)$ vanishes by 
definition of $\Omega$.  We will describe how a unique representative 
can be obtained for the right-hand side in the following two sections.

% Computing in de Rham cohomology  %%%%%%%%%%%%%%%%%%%%%%%%%%%%%%%%%%%%%%%%%%%%

\section{Computing in de~{R}ham cohomology}

This section is devoted to an in-depth description of the computation in 
de~Rham cohomology, exploiting the vector space isomorphism given in 
the previous section.  We begin by setting up the notation and then describe 
the so-called \emph{reduction of poles} procedure in some generality.  
A particular problem, that of finding the coordinates of an element of a 
multivariate polynomial ideal, is treated as a black box, but we turn to it 
in the second subsection where we specialise to the case when the family of 
projective hypersurfaces contains a diagonal fibre.  Finally, we provide 
further details on the matrix computations involved in the third subsection.

% Subsection:  Reduction of poles  %%%%%%%%%%%%%%%%%%%%%%%%%%%%%%%%%%%%%%%%%%%%

\subsection{Reduction of poles}
\label{sec:01-05-reduction}

This section is based on~\citep[Remark~3.2.5]{AbbottKedlayaRoe2006}, 
describing a \emph{reduction of poles} procedure also referred to as the 
Griffiths--Dwork method.  We continue with the same 
Notation~\ref{not:01-05-main}, but since for large parts of the discussion 
it will not matter that $K(t)$ is a function field, we also define $L = K(t)$.

From the description of $H_{dR}^n(\mathfrak{U}/\mathfrak{S})$ in the 
introduction, with its elements being represented by $n$-forms 
$Q \Omega / P^i$ for $i \in \N$, it is clear that it can be equipped 
with a filtration whose $i$th part consists of all elements which can 
be represented by $n$-forms as above with $\deg Q = k d - (n+1)$ for 
$1 \leq k \leq i+1$.

We can obtain a basis for $H_{dR}^n(\mathfrak{U}/\mathfrak{S})$ 
respecting this filtration as follows.  For every $k \in \N$, we find 
an independent set $B_k$ of polynomials of degree $kd-(n+1)$ generating 
the quotient of the space of all such polynomials by the Jacobian ideal 
$(\partial_0 P, \dotsc, \partial_n P)$.  This yields a generating set 
$\bigcup_{k \in \N} \cB_k$ for $H_{dR}^n(\mathfrak{U}/\mathfrak{S})$ where 
$\cB_k = \{Q \Omega / P^k : Q \in B_k\}$.  However, it follows from a 
theorem of Macaulay~\citep[\S~4, (4.11)]{Griffiths1969} that in fact 
the set $\cB_1 \cup \dotsb \cup \cB_n$ already forms a generating set.  In 
the next subsection, we shall exhibit an explicit basis of monomials in the 
case where the family of projective hypersurfaces contains a diagonal fibre.

Now, to obtain a unique representative for the class of $Q \Omega / P^k$ 
in terms of the basis elements in $\cB_1 \cup \dotsb \cup \cB_n$, we 
express $Q$ in terms of $\partial_0 P, \dotsc, \partial_n P$ as well as 
elements of $B_k$, and then iteratively reduce the pole order of the first 
part according to the relations given by the expressions from 
Equation~\eqref{eq:PoleReductionQuotient}.  Assume that we have at our 
disposal a routine {\sc Decompose} which, given a polynomial $Q$ of degree 
$kd - (n+1)$ in the ideal generated by the Jacobian ideal 
$(\partial_0 P, \dotsc, \partial_n P)$ together with the set $B_k$, returns an 
expression of the form 
$Q = Q_0 \partial_0 P + \dotsb + Q_n \partial_n P + \gamma_k$ with 
$Q_0, \dotsc, Q_n$ homogeneous polynomials in $L[x_0, \dotsc, x_n]$ and 
$\gamma_k$ in the $L$-span of $B_k$.  The reduction of poles procedure can 
then be formalised as in Algorithm~\ref{alg:PoleReduction} below, which we 
will refer to as {\sc Reduce}.  In this generality, the correctness of the 
algorithm depends on a theorem of Macaulay~\citep[\S~4, (4.11)]{Griffiths1969}.

\begin{algorithm}
\caption{Reduce $Q \Omega / P^k$ in $H_{dR}^n(\mathfrak{U}/\mathfrak{S})$}
\label{alg:PoleReduction}
\begin{algorithmic}
\vspace{1mm}
\Require \begin{compactitem}
         \item $P$ is a homogeneous polynomial in $L[x_0, \dotsc, x_n]$ of 
               degree $d$, defining a non-singular hypersurface, where $L$ is 
               a field of characteristic zero.
         \item For $1 \leq k \leq n$, $B_k$ is a basis for all homogeneous 
               polynomials of degree $kd - (n+1)$ modulo the Jacobian ideal 
               $(\partial_0 P, \dotsc, \partial_n P)$.
         \item $Q$ is a homogeneous polynomial of degree $kd-(n+1)$.
         \end{compactitem}
\Ensure  Polynomials $\gamma_i$ in the $L$-span of $B_i$, 
         for $1 \leq i \leq n$, with  
         $Q \Omega / P^k \equiv \gamma_{1} \Omega / P^{1} + \dotsb + \gamma_n \Omega / P^n$.
\Procedure{Reduce}{$P, B_1, \dotsc, B_n, Q$}
\While{$k \geq n+1$}
\State $Q_0, \dotsc, Q_n \gets \Call{Decompose}{Q, \partial_0 P, \dotsc, \partial_n P, B_k}$
\State $k \gets k-1$
\State $Q \gets \tfrac{1}{k} \sum_{i=0}^n \partial_i Q_i$
\EndWhile
\State $\gamma_{k+1}, \dotsc, \gamma_{n} \gets 0$
\While{$Q \not \in B_k$}
\State $Q_0, \dotsc, Q_n \gets \Call{Decompose}{Q, \partial_0 P, \dotsc, \partial_n P, B_k}$
\State $\gamma_{k} \gets Q - \sum_{i=0}^n Q_i \partial_i P$
\State $k \gets k-1$
\State $Q \gets \tfrac{1}{k} \sum_{i=0}^n \partial_i Q_i$
\EndWhile
\If{$Q \neq 0$}
\State $\gamma_{k} \gets Q$
\State $k \gets k-1$
\EndIf
\State $\gamma_{1}, \dotsc, \gamma_{k} \gets 0$
\State \textbf{return} $\gamma_{1}, \dotsc, \gamma_n$
\EndProcedure
\end{algorithmic}
\end{algorithm}

% Subsection: Reduction of poles using linear algebra  %%%%%%%%%%%%%%%%%%%%%%%%

\subsection{Reduction of poles using linear algebra}
\label{sec:01-05-linalg}

In this subsection, we specialise to the case of a smooth family of 
projective hypersurfaces containing a diagonal fibre.  We exhibit a basis 
of monomials $B_1 \cup \dotsb \cup B_n$ such that the corresponding set 
$\cB_1 \cup \dotsb \cup \cB_n$ forms a basis for 
$H_{dR}^n(\mathfrak{U}/\mathfrak{S})$ and re-express the problem of 
decomposing a homogeneous polynomial from the previous section in the 
language of linear algebra.  From this description, we furnish an 
explicit reduction procedure in terms of matrices.  The approach is 
based on a generalisation of Sylvester matrices from two polynomials to 
$n+1$~polynomials, following Macaulay~\citep{Macaulay1994}.

The decomposition problem from the previous section can be formulated as 
follows:

\begin{prob} \label{prob:Decomposition1}
Given a homogeneous polynomial $Q \in L[x_0, \dotsc, x_n]$ of degree 
$k d - (n + 1)$, for some $k \in \N$, we try to find homogeneous polynomials 
$Q_0, \dotsc, Q_n$ in $L[x_0, \dotsc, x_n]$ such that 
\begin{equation} \label{eq:PDecomposition1}
Q \equiv Q_0 \partial_0 P + \dotsb + Q_n \partial_n P
\end{equation}
modulo the $L$-span of $B_k$.
\end{prob}

\begin{rem}
Immediately, we see that, for each $0 \leq i \leq n$, either $Q_i$ is 
identically zero or has degree $(k - 1) d - n$ since $\partial_i P$ is 
homogeneous of degree $d - 1$ and also the elements of $B_k$ have degree 
$kd - (n+1)$.
\end{rem}

For the remaining part of this section we consider the following basis 
sets~$B_k$, for $k \in \N$, which as before induce a generating set 
$\cB_1 \cup \dotsb \cup \cB_n$ of $H_{dR}^n(\mathfrak{U}/\mathfrak{S})$ 
as an $L$-vector space respecting the aforementioned filtration:

\begin{notation}
Given a multi-index $i = (i_0, \dotsc, i_n) \in \N_0^{n+1}$, we let $x^i$ 
denote the monomial $x_0^{i_0} \dotsm x_n^{i_n}$ and set 
$\abs{i} = i_0 + \dotsb + i_n$.
\end{notation}

\begin{defn} \label{defn:BasisSets}
For $k \in \N$, we define the following sets of monomials, 
\begin{align*}
F_k & = \{ x^i : \abs{i} = k d - (n+1) \}, \\
B_k & = \{ x^i : \abs{i} = k d - (n+1) \text{ and $i_j < d-1$ for $0 \leq j \leq n$}\}.
\end{align*}
Moreover, we let $\cF_k$ denote the $L$-vector space spanned by the $n$-forms 
$x^i \Omega / P^k$ for $x^i \in F_k$ and $\cB_k$ denote the set of $n$-forms 
$x^i \Omega / P^k$ with $x^i \in B_k$.
\end{defn}

We shall defer the proof of the statement that the corresponding 
set $\cB_1 \cup \dotsb \cup \cB_n$ indeed forms a basis of 
$H_{dR}^n(\mathfrak{U}/\mathfrak{S})$, at least in the case when 
the family of hypersurfaces given by~$P$ contains a diagonal fibre, 
until the end of this section.

\begin{assumption}
In order to ensure that our computations are not vacuous, we assume that 
$H_{dR}^n(\mathfrak{U}/\mathfrak{S}) \neq 0$.  After setting 
\begin{equation*}
\ell = \ceil{\frac{n+1}{d}}, \quad u = \floor{\frac{(n+1)(d-1)}{d}} = n+1 - \ell,
\end{equation*}
we note that $B_k$ is non-empty if and only if $\ell \leq k \leq u$.  The 
assumption is thus that $\ell \leq u$, which is equivalent to the statement 
that $d \geq 2$ whenever $n$ is odd and $d \geq 3$ whenever $n$ is even.
\end{assumption}

When considering Problem~\ref{prob:Decomposition1}, it turns out that, 
when the family of hypersurfaces given by~$P$ contains a diagonal fibre, 
we can place further restrictions on the polynomials $Q_0, \dotsc, Q_n$.  We 
thus consider the following problem instead:

\begin{prob} \label{prob:Decomposition2}
Given a homogeneous polynomial $Q \in L[x_0, \dotsc, x_n]$ of degree 
$k d - (n + 1)$, for some $k \in \N$, we try to find homogeneous polynomials 
$Q_0, \dotsc, Q_n$ in $L[x_0, \dotsc, x_n]$ such that 
\begin{equation} \label{eq:PDecomposition2}
Q \equiv Q_0 \partial_0 P + \dotsb + Q_n \partial_n P
\end{equation}
modulo the $L$-span of $B_k$.  Moreover, for each $1 \leq j \leq n$,  the 
polynomial $Q_j$ may only contain non-zero coefficients for monomials of 
degree $(k-1)d-n$ that are not divisible by any of the monomials 
$x_0^{d-1}, \dotsc, x_{j-1}^{d-1}$.
\end{prob}

\begin{defn} \label{defn:IndexSets}
For $k \in \N$, we define the following sets of monomials in 
$L[x_0, \dotsc, x_n]$.  Let $R_k = F_k - B_k$ be the set of monomials of 
total degree $kd-(n+1)$ and partial degree at least $d-1$ with respect to some 
of the $n+1$ variables.  Let $C_k^{(0)}$ be the set of monomials of total 
degree $(k-1)d - n$, and then inductively, for $j = 1, \dotsc, n$, define 
$C_k^{(j)}$ to be the set of monomials in $C_k^{(j-1)}$ except for those 
divisible by $x_{j-1}^{d-1}$.  Moreover, we define the multi-set $C_k$ as 
the disjoint union of $C_k^{(0)}, \dotsc, C_k^{(n)}$.  We shall write an 
element of this multi-set as $(j, g)$, referring to a monomial $g$ in 
$C_k^{(j)}$.
\end{defn}

With this set-up, the following theorem provides a solution to 
Problem~\ref{prob:Decomposition2} in the cases that we are interested in:

\begin{thm} \label{thm:Isomorphism}
Suppose that the family of projective hypersurfaces given by the 
polynomial~$P$ in $L[x_0, \dotsc, x_n]$ contains a diagonal fibre.  
Let $k \in \N$ and suppose that $R_k$ and $C_k$ are non-empty.  For 
$0 \leq j \leq n$, let $V_k^{(j)}$ be the $L$-vector space of 
polynomials with basis $C_k^{(j)}$, and let $V_k$ denote their cartesian 
product $V_k = V_k^{(0)} \times \dotsb \times V_k^{(n)}$.  Let $W_k$ be 
the $L$-vector space of polynomials with basis~$R_k$.  Then the $L$-linear 
map 
\begin{equation}
\phi_k \colon V_k \to W_k, 
(Q_0, \dotsc, Q_n) \mapsto Q_0 \partial_0 P + \dotsb + Q_n \partial_n P
\end{equation}
is an isomorphism of $L$-vector spaces.
\end{thm}

\begin{proof}
We first show that, for all $k \in \N$, the multi-sets $R_k$ and $C_k$ 
have the same cardinality:

We construct the following bijection $R_k \to C_k$, representing the 
monomials by their exponent tuple.  Let $i = (i_0, \dotsc, i_n)$ be in 
$R_k$.  If $i_0 \geq d-1$, we define the image as
 $(i_0-d-1, i_1, \dotsc, i_n) \in C_k^{(0)}$.  More generally, if 
$i_0 < d-1, \dotsc, i_{j-1} < d-1$ and $i_j \geq d-1$, the image is 
$(i_0, \dotsc, i_{j-1}, i_j-d-1, i_{j+1}, \dotsc, i_n) \in C_k^{(j)}$.  
It is easy to verify that this map is indeed a bijection.

In order to establish that the map $\phi_k \colon V_k \to W_k$ is an 
isomorphism of $L$-vector spaces, we now exhibit its matrix with respect to 
the given basis:

Let $k \in \N$ and suppose that $R_k$ and $C_k$ are non-empty, that is 
to say, $k \geq n/d + 1$.  We define the auxiliary matrix $\Delta_k$ with 
row and column index sets $R_k$ and $C_k$, respectively, as follows.  
Given $f \in R_k$ and $(j,g) \in C_k$, we set the corresponding entry in 
$\Delta_k$ to be the monomial coefficient of $f/g$ in $\partial_j P$ if 
$g$ divides $f$ and $0$ otherwise.  It is immediate that $\Delta_k$ is the 
matrix representing $\phi_k$ with respect to the bases $C_k$ and $R_k$ of 
$V_k$ and $W_k$, respectively.

The assumption that the family~$\mathfrak{X}$ of projective hypersurfaces 
given by~$P$ contains a diagonal hypersurface means that for some~$t_0$ 
the fibre $\mathfrak{X}_{t_0}$ is given by an equation of the form 
\begin{equation*}
P_{t_0}(x_0, \dotsc, x_n) = a_0 x_0^d + \dotsb a_n x_n^d = 0
\end{equation*}
with $a_0, \dotsc, a_n \in K^{\times}$.

We aim to show that the determinant of $\Delta_k$ is non-zero in~$L$.  Since 
the specialisation to the diagonal fibre viz.\ evaluation of the matrix at 
$t = t_0$ commutes with computing the determinant, it suffices to show that 
the determinant of $(\Delta_k) \big |_{t=t_0}$ is non-zero in~$K$.

Since, for $0 \leq j \leq n$, 
$\partial_j P_{t_0} (x_0, \dotsc, x_n) = d a_j x_j^{d-1}$, there is 
precisely one non-zero entry in each column of $\Delta_k$.  Namely, in column 
$(j, g) \in C_k$ and row $g x_j^{d-1} \in R_k$ there is the non-zero entry 
$d \alpha_j$.  Thus, the determinant of $(\Delta_k) \big |_{t=t_0}$ is given 
by the product of all its non-zero entries, which implies that it is non-zero, 
concluding the proof.
\end{proof}

In principle, by including any of the numerous methods available for solving 
linear equations, we are in a position to furnish a routine {\sc Decompose}, 
which we formalise in Algorithm~\ref{alg:Decompose}.

\begin{algorithm}[ht]
\caption{Obtain co-ordinates for $Q$ in the Jacobian ideal modulo basis elements}
\label{alg:Decompose}
\begin{algorithmic}
\Require \begin{compactitem}
         \item $Q$ is a homogeneous polynomial of degree $kd - (n+1)$.
         \item $\partial_0 P, \dotsc, \partial_n P$ are the partial 
               derivatives of a homogeneous polynomial $P$ in 
               $K[t][x_0, \dotsc, x_n]$ of degree $d$, which defines a smooth 
               family of projective hypersurfaces containing a diagonal fibre, 
               where $K$ is a field of characteristic zero.
         \item $B_k$ is the set of all monomials of total degree $kd-(n+1)$ 
               and partial degree less than $d-1$.
         \end{compactitem}
\Ensure  Homogeneous polynomials $Q_0, \dotsc, Q_n$ such that 
         $Q \equiv Q_0 \partial P_0 + \dotsb + Q_n \partial_n P$ modulo the 
         $K(t)$-span of $B_k$.
\State  The multi-sets $R_k$ and $C_k$ are as in 
        Definition~\ref{defn:IndexSets}, the matrix $\Delta_k$ is as in 
        the proof of Theorem~\ref{thm:Isomorphism}.
\Procedure{Decompose}{$Q, \partial_0 P, \dotsc, \partial_n P, B_k$}
\State \begin{compactenum}[\it {Step} I.] \vspace{-1.24em}
\item Let $b$ be the vector of length $\card{R_k}$ such that the entry 
      corresponding to the monomial $x^i \in R_k$ is the coefficient of 
      $x^i$ in $Q$.
\item Let $v$ be the unique vector of length $\card{C_k}$ satisfying 
      $\Delta_k v = b$.  From the description of $C_k$ as a disjoint union, 
      we can write $v$ accordingly as $\bigl(v^{(0)}, \dotsc, v^{(n)}\bigr)$ 
      where, for $0 \leq j \leq n$, $v^{(j)}$ is a vector of length 
      $\cardts{C_k^{(j)}}$.
\item For $j = 0, \dotsc, n$, set $Q_j = \sum_{g \in C_k^{(j)}} v_g^{(j)} g$,
      where $v_g^{(j)}$ is the entry in $v^{(j)}$ corresponding to the 
      monomial $g \in C_k^{(j)}$.
\item \textbf{return} $Q_0, \dotsc, Q_n$
\end{compactenum}
\EndProcedure
\end{algorithmic}
\end{algorithm}

We conclude this section by establishing that the set 
$\cB_1 \cup \dotsb \cup \cB_n$ is indeed a basis for 
$H_{dR}^n(\mathfrak{U}/\mathfrak{S})$, as claimed earlier, 
using the reduction of poles procedure.

\begin{thm} \label{thm:Basis}
Suppose that the family of projective hypersurfaces given by~$P$ contains 
a diagonal fibre.  Then the set $\cB_1 \cup \dotsb \cup \cB_n$ as in 
Definition~\ref{defn:BasisSets} is a basis for the $L$-vector space 
$H_{dR}^n(\mathfrak{U}/\mathfrak{S})$.
\end{thm}

\begin{proof}
We know that $H_{dR}^n(\mathfrak{U}/\mathfrak{S})$ is spanned 
by the classes of the $n$-forms $Q \Omega / P^k$ for all 
homogeneous polynomials~$Q$ of degree $kd-(n+1)$ and $k \in \N$.  
By a theorem of Macaulay~\citep[\S 4, (4.11)]{Griffiths1969}, 
we may assume that $1 \leq k \leq n$, that is to say, any class in 
$H_{dR}^n(\mathfrak{U}/\mathfrak{S})$ can be represented by an 
$n$-form with a pole of order at most~$n$.

Without loss of generality, we may thus start the reduction of poles 
procedure with a homogeneous polynomial $Q$ of degree $(n+1)d-(n+1)$.  Then, 
since $B_{n+1} = \emptyset$ and $R_{n+1} = F_{n+1}$, 
Theorem~\ref{thm:Isomorphism} shows that there exist homogeneous polynomials 
$Q_0, \dotsc, Q_n$ either zero or homogeneous of degree $nd-(n+1)$ such that 
$Q = Q_0 \partial_0 P + \dotsb + Q_n \partial_n P$.  Continuing with the 
reduction of poles procedure as described in 
Algorithm~\ref{alg:PoleReduction}, we obtain an expression for 
$Q \Omega / P^{n+1}$ as an $L$-linear combination of elements in 
$\cB_1 \cup \dotsb \cup \cB_n$.  This shows that this set spans the vector 
space $H_{dR}^n(\mathfrak{U}/\mathfrak{S})$.

To see that this set is linearly independent, note that it contains only 
monomials whose partial degrees are strictly less than $d-1$.  However, since 
$P$ is a homogeneous polynomial of degree $d$ and the family of hypersurfaces 
contains a diagonal fibre, it follows that, for each $0 \leq i \leq n$, the 
partial derivative $\partial_i P$ is a homogeneous polynomial of degree $d-1$ 
and contains precisely one monomial term with partial degree equal to $d-1$, 
namely that of the monomial $x_i^{d-1}$.  It follows that the elements of 
$B_1 \cup \dotsb \cup B_n$ cannot be reduced further modulo the Jacobian 
ideal and hence that the set $\cB_1 \cup \dotsb \cup \cB_n$ is linearly 
independent.
\end{proof}

% Subsection:  Sparse matrix techniques  %%%%%%%%%%%%%%%%%%%%%%%%%%%%%%%%%%%%%%

\subsection{Sparse matrix techniques}
\label{sec:01-05-sparse}

In this section we present a technique for repeatedly solving a sparse 
system of linear equations as in {\it Step II.} of 
Algorithm~\ref{alg:Decompose}.  The methods we present in this section are 
discussed in great detail in a survey by Reid~\citep{Rei76}.

Before continuing, let us briefly justify why spare matrix methods are 
appropriate in this case:

\begin{rem}
Let $\tau$ and $\tau_0, \dotsc, \tau_n$ denote the number of non-zero 
terms of the polynomials $P$ and $\partial_0 P, \dotsc, \partial_n P$, 
respectively.  We observe that the auxiliary matrix $\Delta_k$ described in 
the proof of Theorem~\ref{thm:Isomorphism} contains at most $\tau_j$ 
non-zero entries in column $(j, g)$, since the non-zero entries in this column 
are coefficients of the polynomial $\partial_j P$ corresponding to monomials 
in $R_k$ via multiplication by $g$.  In particular, the number of non-zero 
entries in the matrix is bounded above by $\tau \card{R_k}$.
\end{rem}

\begin{notation}
In order to simplify the notation to what is relevant in this section, we 
shall consider the system of linear equations given by 
\begin{equation} \label{eq:LinearEquations}
A x = b
\end{equation}
where $A$ is a non-singular $n \times n$ matrix over a field of characteristic 
zero.  Moreover, for later reference we let $\tau$ denote the number of 
non-zero entries in~$A$.
\end{notation}

Our first observation is that the reduction of poles procedure from the 
previous section requires a linear system as in 
Equation~\eqref{eq:LinearEquations} to be solved multiple times for the same 
matrix~$A$ but with different column vectors~$b$, which suggests using a 
pre-processing approach such as $LUP$~decomposition.

The $LUP$ decomposition of a matrix~$A$ is a way of performing the 
classical Gaussian elimination with bookkeeping.  It consists of a 
permutation matrix~$P$, a unit lower-triangular matrix~$L$ and an upper 
triangular matrix~$U$ such that $PA = LU$.  Once these are known, the 
original system can be solved in a two step process as
\begin{equation}
L y = P b, \quad U x = y
\end{equation}
which consists of two triangular systems.  We briefly note that, using a 
classical dense approach~\citep{Cor90}, the $LUP$ decomposition can be 
computed using a number cubic in $n$ of operations in the base field, and 
that solving the two triangular systems only requires a number quadratic in 
$n$ of operations in the base field.

However, assuming that the matrix~$A$ is sparse, with further pre-processing 
a much better performance can be realised.  As a first step, we permute the 
rows of the matrix to ensure that the diagonal contains only non-zero entries.  
We can achieve this here because we assume that the matrix is non-singular.  
Effectively, this problem is the same as that of computing a perfect matching 
in the $n \times n$ bipartite graph with vertex sets $\{v_i\}_{i=1}^n$ and 
$\{w_i\}_{i=1}^n$, where an edge $v_i w_j$ is present if and only the element 
$A_{ij}$ is non-zero.  If we find a perfect matching 
$\{v_{\pi i} w_i\}_{i=1}^n$ in the form of a permutation $\pi \in S_n$, then, 
after defining the permutation matrix $P_0$ via 
\begin{equation}
(P_0)_{ij} = \begin{cases} 1 & \text{if $j = \pi i$}, \\
                           0 & \text{otherwise,} \end{cases}
\end{equation}
we conclude that $P_0 A$ has a zero-free diagonal.  A detailed discussion of 
this problem together with an implementation can be found in the work of 
Duff~\citep{Duf81a,Duf81b}.  We end our discussion here by quoting the 
worst-case and experimental average-case complexity results as $\cO(\tau n)$ 
and $\cO(\tau + n)$, respectively.

The second step is to compute the block-triangularization of the 
matrix~$P_0 A$.  We compute a permutation matrix $Q_0$ such that the matrix 
$Q_0 P_0 A Q_0^t$ is block-triangular, that is, is of the form
\begin{equation}
Q_0 P_0 A Q_0^t = \begin{pmatrix}
                  A^{(11)} &                   &          \\
                  A^{(21)} & A^{(22)} &        &          \\
                  \vdots   &          & \ddots &          \\
                  A^{(N1)} & A^{(N2)} & \hdots & A^{(NN)}
                  \end{pmatrix}
\end{equation}
where each diagonal block $A^{(kk)}$ is square and can itself not be 
symmetrically permuted to block-triangular form.  Again an asymptotically 
optimal algorithm for this problem stems from the realm of graphs:  it is 
Tarjan's linear time algorithm for computing the strongly connected 
components in a directed graph.  In this setting, its asymptotic complexity 
is given by $\cO(n + \tau)$.  For further details, we refer to the joint 
work of Duff and Reid~\citep{Duf78a,Duf78b}.

Finally, this enables us to solve $N$ potentially much smaller systems of 
linear equations instead of the one we started with.  We can rewrite our 
original system of equations $Ax = b$ as $A' y = c$ where 
$A' = Q_0 P_0 A Q_0^t$, $y = (Q_0^t)^{-1} x$ and $c = Q_0 P_0 b$ and now use 
the fact that this system is block-triangular.  This allows us to instead 
solve the sequence of systems of linear equations 
\begin{equation}
A^{(kk)} y_k = c_k - \sum_{j=1}^{k-1} A^{(kj)} y_j
\end{equation}
for $k = 1, \dotsc, N$, where we implicitly think of the column vectors $y$ 
and $c$ to be divided into $N$ corresponding blocks $y_1, \dotsc, y_N$ and 
$c_1, \dotsc, c_N$.  An important consequence of this approach is that after 
the above two steps of pre-processing, the off-diagonal blocks are not changed 
further;  in particular, the phenomenon of fill-in during Gaussian 
elimination, which refers to the introduction of new non-zero entries, is 
limited to the diagonal blocks.

% Description of the algorithm  %%%%%%%%%%%%%%%%%%%%%%%%%%%%%%%%%%%%%%%%%%%%%%%

\section{Description of the algorithm}

We now describe the action of the Gauss--Manin connection~$\nabla_{GM}$ on 
basis elements of $H_{dR}^n(\mathfrak{U}/\mathfrak{S})$.  Suppose that we are 
given a basis element $x^i \Omega / P^k \in \cB_k$, where for 
$1 \leq k \leq n$ the set $\cB_k$ is defined as in 
Definition~\ref{defn:BasisSets}.  Following the description in 
Section~\ref{sec:01-05-hypersurfaces}, the action of the connection 
is given by exterior differentiation, that is, differentiation with respect 
to~$t$.  We first compute 
\begin{equation*}
\frac{d}{dt} \biggl(\frac{x^i \Omega}{P^k}\biggr) = \frac{-k x^i P_t \Omega}{P^{k+1}},
\end{equation*}
where $P_t = dP/dt$.  The second step is to repeatedly apply the reduction of 
poles procedure in de~Rham cohomology in order to express the $n$-form above 
as 
\begin{equation*}
\frac{d}{dt} \biggl(\frac{X^i \Omega}{P^k}\biggr) \equiv 
    \frac{\gamma_{k+1} \Omega}{P^{k+1}} + \dotsb + \frac{\gamma_1 \Omega}{P}
\end{equation*}
where, for $1 \leq i \leq k+1$, $\gamma_i$ is an element in the $K(t)$-span 
of~$B_i$.

We formalise this for later reference in Algorithm~\ref{alg:ConnectionMatrix}.

\begin{algorithm}
\caption{Computing the Gauss--Manin connection matrix}
\label{alg:ConnectionMatrix}
\begin{algorithmic}
\Require Homogeneous polynomial $P$ in $K[t][x_0, \dotsc, x_n]$ of degree $d$, 
         defining a family of smooth projective hypersurfaces containing 
         a diagonal fibre, where $K$ is a field of characteristic zero.
\Ensure  Basis $B_1 \cup \dotsb \cup B_n$ for 
         $H_{dR}^n(\mathfrak{U}/\mathfrak{S})$;  connection matrix~$M$ with 
         respect to this basis.
\Procedure{GMConnection}{$P$}
\State \begin{compactenum}[\it {Step} I.] \vspace{-1.24em}
\item Compute the partial derivatives $\partial_0 P, \dotsc, \partial_n P$ and 
      the exterior derivative $dP/dt$.
\item Compute the basis sets $B_1, \dotsc, B_n$.  In the following, we use the 
      convenient way to index their union by $(i,f)$, referring to a 
      polynomial $f \in B_i$.
\item Compute the auxiliary matrices $\Delta_k$, for 
      $k = \floor{n/d}+1, \dotsc, n+1$, and perform pre-processing on each 
      auxiliary matrix as described in 
      Sections~\ref{sec:01-05-linalg} 
      and~\ref{sec:01-05-sparse}.
\item For all $(j, g) \in \bigcup_{k=1}^n B_k$, let $Q = -j g P_t$ and set 
      $\gamma_{1}, \dotsc, \gamma_n$ to the output of 
      {\sc Reduce($P, B_1, \dotsc, B_n, Q$)}.  
      Then, for each $(i, f) \in \bigcup_{k=1}^n B_k$, let $M_{(i,f),(j,g)}$ 
      be the coefficient of $f$ in $\gamma_i$.
\item \textbf{return} $B_1 \cup \dotsb \cup B_n$, $M$
\end{compactenum}
\EndProcedure
\end{algorithmic}
\end{algorithm}

\section{Computational examples}

[[TODO:  Include a few examples, compare with Alan's code and 
Gerkmann's timings where possible.]]

\section{Further remarks}

[[TODO:  Point out that we restrict ourselves to families defined 
over Fp at this point so that we can assume the lift is defined 
over Z and hence the connection matrix is over Q(t).  We are lazy 
and compute exactly over Q(t) as this is seemingly fast enough 
already at this point.  Could do better still by clearing denominators 
as a power of r(t) up front (note r(0) is a unit) and then working 
over Zp[t] to some precision.  Also point out that in general this 
means computing over a number field or Qq.]]

