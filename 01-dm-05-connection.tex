% Chapter:  Constructing the Gauss--Manin connection  %%%%%%%%%%%%%%%%%%%%%%%%%

[[TODO:  Briefly recap the relative set-up.]]

\section{Introduction}

[[TODO:  Give a brief general introduction to Gauss--Manin connection, 
providing references.  Quickly specialise to hypersurfaces in projective 
space.]]

[[TODO:  Fix some notation, including the choice of basis.]]

\section{Computing in de~{R}ham cohomology}

[[TODO:  Describe the fast reduction routine in algebraic 
de~Rham cohomology.]]

\section{Description of the algorithm}

[[TODO:  Provide an algorithmic description of the overall routine 
used to compute the Gauss--Manin connection.]]

\section{Sparse matrix techniques}

[[TODO:  Describe fast matrix routines used to accelerate the 
computations.]]

\section{Computational examples}

[[TODO:  Include a few examples, compare with Alan's code and 
Gerkmann's timings where possible.]]

\section{Further remarks}

[[TODO:  Point out that we restrict ourselves to families defined 
over Fp at this point so that we can assume the lift is defined 
over Z and hence the connection matrix is over Q(t).  We are lazy 
and compute exactly over Q(t) as this is seemingly fast enough 
already at this point.  Could do better still by clearing denominators 
as a power of r(t) up front (note r(0) is a unit) and then working 
over Zp[t] to some precision.  Also point out that in general this 
means computing over a number field or Qq.]]

