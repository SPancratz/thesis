% Chapter:  Introduction to point counting  %%%%%%%%%%%%%%%%%%%%%%%%%%%%%%%%%%%

% Varieties over finite fields and zeta functions %%%%%%%%%%%%%%%%%%%%%%%%%%%%%

\section{Varieties over finite fields and zeta functions}

Let $X$ denote an algebraic variety over $\mathbf{F}_q$, the finite 
field with $q = p^a$ elements.  In particular, $X$ is also defined 
over every algebraic extension $\mathbf{F}_{q^i}$, for $i \geq 0$, 
and we can let $N_i = \abs{ X(\mathbf{F}_{q^i}) }$ denote the number 
of points of $X$ over $\mathbf{F}_{q^i}$.  The \emph{zeta function} 
of $X$ is then defined as the formal power series in $\mathbf{Q}[[T]]$,
\begin{equation*}
Z(X, T) = \exp \Bigl( \sum_{i=0}^{\infty} \frac{N_i T^i}{i} \Bigr).
\end{equation*}
In the above form, the power series~$Z(X, T)$ is represented by an 
infinite series and it is unclear whether this can be explicitly 
computed by an algorithm viz.\ a \emph{finite} routine.  This issue 
was resolved by Dwork~\citep{Dwork1960} in 1960:

\begin{thm}
Let $X$ be an algebraic variety over the finite field~$\mathbf{F}_q$.  
Then the zeta function $Z(X, T)$ is a quotient of two polynomials with 
integer coefficients.
\end{thm}

Moreover, further details include explicit estimates for the 
degrees of the numerator and denominator polynomials for the 
rational function~$Z(X, T)$.  Dwork's theorem is included 
in a group of results known as the \emph{Weil conjectures}, statements 
conjectured by Weil~\citep{Weil1949} in~1949 which have subsequently 
been proved.  We follow Kedlaya~\citep[Theorem~1.2.1]{Kedlaya2011} in 
the statement of the Weil conjectures:

\begin{thm} \label{thm:01-Zetafunctions}
Let $X$ be a separated scheme of finite type over the finite 
field~$\mathbf{F}_q$.  Then the zeta function of~$X$ is the power series 
representation of a rational functions~$T$.  Moreover, if $X$ is smooth 
and proper over~$\mathbf{F}_q$, then there is a unique way to write 
\begin{equation}
Z(X, T) = \prod_{i=0}^{2 \dim(X)} P_i(T)^{(-1)^{i+1}}
\end{equation}
for some polynomials $P_i(T) \in \mathbf{Z}[T]$ with $P_i(0) = 1$, 
satisfying the following conditions.
\begin{enumerate}
\item We have 
\begin{equation*}
P_i(q^{-i} / T) = \pm q^{-i \deg(P_i) / 2} T^{- \deg(P_i)} P_i(T),
\end{equation*}
with the sign being $+$ whenever $i$ is odd.  In other words, the roots 
of~$P_i$ are invariant under the map $r \mapsto q^{-i} / r$, and if $i$ 
is odd then the multiplicities of $\pm q^{-i/2}$ are even.
\item The complex roots of $P_i$ all have absolute value $q^{-i/2}$.
\item If $X \cong \mathfrak{X}_{\mathbf{F}_q}$ for some smooth proper 
scheme~$\mathfrak{X}$ over the local ring $R = \mathfrak{o}_{K,\mathfrak{p}}$, 
for some number field~$K$ and some prime ideal~$\mathfrak{p}$ of 
$\mathfrak{o}_K$ with residue field~$\mathbf{F}_q$, then for any 
embedding~$K \into \mathbf{C}$, 
\begin{equation*}
\deg(P_i) = \dim_{\mathbf{C}} H^{i} \bigl( (\mathfrak{X} \times_R \mathbf{C})^{\text{an}}, \mathbf{C} \bigr).
\end{equation*}
In other words, $\deg(P_i)$ equals the $i$-th Betti number of 
$\mathfrak{X} \times_{R} \mathbf{C}$.
\end{enumerate}
\end{thm}

A discussion of the proof of this theorem is significantly 
beyond the scope of this thesis.  As a starting point for 
further details, we refer the reader to 
Hartshorne~\citep[Appendix~C]{Har77} and Osserman~\citep{Osserman2008}.

Dwork's theorem on the rationality of the zeta function implies 
that $Z(X, T)$ is already defined by a finite initial segment of 
the sequence~$(N_i)_{i \geq 0}$.  In particular, the problem of 
computing the zeta function can be formulated as follows:

\begin{prob} \label{prob:dm-pointcounting}
Given an algebraic variety~$X$ defined by a finite list of 
multivariate polynomials in~$\mathbf{F}_{q}[x_0, \dotsc, x_n]$, 
efficiently compute a rational function in~$\mathbf{Q}(T)$ 
representing $Z(X, T)$.
\end{prob}

% Point counting algorithms %%%%%%%%%%%%%%%%%%%%%%%%%%%%%%%%%%%%%%%%%%%%%%%%%%%

\section{Point counting algorithms}

Computational routines addressing Problem~\ref{prob:dm-pointcounting} 
are known as \emph{point counting} algorithms.

[[TODO:  Write this chapter, following Jan's thesis.  Currently, 
this only contains a statement of the Weil conjectures from 
Kedlaya.]]
