% Chapter:  Introduction to the deformation method  %%%%%%%%%%%%%%%%%%%%%%%%%%%

% Introduction %%%%%%%%%%%%%%%%%%%%%%%%%%%%%%%%%%%%%%%%%%%%%%%%%%%%%%%%%%%%%%%%

\section{Introduction}

In this section we present the problem that we are concerned 
with in this part of the thesis and carefully set up the notation 
that we are using throughout.  We begin by giving a brief overview.

The aim of the deformation method as introduced by 
Lauder~\citep{Lau04} following a paper by Dwork~\citep{Dwork62b} 
is to compute the zeta function of a smooth 
hypersurface, which in hindsight we call $X_{t_1}$, in projective 
$n$-space over a finite field of characteristic~$p$.  This is 
facilitated by computing 
the action of the Frobenius morphism on the associated rigid cohomology 
spaces $H_{rig}^{\bullet}(X_{t_1})$.  The main idea is to embed this 
hypersurface in a family of smooth projective hypersurfaces~$X/S$ 
defined over a finite field~$k$, where $S \subset \mathbf{A}_k^1(t)$, 
containing a fibre~$X_{t_0}$ for which it is easier to compute the Frobenius 
action. Lifting the family~$X/S$ to a family $\mathcal{X}/\mathcal{S}$ in 
characteristic zero, we can utilise a certain $p$-adic differential 
equation involving the Frobenius matrix and the algebraic Gauss--Manin 
connection.  Moreover, due to a comparison theorem of Baldassarri and 
Chiarellotto~\citep{BalChi94}, in this 
part of the algorithm we can work with the algebraic de~Rham cohomology 
spaces~$H_{dR}^{\bullet}(\mathfrak{X}/\mathfrak{S})$ of the generic 
fibres, which are computationally more accessible.  Solving this differential 
equation and using the Frobenius action 
on~$H_{dR}^{\bullet}(\mathfrak{X}_{t_0})$ as the initial 
condition, by an appropriate evaluation we can compute the Frobenius matrix 
for~$H_{dR}^{\bullet}(\mathfrak{X}_{t_1})$ and hence recover the 
zeta function of the fibre~$X_{t_1}$.

% Families of smooth, projective hypersurfaces %%%%%%%%%%%%%%%%%%%%%%%%%%%%%%%%

\section{Families of smooth, projective hypersurfaces}

For the remaining part of this thesis, we restrict ourselves to the case 
where the family~$X/S$ is defined over a prime field, $t_0 = 0$, and the 
initial fibre~$X_0$ is diagonal.  Specifically, we establish our set-up 
closely following Tuitman~\citep[\S 3.6]{Tuitman2011} and using the same 
choice of resultant as Lauder~\citep[\S 2.3.2]{Lauder2011}.

\begin{notation} \label{not:01-02-main}
Let $\mathbf{F}_q$ be the finite field with $q = p^a$ elements, 
let $\mathbf{Q}_q$ denote the unique unramified extension of $\mathbf{Q}_p$ 
with residue field $\mathbf{F}_q$, and let $\mathbf{Z}_q$ denote its 
ring of integers.

We let $P \in \mathbf{Z}[t][x_0,\dotsc,x_n]$ be a homogenous polynomial 
of degree~$d$ with $n, d \geq 1$.  We define $r(t) \in \mathbf{Z}[t]$ as 
the \emph{Macaulay resultant} of the partial derivatives of~$P$ as 
in~\citep[Page~7, Chapter~I.6]{Macaulay1994}.  We can now define two 
smooth $\mathbf{Z}_q$-schemes $\mathcal{X}$ and $\mathcal{S}$ as 
\begin{align*}
\mathcal{X} & = \Spec\bigl( \mathbf{Z}_q[t,r(t)^{-1}][x_0,\dotsc,x_n] / (P)\bigr), \\
\mathcal{S} & = \Spec\bigl( \mathbf{Z}_q[t,r(t)^{-1}] \bigr)
\end{align*}
together with the obvious smooth projective morphism 
$\mathcal{X} \to \mathcal{S}$.  We also define the generic fibres 
$\mathfrak{X}$ and $\mathfrak{S}$ of $\mathcal{X}$ and $\mathcal{S}$ as 
\begin{align*}
\mathfrak{X} & = \Spec\bigl( \mathbf{Q}_q[t,r(t)^{-1}][x_0,\dotsc,x_n] / (P)\bigr), \\
\mathfrak{S} & = \Spec\bigl( \mathbf{Q}_q[t,r(t)^{-1}] \bigr),
\end{align*}
respectively.  Moreover, we let $X$ and $S$ denote the projections 
of $\mathcal{X}$ and $\mathcal{S}$ to characteristic~$p$. 
Finally, we define complements $\mathcal{U}$, $\mathfrak{U}$, and $U$ 
of $\mathcal{X}$, $\mathfrak{X}$, and $X$, respectively, in the 
ambient projective $n$-space.

We assume that $r(0) \neq 0 \pmod{p}$ and hence that $\mathcal{X}_0$ is 
a smooth, projective hypersurface, which we will moreover assume to be 
diagonal, and that $t_1 \in S$ is a non-zero point.
\end{notation}

\begin{notation}
Furthermore, as an artefact of our particular choice of basis for 
$H_{dR}^n(\mathfrak{U}/\mathfrak{S})$ in Definition~\ref{defn:01-04-basis}, 
we assume that $d \geq 2$ whenever $n$ is odd and $d \geq 3$ whenever $n$ 
is even.
\end{notation}

\begin{rem}
By considering the diagonal fibre~$X_0$, which is a smooth, 
projective hypersurface over $\mathbf{F}_p$, we see that our 
assumptions imply $p \nmid d$.
\end{rem}

% Gauss--Manin connection and Frobenius structure %%%%%%%%%%%%%%%%%%%%%%%%%%%%%

\section{Gauss--Manin connection and Frobenius structures}

We will now briefly describe the actions of the Gauss--Manin connection 
and Frobenius on the vector space $H_{dR}^{n}(\mathfrak{U}/\mathfrak{S})$, 
following Tuitman~\citep[\S 3.4.2, \S 3.6.1]{Tuitman2011} and 
Walker~\citep[\S 3.2.2.2]{Walker2009}.

Suppose that $(v_1,\dotsc,v_b)$ is a basis for the finite-dimensional 
vector space $H_{dR}^{n}(\mathfrak{U}/\mathfrak{S})$ over 
$\mathbf{Q}_q[t,r(t)^{-1}]$.  The Gauss--Manin connection is a map 
\begin{equation*}
\nabla \colon H_{dR}^{n}(\mathfrak{U}/\mathfrak{S}) 
    \to \Omega_{\mathfrak{S}}^{1}
    \otimes_{\mathbf{Q}_q[t,r(t)^{-1}]} H_{dR}^{n}(\mathfrak{U}/\mathfrak{S})
\end{equation*}
and its Leibniz-linear action is represented by a matrix~$M$ over 
$\mathbf{Q}_q[t,r(t)^{-1}]$ with 
\begin{equation*}
\nabla(v_j) = dt \otimes \sum_{i=1}^{b} M_{ij} v_i
\end{equation*}
As the polynomial~$P$ is a polynomial over the integers and 
as our family~$X$ is defined over base field~$\mathbf{F}_p$ 
already, the entries of~$M$ actually lie in~$\mathbf{Q}[t,r(t)^{-1}]$. 
We present further details of this construction and give an explicit 
computational routine for the computation of the matrix~$M$ in 
Chapter~\ref{ch:GMConnection}.

Now let $F_p$ denote the standard lift of the $p$-th power Frobenius 
from $\mathbf{F}_q$ to $\mathbf{Q}_q[t]$, which is equal to 
$\sigma \in \Gal(\mathbf{Q}_q/\mathbf{Q}_p)$ on $\mathbf{Q}_q$ and 
is extended to $\mathbf{Q}_q[t]$ by sending $t$ to $t^p$.  Here, $\sigma$ 
is the unique lift of the $p$-th power Frobenius map to $\mathbf{Q}_q$.  
The map 
\begin{equation*}
p^{-1} F_p \colon (p^{-1} F_p)^{*} H_{dR}^{n}(\mathfrak{U}/\mathfrak{S}) 
    \to H_{dR}^{n}(\mathfrak{U}/\mathfrak{S})
\end{equation*}
can be represented by a matrix~$\Phi$ such that 
\begin{equation*}
p^{-1} F_p (v_j) = \sum_{i=1}^{b} \Phi_{ij} v_i.
\end{equation*}
It can be shown that the entries of~$\Phi$ are in fact overconvergent 
power series, that is, they lie in the ring 
$\mathbf{Q}_q\langle t, r(t)^{-1}\rangle^{\dagger}$ as defined in 
Definition~\ref{defn:Overconvergence}.  This means that the matrix~$\Phi$ 
converges on $\mathbf{P}^{1}(\mathbf{Q}_q)$ with disks of radius strictly 
less than~$1$ removed around the roots of $r(t)$ and~$\infty$.  Thus, the 
entries of~$\Phi$ can be approximated modulo any given power of~$p$ by 
rational functions with denominators some power of~$r(t)$.  We provide 
explicit estimates for this convergence in 
Section~\ref{sec:01-dm-06-continuation}.

As we do not present any original, theoretic improvements to this part of 
the deformation method, that is, the computation of the action of Frobenius 
on the generic fibre, we provide fewer details.

[[TODO:  Mention that we omit lots of details of rigid cohomology etc, 
give some appropriate references.]]

Finally, we note that the Gauss--Manin connection and the action of 
Frobenius satisfy the following commutative diagram,
\begin{equation*}
\def\labelstyle{\textstyle}
\xymatrix@C=6.8em@R=4em{
H_{dR}^{n}(\mathfrak{U}/\mathfrak{S}) \ar[d]_{q^{-1} F_q} \ar[r]^{\nabla} & 
\Omega_{\mathfrak{S}}^{1} \otimes H_{dR}^{n}(\mathfrak{U}/\mathfrak{S}) \ar[d]^{q^{-1} F_q} \\
H_{dR}^{n}(\mathfrak{U}/\mathfrak{S}) \ar[r]^{\nabla} & 
\Omega_{\mathfrak{S}}^{1} \otimes H_{dR}^{n}(\mathfrak{U}/\mathfrak{S}) 
}
\end{equation*}
This implies that the matrices~$\Phi$ and~$M$ satisfy the $p$-adic 
differential equation given by 
\begin{equation*}
\Bigl( \frac{d}{dt} + M \Bigr) \Phi = p t^{p-1} \Phi \sigma(M).
\end{equation*}
Further details regarding this can be found in \citep[\S 5]{Gerkmann2007}.

% Outline %%%%%%%%%%%%%%%%%%%%%%%%%%%%%%%%%%%%%%%%%%%%%%%%%%%%%%%%%%%%%%%%%%%%%

\section{Outline of the algorithm}

We now provide an outline of the deformation method.  The list below 
includes nearly full details of the computation, omitting only the 
specific choice of precisions $N_0, \dotsc, N_4$, and $m, K$.  
These are provided and justified in Chapter~\ref{ch:01-Main}.

\begin{enumerate}
\item[Step~$I$.]
Let $\Phi_0$ denote the matrix for the action of $p^{-1} F_p$ on 
$H_{dR}^{n}(\mathfrak{U}_0)$, which has entries in $\mathbf{Q}_p$.  
We compute an approximation to $p$-adic precision~$N_4$.
\item[Step~$II$.]
We compute the matrix~$M$, which denotes the connection matrix 
on $H_{dR}^{n}(\mathfrak{U}/\mathfrak{S})$ and has entries in 
$\mathbf{Q}[t,r(t)^{-1}]$.
\item[Step~$III$.]
Let $C$ denote the matrix over $\mathbf{Q}[[t]]$ for the local 
expansion of the solution to the $p$-adic differential equation 
$\bigl(\tfrac{d}{dt} + M\bigr) C = 0$.  We compute an approximation 
modulo~$t^K$ to $p$-adic precision~$N_3$.  We also compute an 
approximation for the matrix~$C(t^p)^{-1}$ modulo~$t^K$ to $p$-adic 
precision~$N_3'$.
\item[Step~$IV$.]
Let $\Phi(t)$ denote the matrix for the action of $p^{-1} F_p$ on 
$H_{dR}^{n}(\mathfrak{U}_t)$, which satisfies the equation 
$\Phi(t) = C(t) \Phi_0 C(t^p)^{-1}$ and has entries in 
$\mathbf{Q}_p[t,r(t)^{-1}]$.  We compute an approximation modulo~$t^K$ 
to $p$-adic precision~$N_2$.
\item[Step~$V$.]
Let $\Phi_1$ denote the matrix for the action of $p^{-1} F_p$ 
on $H_{dR}^{n}(\mathfrak{U}_1)$, which has entries in $\mathbf{Q}_q$. 
Firstly, we compute an approximation to $\Psi(t) = r(t)^m \Phi(t)$ 
modulo~$t^K$ to $p$-adic precision~$N_2$.  We can then compute an 
approximation for $\Phi_1$ to $p$-adic precision~$N_2$ as 
$r(\hat{t}_1)^{-m} \Psi(\hat{t}_1)$.
\item[Step~$VI$.]
Let $\Phi_1^{(q)}$ denote the matrix for the action of $q^{-1} F_q$ on 
$H_{dR}^{n}(\mathfrak{U}_1)$, which satisfies 
$\Phi_1^{(q)} = \Phi_1 \sigma(\Phi_1) \dotsm \sigma^{a-1}(\Phi_1)$ and 
has entries in $\mathbf{Q}_q$.  We compute an approximation to this 
matrix with $p$-adic precision~$N_1$.
\item[Step~$VII$.]
Let $p(T) = \det\bigl(1 - T q^{-1} F_q | H_{dR}^n(\mathfrak{U}_{t_1})\bigr)$, 
which is an integer polynomial.  We recover this exactly by computing 
an approximation to the reverse characteristic polynomial of $\Phi_1^{(q)}$ 
to $p$-adic precision~$N_0$.
\end{enumerate}

