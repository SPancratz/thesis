% Chapter:  From the differential system to the zeta function %%%%%%%%%%%%%%%%%

In this chapter we describe the remaining steps of the deformation method:  
how to use the differential system governed by the Gauss--Manin connection 
matrix and the action of Frobenius on the initial fibre $\mathfrak{X}_0$ 
to determine the action of Frobenius on the fibre $\mathfrak{X}_{t_1}$ 
and hence the zeta function of the hypersurface $X_{t_1}$.
We point out that there are essentially no original contributions from the 
author;  we present the work of Lauder~\citep{Lau04} and also follow the 
description by Gerkmann~\citep{Gerkmann2007}, focussing on the details 
relevant to explicit computations.  The improvements in practical runtimes 
that we observe in this part of the algorithm are due to improved estimates 
of certain $p$-adic valuations by Kedlaya~\citep{Kedlaya2010} and estimates 
of pole orders by Kedlaya and Tuitman~\citep{KedlayaTuitman2012} as well as 
implementation details.

%%%%%%%%%%%%%%%%%%%%%%%%%%%%%%%%%%%%%%%%%%%%%%%%%%%%%%%%%%%%%%%%%%%%%%%%%%%%%%%
\section{$p$-Adic differential system and local expansion of Frobenius}

We let $M$ denote the matrix for the Leibniz-linear connection on 
$H_{dR}^n(\mathfrak{U})$, and $\Phi$ the matrix for the $\sigma$-semilinear 
action of $p^{-1} F_p$, where $\sigma$ is the standard lift of the $p$th-power 
Frobenius to~$\mathbf{Q}_q$ and~$\mathbf{Q}_q(t)$, sending $t \mapsto t^p$.  
As described for example in~\citep[\S 5]{Gerkmann2007}, these matrices satisfy 
the following differential system,
\begin{equation} \label{eq:de_phi}
\Bigl(\frac{d}{dt} + M\Bigr) \Phi = p t^{p-1} \Phi \sigma(M).
\end{equation}
Our first goal, discussed in this section, is the computation of 
a power series expansion of $\Phi$ around the origin, satisfying 
Equation~\eqref{eq:de_phi} and the initial condition that $\Phi(0)$ 
be the matrix for the action of $p^{-1} F_p$ on $H_{dR}^n(\mathfrak{U}_0)$.

We begin by observing that, upon setting 
\begin{equation}
\Phi = C \Phi(0) \sigma(C)^{-1},
\end{equation}
the above differential system is equivalent to the homogenous system 
\begin{equation} \label{eq:de}
\Bigl(\frac{d}{dt} + M\Bigr) C = 0
\end{equation}
with initial condition $C(0) = \text{Id}$.  Let us write 
\begin{equation}
M(t) = \frac{B(t)}{r(t)}
\end{equation}
where $B(t) = \sum_{i=0}^{\deg(B)} B_i t^i$ is a matrix over $\mathbf{Z}[t]$ 
and $r(t) \in \mathbf{Z}[t]$.  We can now obtain a power series solution 
$C(t) = \sum_{i \geq 0} C_i t^i$ around $t=0$ for 
\begin{equation}
r(t) \frac{dC}{dt} + B(t) C(t) = 0
\end{equation}
using the recursion 
\begin{equation}
C_{i+1} = \frac{-1}{r_0 (i+1)} \biggl(
    \sum_{j=\max{\{0,i-\deg(B)\}}}^i B_{i-j} C_j + 
    \sum_{j=\max{\{0,i-\deg(r)\}}+1}^i r_{i-j+1} j C_j \biggr).
\end{equation}
where we recall that by assumption $r_0 \neq 0 \pmod{p}$ and so, 
in particular, $r_0 \neq 0$.  [[TODO:  Include this somewhere.]]

From a computational perspective, we are only interested in a power 
series solution to some $p$-adic and $t$-adic precisions $N$ and $K$, 
respectively, as discussed in the next two sections.  As such we need 
to understand the loss of $p$-adic precision involved in the computation 
of $C_0, C_1, \dotsc, C_{K-1}$.  More precisely, let $D_0, D_1, \dotsc$ 
denote an approximation defined by $D_0 = \text{Id}$ and 
\begin{equation}
D_{i+1} = \frac{-1}{r_0 (i+1)} \biggl(
    \sum_{j=\max{\{0,i-\deg(B)\}}}^i B_{i-j} D_j + 
    \sum_{j=\max{\{0,i-\deg(r)\}}+1}^i r_{i-j+1} j D_j \biggr) + 
    p^{\tilde{N}} E_{i+1},
\end{equation}
where $(E_i)_{i \geq 1}$ is a sequence of $p$-adically integral matrices. 

The relationship between the desired precision~$N$ and the working 
precision~$\tilde{N}$ is described by Lauder~\citep[Theorem~5.1]{Lauder2006}, 
but this can be improved significantly by including recent bounds on the 
valuation of $C_0, C_1, \dotsc$ by Kedlaya~\citep{Kedlaya2010}.

\begin{thm} \label{thm:valC}
For all $i \geq 0$, the $p$-adic valuation of the matrix~$C_i$ 
can be bounded by 
\begin{equation}
\ord_p(C_i) \geq - \bigl(2 (r + s) + (n - 1)\bigr) \ceil{\log_p i}
\end{equation}
as well as 
\begin{equation}
\ord_p(C_i) \geq (b - 1) \ord_p(M) - \bigl(2 (r + s) + (n - 1)\bigr) \floor{\log_p i}.
\end{equation}
\end{thm}

\begin{proof}
Let us recall from~\citep[Lemma~3.3]{Gerkmann2007} that 
$\ord_p(\Phi) \geq -(r+s)$, and hence $\ord_p(p^{n-1}\Phi^{-1}) \geq -(r+s)$ 
by Poincar\'e duality.  Thus, we see that 
\begin{equation}
\ord_p(\Phi) + \ord_p(\Phi^{-1}) \geq - 2 (r+s) - (n-1)
\end{equation}
and the two bounds follow now from Theorem~{18.3.3} and Remark~{18.3.4} 
in~\citep{Kedlaya2010}, respectively.
\end{proof}

\begin{thm}
Let the sequences of matrices $(C_i)$ and $(D_i)$ over $\mathbf{Q}$ and 
$\mathbf{Q}_p$, respectively, be defined as above.  Then, for all $i \geq 0$, 
\begin{equation}
\ord_p(C_i - D_i) \geq 
    \tilde{N} - \Bigl(2 \bigl(2 (r+s) + (n-1)\bigr) + 1\Bigr) \ceil{\log_p i}.
\end{equation}
\end{thm}

\begin{proof}
This follows analogously to the result of 
Lauder~\citep[Theorem~5.1]{Lauder2006}.  
Indeed, its proof shows more specifically that 
\begin{equation}
\ord_p(C_i - D_i) \geq 
    \tilde{N} + \min_{k+\ell=i} \Bigl(\ord_p(C_k) + 
                                      \ord_p(\ell^{-1} C_{\ell-1}^{-1})\Bigr).
\end{equation}
Using the bound from Theorem~\ref{thm:valC} and observing that 
they also apply to the inverse matrix~$C^{-1}$, the result follows.
\end{proof}

In order to determine the power series expansion of the matrix $\Phi$, 
we also need to compute the matrix $\sigma(C)^{-1}$.  As we assume that 
the family $X$ of hypersurfaces is defined over $\mathbf{Z}$, the connection 
matrix~$M$ is a matrix over $\mathbf{Q}(t)$ and the local solution~$C$ is 
over $\mathbf{Q}_p[[t]]$.  Thus, $\sigma(C)^{-1} = C(t^p)^{-1} = C^{-1}(t^p)$. 
The matrix $C^{-1}$ could be computed using matrix inversion over the ring 
$\mathbf{Q}_p[[t]]$.  An alternative approach, which is typically favourable 
in practice, follows from observing that whenever $C$ satisfies 
Equation~\eqref{eq:de}, $C^{-1}$ satisfies 
\begin{equation}
\Bigl(\frac{d}{dt} - M^t\Bigr) \bigl(C^{-1}\bigr)^t = 0.
\end{equation}

With $p$-adic approximations to the power series modulo $t^K$ available 
for $C$, $\Phi(0)$, and $\sigma(C)^{-1}$, we are in a position to determine 
the power series for $\Phi = C \Phi(0) \sigma(C)^{-1}$.  The final issue 
to address in this section is the $p$-adic precision that we require for 
the three matrices $C$, $\Phi(0)$, and $C^{-1}$.

We begin by developing generic bounds on the products of $p$-adic numbers 
and matrices:

\begin{prop} \label{prop:productval}
Let $x_1, \dotsc, x_{\ell} \in \mathbf{Q}_p$, where $\ell \geq 2$, 
be given as $x_i = p^{v_i} u_i$ with $v_i = \ord_p(x_i) \in \mathbf{Z}$ 
and $u_i \in \mathbf{Z}_p^{\times}$ whenever $x_i \neq 0$ and 
$u_i = v_i = 0$ otherwise.  Suppose that $N \in \mathbf{Z}$ is given 
such that $N > \sum_{j=1}^{\ell} v_j$ and, for all $i$, $N > \sum_{j \neq i} v_j$.

Let $\tilde{x}_1, \dotsc, \tilde{x}_{\ell}$ be $p$-adic approximations 
satisfying $\ord_p(x_i - \tilde{x}_i) \geq N - \sum_{j \neq i} v_j$ 
for all $i$.  Then 
\begin{equation*}
\ord_p(x_1 \dotsm x_{\ell} - \tilde{x}_1 \dotsm \tilde{x}_{\ell}) \geq N.
\end{equation*}
\end{prop}

\begin{proof}
First, note that $x_i \neq 0$ implies that $\tilde{x}_i \neq 0$. 
Otherwise, if $\tilde{x}_i = 0$ then
\begin{equation*}
N - \sum_{j \neq i} v_j \leq \ord_p(x_i - \tilde{x}_i) = \ord_p(x_i) = v_i
\end{equation*}
which implies that $N \leq \sum v_j$, a contradiction.

Represent $\tilde{x}_i$ as $p^{\tilde{v}_i} \tilde{u}_i$ with the 
same convention that $\tilde{u}_i \in \mathbf{Z}_p^{\times}$ unless 
$\tilde{x}_i = 0$, in which case $\tilde{u}_i = \tilde{v}_i = 0$.  
We first show that $v_i = \tilde{v}_i$ whenever $x_i \neq 0$.  Indeed, 
if $v_i \neq \tilde{v}_i$ then 
\begin{equation*}
\ord_p(x_i - \tilde{x}_i) = \min\{\ord_p(x_i), \ord_p(\tilde{x}_i)\} \leq \ord_p(x_i) = v_i
\end{equation*}
whereas from the assumptions we derive 
\begin{equation*}
\ord_p(x_i - \tilde{x}_i) \geq N - \sum_{j \neq i} v_j > v_i,
\end{equation*}
a contradiction.

Note that, for all $i$ such that $x_i \neq 0$, the assumption 
$\ord_p(x_i - \tilde{x}_i) \geq N - \sum_{j \neq i} v_j$ implies that 
$\ord_p(u_i - \tilde{u}_i) \geq N - \sum_j v_j$.  Specifically, 
\begin{equation*}
\ord_p(u_i - \tilde{u}_i) = \ord_p(x_i - \tilde{x}_i) - v_i \geq N - \sum_{j=1}^{\ell} v_j.
\end{equation*}

Considering the product $x_1 \dotsm x_{\ell}$, we distinguish two 
cases.  First, assume that $x_i \neq 0$ for all $i$.  We find that 
\begin{equation*}
\ord_p(x_1 \dotsm x_{\ell} - \tilde{x}_1 \dotsm \tilde{x}_{\ell}) = 
    \sum_{j} v_j + \ord_p(u_1 \dotsm u_{\ell} - \tilde{u}_1 \dotsm \tilde{u}_{\ell}) \geq N.
\end{equation*}
In the other case, suppose that for some $k < \ell$, $x_1, \dotsc, x_k$ 
are non-zero and that $x_{k+1}, \dotsc, x_{\ell}$ are all zero.  In 
particular, $v_{k+1} = \dotsb = v_{\ell} = 0$.  Then 
\begin{equation*}
\begin{split}
\ord_p(x_1 \dotsm x_{\ell} - \tilde{x}_1 \dotsm \tilde{x}_{\ell}) 
  & = \ord_p(\tilde{x}_1 \dotsm \tilde{x}_{\ell}) \\
  & \geq v_1 + \dotsb + v_k + N - \sum_{j \neq k+1} v_j + \dotsb + N - \sum_{j \neq \ell} v_j \\
  & = (\ell - k) N - (\ell - k - 1) (v_1 + \dotsb + v_k) \\
  & > N. \qedhere
\end{split}
\end{equation*}
\end{proof}

\begin{prop} \label{prop:matrixproductval}
Let $A_1, \dotsc, A_{\ell}$ be $b \times b$ matrices over $\mathbf{Q}_p$, 
where $\ell \geq 2$, given as $A_i = p^{v_i} U_i$ with 
$v_i = \ord_p(A_i) \in \mathbf{Z}$ and at least one entry of $U_i$ a 
$p$-adic unit $A_i \neq 0$ and $v_i = 0$ and $U_i$ the zero matrix 
otherwise.  Suppose that $N \in \mathbf{Z}$ is given such that 
$N > \sum_{j=1}^{\ell} v_j$ and, for all $i$, $N > \sum_{j \neq i} v_j$.

Let $\tilde{A}_1, \dotsc, \tilde{A}_{\ell}$ be $p$-adic approximations 
satisfying $\ord_p(A_i - \tilde{A}_i) \geq N - \sum_{j \neq i} v_j$ 
for all $i$.  Then 
\begin{equation*}
\ord_p(A_1 \dotsm A_{\ell} - \tilde{A}_1 \dotsm \tilde{A}_{\ell}) \geq N.
\end{equation*}
\end{prop}

\begin{proof}
We can follow the proof of Proposition~\ref{prop:productval}, 
observing that, for matrices $A$, $B$ over $\mathbf{Q}_p$, 
we have $\ord_p(A + B) \geq \min \{\ord_p(A), \ord_p(B)\}$.
\end{proof}

\begin{cor}
Suppose that $K, N_2$ are positive integers such that $N_2$ is greater than 
the sum of two or three of $\ord_p(\Phi(0))$, $\ord_p\bigl(C \bmod t^K\bigr)$, 
and $\ord_p\bigl(C^{-1} \bmod{t^{\ceil{K/p}}}\bigr)$.
In order to compute the power series expansion around the origin 
of the matrix~$\Phi$ modulo $p^{N_2}$ and $t^K$, it suffices to 
compute the matrices $C$, $C^{-1}$ and $\Phi(0)$ to $p$-adic 
precision $N_3$, $N_3'$ and $N_4$, respectively, where 
\begin{align*}
N_3  & \geq N_2 + \bigl(2(r+s)+(n-1)\bigr) \ceil{\log_p \ceil{K/p}} + (r+s), \\
N_3' & \geq N_2 + \bigl(2(r+s)+(n-1)\bigr) \ceil{\log_p K} + (r+s), \\
N_4  & \geq N_2 + \bigl(2(r+s)+(n-1)\bigr) \bigl(\ceil{\log_p K} + \ceil{\log_p \ceil{K/p}}\bigr).
\end{align*}
\end{cor}

\begin{proof}
This follows from Proposition~\ref{prop:matrixproductval}, the bounds 
on the valuations of $C$ and $C^{-1}$ from Theorem~\ref{thm:valC} and 
the bound $\ord_p(\Phi(0)) \geq -(r+s)$ from~\citep[Lemma~3.3]{Gerkmann2007}.
\end{proof}

%%%%%%%%%%%%%%%%%%%%%%%%%%%%%%%%%%%%%%%%%%%%%%%%%%%%%%%%%%%%%%%%%%%%%%%%%%%%%%%
\section{Analytic continuation and evaluation}

The matrix~$\Phi(t)$ is computed as matrix of power series in the first place 
by finding the local expansion of the solution to the differential equation 
around~$t = 0$.  Wishing to evaluate its entries at a point different 
from the origin, we are prompted to compute its analytic continuation, that is 
to say, express the entries as rational functions instead of power series.

[[TODO:  Say that the rational functions can only have denominators that 
are powers of $r(t)$.]]

[[TODO:  Say that in order to recover the rational functions to some 
$p$-adic precision, we need determine the power series to a certain 
$t$-adic and $p$-adic precisions.]]

Given a desired $p$-adic precision $N_2$, we would like to determine 
integers $m$ and $K$ such that, modulo $p^{N_2}$, the entries of the 
matrix $r(t)^m \Phi(t)$ are polynomials of degree at most $K$.  We 
will then be able to recover these polynomials from our power series 
approximation modulo $t^K$ computed in the previous section.

Gerkmann developed suitable results in~\citep[\S 6]{Gerkmann2007}, 
however the estimates are not sharp in practice.  There has been recent 
progress by Kedlaya and Tuitman~\citep[Theorem~2.1]{KedlayaTuitman2012}, 
which we present in a slightly simplified form:

\begin{thm} \label{thm:KedlayaTuitman}
Suppose that the poles of the connection matrix~$M$ have distinct 
reductions modulo~$p$, that its poles are simple, and that at each 
pole~$z$ the \emph{exponents} of the connection $\lambda_1, \dotsc, \lambda_b$,
which are defined as the eigenvalues of $(t-z) M \vert_{t=z}$ and known to be 
rational numbers, have non-negative $p$-adic valuation.  Then, modulo $p^N$, 
the matrix $\Phi(t)$ for the action of $p^{-1} F_p$ has a pole at $z$ of order 
at most 
\begin{equation} \label{eq:KedlayaTuitman}
\max_{i} \lambda_i - p \min_{i} \lambda_i + p g(N)
\end{equation}
where $g(N)$ is defined by 
\begin{equation}
g(N) = \max\Bigl\{ i \in \mathbf{N} : i - (r+s) - \bigl(2(r+s) + (n-1)\bigr) \ceil{\log_p i} < N \Bigr\}.
\end{equation}
\end{thm}

\begin{proof}
See \citep[Theorem~2.1]{KedlayaTuitman2012}.
\end{proof}

\begin{rem}
In practice, it might be convenient to avoid computing the exponents and 
verifying the hypotheses of the previous theorem.  This is more relevant 
since even when the hypotheses are not satisfied, the required bounds are 
often closer to the result of Kedlaya and Tuitman than to the estimates 
provided by Gerkmann.

Typically, the contribution of the terms $\max_i \lambda_i - p \min_i \lambda_i$ 
in Equation~\citep{eq:KedlayaTuitman} is small compared to the term $p g(N_2)$. 
Thefore, in a heuristic implementation, one could choose e.g.\ 
$m = 1.1 \times p N_2$ and $K = \deg(r) m$.
\end{rem}

We conclude this section by consider the evaluation of $\Phi(t)$ at the point 
$t=t_1$.  The matrix $\Phi_1$ is defined as the matrix of $p^{-1} F_p$ 
on $H_{dR}^n(\mathfrak{U}_1)$ and it is computed modulo $p^{N_2}$ as 
\begin{equation*}
Phi_1 = r(\hat{t}_1)^{-m} G(\hat{t}_1) \pmod{p^{N_2}}.
\end{equation*}
We claim that it suffices to compute $\hat{t}_1$ and $G$ to precision~$N_2$. 
Indeed, by assumption $r(t_1)$ is a $p$-adic unit and hence so is 
$r(\hat{z}_1)^{-m}$, which implies that it can be computed modulo $p^{N_2}$ 
without precision loss.  The matrix $G$ is computed as $G = r^m Phi$ over 
$\mathbf{Q}_p[[t]]$ modulo~$(p^{N_2}, t^{K})$.

%%%%%%%%%%%%%%%%%%%%%%%%%%%%%%%%%%%%%%%%%%%%%%%%%%%%%%%%%%%%%%%%%%%%%%%%%%%%%%%
\section{Recovering zeta functions}

Recall that the zeta function of $X_1$ is of the form,
\begin{equation*}
Z(X_1,T) = \frac{p(T)^{(-1)^n}}{(1 - T) (1 - qT) \dotsm (1 - q^{n-1}T)}
\end{equation*}
where $p(T) = \det \bigl( 1 - T q^{-1} F_q | H^n(\mathfrak{U}_1) \bigr)$ 
is a polynomial defined over the integers.

Thus, the final two steps in the deformation method are as follows.  
Firstly, we obtain the matrix of the $q$th-power Frobenius from the matrix of 
the $p$th-power Frobenius.  Secondly, we compute its reverse characteristic 
polynomial to suitable $p$-adic precision in order to recover the zeta 
function.

\subsection{Computing the action of $q^{-1} F_q$ on $H(\mathfrak{U}_1)$}

Let $\Phi_1$ and $\Phi_{q,1}$ denote the matrices representing the 
actions of $p^{-1} F_p$ and $q^{-1} F_q$ on $H_{dR}^{n}(\mathfrak{U}_1)$, 
respectively.  Then 
\begin{equation*}
\Phi_{q,1} = \Phi_1 \Phi_1^{\sigma} \dotsm \Phi_1^{\sigma^{a-1}}.
\end{equation*}
where $a = \log_p q$.  Note that the map 
$\sigma \in \Gal(\mathbf{Q}_q / \mathbf{Q}_p)$ is valuation preserving 
and hence the valuations of the matrices 
$\Phi_1, \Phi_1^{\sigma}, \dotsc, \Phi_1^{\sigma^{a-1}}$ are at least 
$-(r+s)$ by Gerkmann~\citep[Lemma~3.3]{Gerkmann2007}.

It now follows from Proposition~\ref{prop:matrixproductval} that 
we it suffices to determine $\Phi_1$ to precision $N_2 \geq N_1 + (a-1) (r+s)$ 
in order to compute $\Phi_{q,1}$ to precision $N_1$.

\subsection{Computing characteristic polynomials}

In this section we address the precision loss when computing the reverse 
characteristic polynomial $\det(1 - t A)$ of a $b \times b$ matrix~$A$ 
given to finite precision over $\mathbf{Q}_p$.

In general, from the definition of the determinant function and 
Proposition~\ref{prop:productval}, it appears that the precision 
loss should be at most $(b-1) \ord_p(A)$.  However, in the case 
of the matrix representing the action of Frobenius 
on $H_{dR}^n(\mathfrak{U}_1)$, much better bounds are available:

\begin{lem} \label{lem:charpoly}
Let $A$ denote the matrix representing the action of $q^{-1} F_q$ 
on $H_{dR}^{n}(\mathfrak{U}_1)$ with respect to some basis.  Define 
two integers $r$ and $s$ via \mbox{$r = \ord_p((n-1)!)$} and 
\mbox{$s = (n + 1) \floorts{\log_p (n-1)}$} and suppose that 
$A$ and $\tilde{A}$ are $p$-adically close, $\ord_p(A-\tilde{A}) \geq N + (r + s)$.  
Then 
\begin{equation*}
\ord_p \bigl( \det(1 - t A) - \det(1 - t \tilde{A}) \bigr) \geq N.
\end{equation*}
\end{lem}

\begin{proof} 
See Gerkmann~\citep[Lemma~3.3, Lemma~3.4]{Gerkmann2007}.
\end{proof}

\begin{defn}
As a consequence of Lemma~\ref{lem:charpoly}, we can take 
$N_1 = N_0 + r + s$ in our description of the deformation 
algorithm.
\end{defn}

\subsection{Computing Weil polynomials}

In the final step of the deformation method, we compute the 
reverse characteristic polynomial $f(T)$ of the matrix $F_{q,1}$, 
which represents $q^{-1} F_q$ on $H^n(\mathfrak{U}_1)$ to some finite 
$p$-adic precision~$N_0$.  Which precision is necessary in order to 
correctly recover the Weil polynomial $p(T)$ over the integers?

We explicitly combine Theorem~{3.2} from Gerkmann~\citep{Gerkmann2007} 
with an improvement suggested by Kedlaya~\citep[Lemma~1.2.3]{Kedlaya2011}
to obtain the following precision bounds:

\begin{thm}
In order to recover $p(T)$ over $\mathbf{Z}$ it suffices to compute 
$f(T)$ modulo~$p^N$ where 
\begin{equation*}
N > \begin{cases}
    2 & (q,n,d) = (2,2,3) \\
    4 & (q,n,d) = (2,2,4) \\
    5 & (q,n,d) = (2,2,5) \\
    1 & (q,n,d) = (3,2,3) \\
    \log_p \bigl( 2 b \floorts{b/2}^{-1} q^{\floor{b/2} (n - 1) / 2} \bigr) & \text{otherwise}
    \end{cases}
\end{equation*}
provided that $\varepsilon = \sgn(\det(q^{-1} F_q))$ is known.
\end{thm}

\begin{proof}
Let us write the reverse characteristic polynomial of $q^{-1} F_q$ as 
\begin{equation*}
p(T) = \sum_{i=0}^{b} a_i T = \prod_{j=1}^{b} (1 - \alpha_j T).
\end{equation*}
We first recall three properties guaranteed by the Weil conjectures:
\begin{enumerate}
\item The map $\alpha \mapsto q^{n-1} \alpha^{-1}$ permutes the roots of $p(T)$.
\item The roots $\alpha_j$ have absolute value $\abs{\alpha_j} = q^{(n-1)/2}$.
\item The zeta function satisfies a function equation, which in terms of 
      the coefficients $a_0, \dotsb, a_b$ yields, for $0 \leq i \leq b$, 
      \begin{equation*}
      a_{b-i} = (-1)^b \varepsilon q^{(n-1) (b/2 - i)} a_i.
      \end{equation*}
\end{enumerate}

We now define the power sums $s_i$, for $i \geq 1$, 
\begin{equation*}
s_i = \alpha_{1}^i + \dotsb + \alpha_{b}^i
\end{equation*}
which satisfy the Newton identities 
\begin{equation*}
s_i + a_1 s_{i-1} + \dotsb a_{i-1} s_1 + i a_i = 0.
\end{equation*}
In particular, $s_i$ is an integer and 
$\abs{s_i} \leq b \max_j{\abs{\alpha_j}}^i = b q^{i(n-1)/2}$.
Assuming that we know $a_1, \dotsc, a_{i-1}$ and $s_1, \dotsc, s_{i-1}$ 
exactly, we can determine $s_i$ as an integer provided we know 
$i a_i$ modulo~$p^N$ with \mbox{$p^N > 2 b q^{i(n-1)/2}$}.

Using the functional equation to relate $a_{i}$ and $a_{b-i}$ 
and knowing the sign~$\varepsilon$, we can compute the top half 
of the coefficients once we have found $a_i$ for 
$0 \leq i \leq \floorts{b/2}$.

Thus, we can recover $p(T)$ as a polynomial over the integers 
from its residue modulo~$p^N$ where 
\begin{equation*}
p^N > \max \biggl\{ \frac{2b q^{k(n-1)/2}}{k} : 0 \leq k \leq \floorts{b/2} \biggr\}.
\end{equation*}

We observe that the sequence $\gamma_k = k^{-1} q^{k (n - 1) / 2}$ 
with $k \geq 1$ is increasing unless $(q,n)$ equals $(2,2)$ or $(3,2)$.  
In these remaining two cases, 
\begin{align*}
\gamma_3 & < \gamma_2 = \gamma_4 < \gamma_5 < \gamma_6 < \gamma_1 < \gamma_7 < \dotsc & (q,n) & = (2,2) \\
\gamma_2 & < \gamma_1 = \gamma_3 < \gamma_4 < \dotsc & (q,n) & = (3,2)
\end{align*}
Thus, we obtain the condition that 
\begin{equation*}
N > \begin{cases}
    2 & (q,n,d) = (2,2,3) \\
    4 & (q,n,d) = (2,2,4) \\
    5 & (q,n,d) = (2,2,5) \\
    1 & (q,n,d) = (3,2,3) \\
    \log_p \bigl( 2 b \floorts{b/2}^{-1} q^{\floor{b/2} (n - 1) / 2} \bigr) & \text{otherwise}
    \end{cases}
\end{equation*}
\end{proof}

\begin{rem}
We observe that $\varepsilon = \sgn(\det(q^{-1} F_q)) = 1$ whenever $b$ is even.

Indeed, from the Weil conjectures, we know that $\alpha \mapsto q^{n-1} \alpha^{-1}$ 
is a permutation of the roots.  Since it is a self-inverse, it partitions the 
set of roots into pairs and hence we can evaluate the product of the roots,
\begin{equation*}
a_b = \prod_{j=1}^{b} \alpha_j = q^{(n-1)b/2}.
\end{equation*}
Using the functional equation to relate $a_0 = 1$ and $a_b$, 
\begin{equation*}
a_{0} = (-1)^{b} \varepsilon q^{-(n-1) b / 2} a_{b}
\end{equation*}
it follows that $\varepsilon = (-1)^b$, which is equal to $1$ if $b$ is even.
\end{rem}

\begin{rem}
Moreover, $b$ is odd if and only if $n$ is odd and $d$ is even, which leaves 
one quarter of the cases to consider.  We suggest two possible ways in which 
one can arrive at a generic and provably correct implementation.

The first approach is to follow the above proof without making use of the 
functional equation, instead using the analogous precision estimate for the 
top coefficient~$a_b$.  This approach roughly doubles the number of $p$-adic 
digits that are required.  Specifically, the analogous result is to require 
\begin{equation*}
N > \begin{cases}
    2 & (q,n,d) = (2,2,3) \\
    5 & (q,n,d) = (2,2,4) \\
    1 & (q,n,d) = (3,2,3) \\
    \log_p \bigl( 2 q^{b (n-1) / 2} \bigr) & \text{otherwise}
    \end{cases}
\end{equation*}

An alternative approach is to first compute an $i' > \floorts{b/2}$ such 
that \mbox{$a_{i'} \neq 0$}.  This can be achieved, for example, by going 
through the entire deformation procedure to compute the reverse characteristic 
polynomial modulo~$p$ only.  In a second pass one then uses the analogous 
precision estimate obtained from the above proof for $a_{i'}$ in order to 
correctly determine the coefficients 
$a_0, \dotsc, a_{\floor{b/2}}, \dotsc, a_{i'}$.  Finally, the functional 
equation can be used to related $a_{b-i'}$ and $a_{i'}$ in order to determine 
$\varepsilon$.
\end{rem}


