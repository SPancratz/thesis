% Chapter:  Examples %%%%%%%%%%%%%%%%%%%%%%%%%%%%%%%%%%%%%%%%%%%%%%%%%%%%%%%%%%

In this chapter we present a small selection of computational examples, 
demonstrating the employability of the deformation method for point counting 
in practice.

We point out that the runtimes that we include in our presentation 
do not include the computation of the exponents of the connection 
matrix occurring in Theorem~\ref{thm:KedlayaTuitman}.  However, 
this computation is not time-consuming when compared to the remaining 
parts of the deformation method.  The omission is a consequence of 
our implementation\footnotemark in the C~language on top of 
FLINT~\citep{FLINT}, lacking the required functionality of root finding 
algorithms and number field arithmetic that is necessary for the 
computation of the exponents.

\footnotetext{The source code is available in a git repository at 
\url{https://github.com/SPancratz/deformation}.}

In some cases, we compute the bounds derived from 
Theorem~\ref{thm:KedlayaTuitman} by hand in order to arrive at 
provably correct results.  But in other cases we use the heuristic 
bounds from Remark~\ref{rem:KedlayaTuitman}.  For practical purposes, 
this does not present a problem at all:  we know that the correct 
output~$p(T)$ is the reverse of a Weil polynomial, that is, 
it is a polynomial with constant term one and all its roots have 
the same absolute value.  This property allows us to at least 
heuristically verify the output, even when computed with insufficient 
precision.

Further work has been carried out by Kedlaya~\citep{Kedlaya2008} on 
verifying the correctness of a computation that has been carried out 
with intermediate $p$-adic precisions too low to be provably correct. 
The methods investigated and implemented are crucially based on the 
fact that the polynomial 
\mbox{$\det\bigl(1 - T q^{-1} F_q | H_{dR}^{n}(\mathfrak{U}_{t_1})\bigr)$} 
is the reverse of a Weil polynomial, and in particular that its 
roots all have absolute value~$q^{(1-n)/2}$ and that the bottom 
coefficients can be obtained correctly even with low $p$-adic precision, 
imposing further restrictions on the remaining coefficients.

The timings that we present in this chapter were obtained on the 
departmental machine {\tt{wolverine.maths.ox.ac.uk}}, featuring twenty-four 
\mbox{Intel Xeon} processors running at~2.93GHz, although none of the 
computations utilise more than one processor.  In the following sections, 
we describe some runtimes as \emph{near instantaneous}, by which we mean 
that the resolution provided by the C~language command {\tt{clock()}} is 
insufficient.

\section{A genus~$6$ curve}

We consider the family of genus~$6$ curves given by 
\begin{equation}
P = x^5 + y^5 + z^5 + t x y z^3 \in \mathbf{F}_3[t][x,y,z]
\end{equation}
which Gerkmann considers in~\citep[\S 7.4]{Gerkmann2007} with 
$a = [\mathbf{F}_q : \mathbf{F}_p] = 40$ and $t_1$ a 
generator for $\mathbf{F}_{3^{40}}^{\times}$.  As the dimension 
$n - 1$ of the hypersurface $X_{t_1}$ is odd, we only need to 
determine the bottom half of the coefficients of the polynomial 
$p(T)$ directly.  In our computation we use the following choice of 
$p$-adic precisions $N_0 = N_1 = N_2 = 127$, $N_3 = 134$, $N_3' = 135$, 
$N_4 = 142$ and note that in the analytic continuation step we 
take $m = 419$ and $K = 2514$.

We now compare the performance of our implementation with the 
timings reported in~\citep[\S 7.4]{Gerkmann2007} where possible, 
providing Gerkmann's timings in parentheses.  The computation of the 
Gauss--Manin connection is near instantaneous with our implementation.  
The computation of the Frobenius matrix on the diagonal fibre takes 
$0.04$~seconds ($5.86$ minutes).  The local solution $C(t)$, its 
inverse $C(t)^{-1}$ and the local expansion for $p^{-1}F_p$ on 
$H_{dR}^n(\mathfrak{U})$ are computed in $0.38$, $0.11$ and $1.28$~seconds, 
respectively, adding up to $1.77$~seconds ($83$ seconds).  The analytic 
continuation and subsequent evaluation at a Teichm\"uller lift of~$t_1$ 
take $0.05$ and $0.2$~seconds, and the computation of the $q$th-power 
Frobenius requires another $7.33$~seconds ($101.4$ minutes).

During the computation of the connection matrix over $\mathbf{Q}(t)$, 
we find that it has denominator $r(t) = 135 t^5 + 15625$.  We 
determine the exponents at each of the poles as $-1$ (with multiplicity~$5$) 
and $0$ (with multiplicity~$7$), which allows us to justify the choice 
of precisions using Theorem~\ref{thm:KedlayaTuitman}.

\section{A quartic surface}

We consider another example that allows for comparison with 
Gerkmann~\citep[\S 7.5]{Gerkmann2007},
\begin{equation}
x_0^4 + x_1^4 + x_2^4 + x_3^4 + t x_0 x_1 x_2 x_3 \in \mathbf{F}_3[t][x_0,x_1,x_2,x_3].
\end{equation}
Here, we choose $a = 20$ and $t_1 = \alpha^{2345}$ where 
$\alpha$ is a generator for $\mathbf{F}_{3^{20}}^{\times}$. 
As the dimension $n - 1 = 2$ is even, a priori the sign of the 
functional equation is unknown.  However, we can use 
Remark~\ref{rem:N0Surfaces} and determine the polynomial 
$3^{20} p\bigl(3^{-20} T\bigr) \in \mathbf{Z}[T]$ instead of $p(T)$. 
Thus, we take $N_0 = 33$, $N_1 = N_2 = 53$, $N_3 = 65$, 
$N_3' = 67$, $N_4 = 79$ as well as $m = 174$ and $K = 870$.

The computation of the connection matrix is near instantaneous 
again, although in this case this is largely due to the sparseness 
of multivariate polynomial defining the family.  The Frobenius 
matrix for the fibre at $t = 0$ is determined in $0.02$~seconds 
($45.26$~minutes).  The local solutions $C(t)$ and $C(t)^{-1}$ 
require $0.27$ and $0.08$ seconds, which together with the 
computation local expansion of Frobenius in $0.15$ seconds 
yields $0.5$~seconds ($19.9$~minutes).  The analytic continuation, 
evaluation at the Teichm\"uller lift of $t_1$  and construction 
of the $q$th-power Frobenius take $0.02$, $0.04$ and $0.62$ seconds, 
respectively, adding up to $0.68$~seconds ($18.66$~minutes).  The 
root-unitary polynomial that we find is given by 
\begin{align*}
3^{20} p\bigl(3^{-20} T\bigr) & = 
  -3486784401 T^{21} - 39675197243 T^{20} - 191506614866 T^{19} \\
& \quad - 482588946510 T^{18} - 552821487569 T^{17} + 243001138765 T^{16} \\
& \quad + 1641410078472 T^{15} + 1793016627512 T^{14} - 410199003010 T^{13} \\
& \quad - 2617001208822 T^{12} - 1586643774924 T^{11} + 1586643774924 T^{10} \\
& \quad + 2617001208822 T^9 + 410199003010 T^8 - 1793016627512 T^7 \\
& \quad - 1641410078472 T^6 - 243001138765 T^5 + 552821487569 T^4 \\
& \quad + 482588946510 T^3 + 191506614866 T^2 + 39675197243 T + 3486784401.
\end{align*}

As before, we prove that our heuristic choice of $t$-adic precisions 
was sufficient by computing the exponents of the connection. 
We find that the connection matrix has denominator $r(t) = 2 t^4 - 512$, 
which has distinct roots modulo~$3$, and at each of the four poles we 
compute the exponents as $-3/2$ (with multiplicity~$1$), $-1/2$ (with 
multiplicity~$15$), and $0$ (with multiplicity~$5$).
Now Theorem~\ref{thm:KedlayaTuitman} establishes that the chosen 
parameters were sufficient.

\section{A quintic surface}

The final example from Gerkmann~\citep[\S 7.6]{Gerkmann2007}, 
containing a diagonal fibre and hence being suitable for comparison, 
is given by 
\begin{equation}
x_0^5 + x_1^5 + x_2^5 + x_3^5 + t x_0^2 x_1 x_2 x_3 \in \mathbf{F}_2[t][x_0,x_1,x_2,x_3],
\end{equation}
with the point $t_1$ chosen as a generator of $\mathbf{F}_{2^{10}}^{\times}$. 
Note that here we do \emph{not} have $p \geq n$ so the Frobenius matrix is not 
automatically integral;  specifically, we find that $r = 1$, $s = 4$.  Thus, 
we can not employ Remark~\ref{rem:N0Surfaces}.  Moreover, the 
dimension~$n-1 = 2$ of the hypersurface $X_{t_1}$ is even and hence 
the sign of the functional equation is a priori unknown.
We choose the $p$-adic precisions $N_0 = 522$, $N_1 = 527$, $N_2 = 572$, 
$N_3 = 721$, $N_3' = 890$, and $N_4 = 872$ and let $m = 1159$, $K = 6954$.

The computation of the connection matrix, which has density $98 / 52^2$ 
or approximately~$0.036$, requires $0.07$~seconds ($40$~seconds).  
The Frobenius matrix on the diagonal fibre requires $3.4$~seconds 
($64.13$~minutes).  The computations of the matrices $C(t)$ and $C(t)^{-1}$ 
are computed in $33.05$ and $17.07$~seconds, and the matrix product 
$C(t) F(0) C^{-1}(t^p)$ is requires another $40.08$~seconds, leading 
to an overall time of $90.2$~seconds ($31.03$~minutes) for the local 
expansion of Frobenius.  Finally, the analytic continuation, evaluation 
at~$\hat{t}_1$ and computation of the $q$th-power Frobenius take $3.09$, 
$0.66$, and $1.16$~seconds, yielding a total of $4.91$~seconds ($4.9$ hours) 
for this part of the computation.

We note that Gerkmann observes that the family in this example does 
not contain any singular fibres, which implies that a much lower value 
of~$m$ would be sufficient.  This observation is important here as we 
cannot apply Theorem~\ref{thm:KedlayaTuitman} in this case.  
We find the denominator of the connection matrix $r(t) = 20 t^5+15625$ 
and compute the exponents $-3/2$ (with multiplicity $4$), 
$-1/2$ (with multiplicity $21$), and $0$ (with multiplicity $27$), 
only to observe that some of them have negative $2$-adic valuation.

\section{A cyclic cubic threefold}

We now consider the family of threefolds given by 
\begin{equation}
-x_0^3 + x_1^3 + x_2^3 + x_3^3 + x_4^3 
+ t\bigl((x_1+x_2)(x_1+2x_3)(x_1+3x_4) + 3 x_2 x_3 (x_1 + x_2 + x_4)\bigr) 
\end{equation}
defined over $\mathbf{F}_{7}$.  With the final fibre defined by $t_1 = 1$, 
this is an example that is extensively discussed by 
Kedlaya~\citep[Example~1.6.1]{Kedlaya2011}.  

Our work demonstrates that examples like this one can be computed in an 
automated fashion in short time.  As the dimension $n-1 = 3$ is odd, our 
choice of $p$-adic precisions is $N_0 = N_1 = N_2 = 12$, $N_3 = 24$, 
$N_3' = 27$, $N_4 = 39$ and we have heuristic $t$-adic parameters $m = 92$ 
and $K = 5796$.

The Gauss--Manin connection matrix is a $10 \times 10$~matrix with 
$50$~non-zero entries.  We find that the denominator $r(t)$ is a degree~$62$ 
polynomial and has $r(1) \equiv 2 \pmod{7}$.  This part of the computation 
requires $0.89$~seconds.  The Frobenius matrix for the diagonal fibre is 
obtained in $0.01$~seconds.  The matrices $C(t)$ and $C(t)^{-1}$ are 
computed in $9.99$ and $1.34$~seconds, respectively, followed by a matrix 
product to compute the local expansion of Frobenius at $t = 0$ in 
$3.67$~seconds.  The analytic continuation step now requires $0.21$~seconds, 
but the remaining part of the computation is near instantaneous.

We also verify that our output polynomial 
\begin{align*}
p(T) & = 4747561509943 T^{10}+96889010407 T^9+17795940687 T^8 +80707214 T^7 \\
& \quad -25529833 T^6 
-756315 T^5-74431 T^4+686 T^3+441 T^2+7 T+1
\end{align*}
agrees with Kedlaya's work~\citep[p.~20]{Kedlaya2011}.

\section{A generic quintic curve}

A family of quintic curves is given by a homogenous polynomial 
in three variables of degree~$5$, which generically has 
$\binom{3+5-1}{5} = 21$ non-zero coefficients.  We choose 
\begin{equation*}
\begin{split}
x^5 + y^5 + z^5 + t \bigl( 
& 3 x^4 y - x^4 z + 2 x y^4 + y^4 z + 4 x z^4 + 5 y z^4 + x^3 y^2 + x^3 z^2 + x^2 y^3 \\
& + y^3 z^2 + x^2 z^3 + y^2 z^3 + x^3 y z + x y^3 z + x y z^3 + x^2 y^2 z + x^2 y z^2 + x y^2 z^2 \bigr)
\end{split}
\end{equation*}
over $\mathbf{F}_{11}$ and $t_1 = 1$.

We note that $n-1 = 1$ is odd, that the sign of the functional 
equation is positive and hence we only need to determine the 
bottom half of the coefficients directly.  This allows us to 
choose the $p$-adic precisions as $N_0 = N_1 = N_2 = 7$, 
$N_3 = 10$, $N_3' = 11$, and $N_4 = 14$ and let $m = 84$, $K = 6636$.

The computation of the connection matrix requires $6.55$~seconds, 
yielding a dense matrix with denominator $r(t)$ a degree~$78$ 
polynomial, and we verify that $r(1) \equiv 3 \pmod{11}$.  We obtain 
the Frobenius matrix for the diagonal fibre in $0.02$~seconds.  The 
local solutions $C(t)$ and $C(t)^{-1}$ are computed in $36.64$ and 
$2.75$~seconds, respectively, which then allows us to compute the 
local expansion of the Frobenius matrix around $t = 0$ in an additional 
$14.45$~seconds.  Finally, the analytic continuation and evaluation steps 
require $0.54$ and $0.02$~seconds.

The output of the computation is the polynomial 
\begin{align*}
p(T) & = 1771561 T^{12}-322102 T^{11}+14641 T^{10}+17303 T^9-1936 T^8 \\
     & \quad +2310 T^7-250 T^6+210 T^5-16 T^4+13 T^3+T^2-2 T+1.
\end{align*}
We heuristically confirm the correctness of this result by 
numerically computing its complex roots, finding that they 
have absolute value $11^{-1/2}$ as required.

\section{A generic sextic curve}

A family of sextic curves is given by a homogenous polynomial 
in three variables of degree~$6$, which generically has 
$\binom{3+6-1}{6} = 28$ non-zero coefficients.  We choose 
\begin{equation*}
\begin{split}
x^6 + y^6 + z^6 + t \bigl( 
& - x^5 y + 7 x^5 z + 2 x y^5 + y^5 z + 2 x z^5 + y z^5 + 2 x^4 y^2 + 2 x^4 z^2 + 3 x^2 y^4 \\ 
& + y^4 z^2 + 3 x^2 z^4 + y^2 x + 2^4 + 3 x^4 y z + 3 x y^4 z + x y z^4 - x^3 y^3 - 2 x^3 z^3 \\
& + 4 y^3 z^3 + 2 x^3 y^2 z + x^3 y z^2 - x^2 y^3 z + x y^3 z^2 + 2 x^2 y z^3 + x y^2 z^3 
+ x^2 y^2 z^2 \bigr)
\end{split}
\end{equation*}
over $\mathbf{F}_{5}$ and $t_1 = 2$.

In this example, we note that $n-1 = 1$ is odd and hence we only need 
to determine the bottom half of the coefficients of $p(T)$ directly.  
The $p$-adic precisions we are led to use are $N_0 = N_1 = N_2 = 13$, 
$N_3 = 18$, $N_3' = 19$, and $N_4 = 24$, and we take heuristic $t$-adic 
parameters $m = 71, K = 8591$.

The Gauss--Manin connection matrix, which is dense and has a degree~$120$ 
denominator $r(t)$ in this example, is computed in $60.08$~seconds.  We 
also calculate $r(t_1) \equiv 1 \pmod{5}$.  The computation of the 
Frobenius matrix for the diagonal fibre takes $0.01$~seconds.  The 
matrices $C(t)$ and $C(t)^{-1}$ require $530.08$ and $99.71$~seconds, 
respectively, and we then compute the local expansion of the Frobenius 
matrix in $94.87$~seconds.  The remaining two steps, the analytic continuation 
and evaluation require $2.71$ and $0.11$~seconds, respectively.

The final result of the computation is the polynomial 
\begin{align*}
p(T) & = 9765625 T^{20}-7812500 T^{19}+1562500 T^{18}+781250 T^{17}-531250 T^{16}\\
     & \quad -56250 T^{15}+108125 T^{14}+5375 T^{13}-21475 T^{12}+4945 T^{11}-503 T^{10}\\
     & \quad +989 T^9-859 T^8+43 T^7+173 T^6-18 T^5-34 T^4+10 T^3+4 T^2-4 T+1
\end{align*}
We verify that the reverse of $p(T)$ is a Weil polynomial and numerically 
find that the absolute value of its complex roots is~$5^{-1/2}$.

\section{A generic quartic surface}

We consider a generic quartic surface, defined by a homogenous equation 
with $35$~non-zero coefficients,
\begin{equation*}
\begin{split}
w^4 + x^4 + y^4 + z^4 
+ t \bigl( & -3 w^3 x + 2 w^3 y - 2 w x y z + w^3 z - w x^3 - 3 x^3 y + y^3 z 
          + 2 x^3 z \\ & + w y^3 - 2 x y^3 - w z^3 + x z^3 + 3 y z^3 + w^2 x^2 
          + 3 w^2 y^2 + w^2 z^2 \\ & + 2 x^2 y^2 - 2 x^2 z^2 + y^2 z^2 
          + 2 w^2 x y + w^2 x z + 3 w^3 y z - w x^2 y \\ & + 2 w x^2 z 
          + 3 x^2 y z - w x y^2 
          + 3 w y^2 z + x y^2 z + 2 w x z^2 + 2 w y z^2 + 2 x y z^2 \bigr)
\end{split}
\end{equation*}
defined over $\mathbf{F}_5$ with $t_1 = 1$.

This example is a surface and we have $p \geq n$, hence the matrix of Frobenius 
is automatically integral.  We aim to compute $5 p(T/5)$ instead of $p(T)$, 
following Remark~\ref{rem:N0Surfaces}.  This allows us to take $N_0 = 10$, 
$N_1 = N_2 = 11$, $N_3 = 21$, $N_3' = 23$, and $N_4 = 33$, together with 
$m = 60$ and $K = 13740$.

The Gauss--Manin connection matrix is computed in $323.55$~seconds and found 
to have denominator $r(t)$ of degree~$228$, which explains the large value 
of the $t$-adic parameter~$K$.  The matrix for $p^{-1} F_p$ on 
$H_{dR}^{n}(\mathfrak{U}_0)$ is computed in $0.01$~seconds.  The local 
solutions $C$ and $C^{-1}$ form the most time-consuming part in this example, 
requiring $1909.3$ and $342.22$~seconds.  The matrix product to determine the 
local expansion of $p^{-1} F_p$ on $H_{dR}^{n}(\mathfrak{U}_t)$ takes 
$262.6$~seconds. Finally, the analytic continuation and evaluation steps 
require $4.53$ and $0.18$~seconds, respectively.

\section{Another quartic surface}

We consider family of quartic surfaces inspired by an example of 
Abbott, Kedlaya, and Roe~\citep[Example~4.2.1]{AbbottKedlayaRoe2006}, 
\begin{equation*}
w^4 + x^4 + y^4 + z^4 + t \bigl(
    -x y^3 + w x y^2 + w x y z  + w^2 x y - w^2 x z + w y^3 - w y^2 z \bigr) 
\end{equation*}
defined over $\mathbf{F}_3$.  They consider the smooth surface~$X_1$ 
and report a runtime of $40,144$~seconds, or $11.15$~hours.  Unfortunately, 
with our current implementation we cannot tackle this exact example 
as, with our fixed choice of basis, the Gauss--Manin connection matrix 
has an apparent singularity at $t = 1$ (and also at $t = 2$).  Instead, 
we let $t_1$ be a generator of $\mathbf{F}_{3^2}^{\times}$.

Once again, we can employ Remark~\ref{rem:N0Surfaces}, which here allows 
us to take $p$-adic precisions $N_0 = 15$, $N_1 = N_2 = 17$, $N_3 = 33$, 
$N_3' = 35$, $N_51$, and $t$-adic parameters $m = 56$ and $K = 9632$.

The computation of the Gauss--Manin connection matrix requires $36.6$~seconds, 
which also shows that the denominator $r(t)$ is a degree~$168$ polynomial.
The action of $p^{-1} F_p$ for the diagonal fibre is determined in 
$0.01$~seconds.  The local solutions $C(t)$ and $C(t)^{-1}$ are found in 
$953.59$ and $311$~seconds, which then allow us to compute the local expansion 
of Frobenius in $153.44$~seconds.  The analytic expansion and 
evaluation require $3.77$ and $0.24$~seconds, respectively, and the final 
construction of the $q$th-power Frobenius is near instantaneous. 
This yields a total runtime of about $1,458$~seconds.

\section{A cubic surface}

We consider the following example of a cubic surface over 
$\mathbf{F}_5$ due to Abbott, Kedlaya, and 
Roe~\citep[Example~4.3.2]{AbbottKedlayaRoe2006},
\begin{equation*}
w^3 + 3 x^3 + y^3 + 2 z^3 
+ t \bigl( 3 x y^2 - w x y + 3 w x z - w^2 x - w y^2 \bigr).
\end{equation*}
They report that the computation of the matrix representing 
the action of $p^{-1} F_p$ for the fibre at $t_1 = 1$ modulo 
$5^3$ required $5,482$~seconds.

For smooth projective cubic surfaces, $h_{0,2} = 0$ and hence 
we aim to compute $p(T/5)$, using Remark~\ref{rem:N0Surfaces}. 
This also shows that it is sufficient to take $N_0 = 3$ as 
in~\citep{AbbottKedlayaRoe2006}.  We are led to 
take $N_1 = N_2 = 4$, $N_3 = 10$, $N_3' = 12$, $N_4 = 18$, and 
$m = 22$, $K = 616$.

Using our implementation, the connection matrix is computed 
in $0.16$~seconds, the computation of the action of Frobenius 
on the diagonal fibre is near instantaneous, we obtain $C(t)$, 
$C(t)^{-1}$ and the local expansion of $p^{-1} F_p$ around the 
origin together in $0.2$~seconds, and the remaining steps of 
the deformation method are near instantaneous again.  This yields 
an overall runtime of about $0.36$~seconds, which improves on 
the runtime reported in~\citep{AbbottKedlayaRoe2006} by a factor 
of $15,227$.

