% Chapter:  Precision estimates  %%%%%%%%%%%%%%%%%%%%%%%%%%%%%%%%%%%%%%%%%%%%%%

\section{Precision estimates}

Most of the $p$-adic computations involved in the deformation 
method can only be carried out to finite precision, which means 
that we need to be able to effectively bound the $p$-adic 
precision lost during the various steps.  Determining the precison 
that we require for the final computation and working backwards 
through the algorithm, we find the precision required at 
each step.

\subsection{Computing Weil polynomials}

Recall that the zeta function of $X_1$ is of the form,
\begin{equation*}
Z(X_1,T) = \frac{p(T)^{(-1)^n}}{(1 - T) (1 - qT) \dotsm (1 - q^{n-1}T)}
\end{equation*}
where $p(T) = \det \bigl( 1 - T q^{-1} F_q | H^n(\mathfrak{U}_1) \bigr)$ 
is a polynomial defined over the integers.

In the final step of the deformation method, we compute the 
reverse characteristic polynomial $f(T)$ of the matrix $F_{q,1}$, 
which represents $q^{-1} F_q$ on $H^n(\mathfrak{U}_1)$ to some finite 
$p$-adic precision~$N_0$.  Which precision is necessary in order to 
correctly recover the Weil polynomial $p(T)$ over the integers?

We explicitly combine Theorem~{3.2} from Gerkmann~\citep{Gerkmann2007} 
with an improvement suggested by Kedlaya~\citep[Lemma~1.2.3]{Kedlaya2011}
to obtain the following precision bounds:

\begin{thm}
In order to recover $p(T)$ over $\mathbf{Z}$ it suffices to compute 
$f(T)$ modulo~$p^N$ where 
\begin{equation*}
N > \begin{cases}
    2 & (q,n,d) = (2,2,3) \\
    4 & (q,n,d) = (2,2,4) \\
    5 & (q,n,d) = (2,2,5) \\
    1 & (q,n,d) = (3,2,3) \\
    \log_p \bigl( 2 b \floorts{b/2}^{-1} q^{\floor{b/2} (n - 1) / 2} \bigr) & \text{otherwise}
    \end{cases}
\end{equation*}
provided that $\varepsilon = \sgn(\det(q^{-1} F_q))$ is known.
\end{thm}

\begin{proof}
Let us write the reverse characteristic polynomial of $q^{-1} F_q$ as 
\begin{equation*}
p(T) = \sum_{i=0}^{b} a_i T = \prod_{j=1}^{b} (1 - \alpha_j T).
\end{equation*}
We first recall three properties guaranteed by the Weil conjectures:
\begin{enumerate}
\item The map $\alpha \mapsto q^{n-1} \alpha^{-1}$ permutes the roots of $p(T)$.
\item The roots $\alpha_j$ have absolute value $\abs{\alpha_j} = q^{(n-1)/2}$.
\item The zeta function satisfies a function equation, which in terms of 
      the coefficients $a_0, \dotsb, a_b$ yields, for $0 \leq i \leq b$, 
      \begin{equation*}
      a_{b-i} = (-1)^b \varepsilon q^{(n-1) (b/2 - i)} a_i.
      \end{equation*}
\end{enumerate}

We now define the power sums $s_i$, for $i \geq 1$, 
\begin{equation*}
s_i = \alpha_{1}^i + \dotsb + \alpha_{b}^i
\end{equation*}
which satisfy the Newton identities 
\begin{equation*}
s_i + a_1 s_{i-1} + \dotsb a_{i-1} s_1 + i a_i = 0.
\end{equation*}
In particular, $s_i$ is an integer and 
$\abs{s_i} \leq b \max_j{\abs{\alpha_j}}^i = b q^{i(n-1)/2}$.
Assuming that we know $a_1, \dotsc, a_{i-1}$ and $s_1, \dotsc, s_{i-1}$ 
exactly, we can determine $s_i$ as an integer provided we know 
$i a_i$ modulo~$p^N$ with \mbox{$p^N > 2 b q^{i(n-1)/2}$}.

Using the functional equation to relate $a_{i}$ and $a_{b-i}$ 
and knowing the sign~$\varepsilon$, we can compute the top half 
of the coefficients once we have found $a_i$ for 
$0 \leq i \leq \floorts{b/2}$.

Thus, we can recover $p(T)$ as a polynomial over the integers 
from its residue modulo~$p^N$ where 
\begin{equation*}
p^N > \max \biggl\{ \frac{2b q^{k(n-1)/2}}{k} : 0 \leq k \leq \floorts{b/2} \biggr\}.
\end{equation*}

We observe that the sequence $\gamma_k = k^{-1} q^{k (n - 1) / 2}$ 
with $k \geq 1$ is increasing unless $(q,n)$ equals $(2,2)$ or $(3,2)$.  
In these remaining two cases, 
\begin{align*}
\gamma_3 & < \gamma_2 = \gamma_4 < \gamma_5 < \gamma_6 < \gamma_1 < \gamma_7 < \dotsc & (q,n) & = (2,2) \\
\gamma_2 & < \gamma_1 = \gamma_3 < \gamma_4 < \dotsc & (q,n) & = (3,2)
\end{align*}
Thus, we obtain the condition that 
\begin{equation*}
N > \begin{cases}
    2 & (q,n,d) = (2,2,3) \\
    4 & (q,n,d) = (2,2,4) \\
    5 & (q,n,d) = (2,2,5) \\
    1 & (q,n,d) = (3,2,3) \\
    \log_p \bigl( 2 b \floorts{b/2}^{-1} q^{\floor{b/2} (n - 1) / 2} \bigr) & \text{otherwise}
    \end{cases}
\end{equation*}
\end{proof}

\begin{rem}
We observe that $\varepsilon = \sgn(\det(q^{-1} F_q)) = 1$ whenever $b$ is even.

Indeed, from the Weil conjectures, we know that $\alpha \mapsto q^{n-1} \alpha^{-1}$ 
is a permutation of the roots.  Since it is a self-inverse, it partitions the 
set of roots into pairs and hence we can evaluate the product of the roots,
\begin{equation*}
a_b = \prod_{j=1}^{b} \alpha_j = q^{(n-1)b/2}.
\end{equation*}
Using the functional equation to relate $a_0 = 1$ and $a_b$, 
\begin{equation*}
a_{0} = (-1)^{b} \varepsilon q^{-(n-1) b / 2} a_{b}
\end{equation*}
it follows that $\varepsilon = (-1)^b$, which is equal to $1$ if $b$ is even.
\end{rem}

\begin{rem}
Moreover, $b$ is odd if and only if $n$ is odd and $d$ is even, which leaves 
one quarter of the cases to consider.  We suggest two possible ways in which 
one can arrive at a generic and provably correct implementation.

The first approach is to follow the above proof without making use of the 
functional equation, instead using the analogous precision estimate for the 
top coefficient~$a_b$.  This approach roughly doubles the number of $p$-adic 
digits that are required.  Specifically, the analogous result is to require 
\begin{equation*}
N > \begin{cases}
    2 & (q,n,d) = (2,2,3) \\
    5 & (q,n,d) = (2,2,4) \\
    1 & (q,n,d) = (3,2,3) \\
    \log_p \bigl( 2 q^{b (n-1) / 2} \bigr) & \text{otherwise}
    \end{cases}
\end{equation*}

An alternative approach is to first compute an $i' > \floorts{b/2}$ such 
that \mbox{$a_{i'} \neq 0$}.  This can be achieved, for example, by going 
through the entire deformation procedure to compute the reverse characteristic 
polynomial modulo~$p$ only.  In a second pass one then uses the analogous 
precision estimate obtained from the above proof for $a_{i'}$ in order to 
correctly determine the coefficients 
$a_0, \dotsc, a_{\floor{b/2}}, \dotsc, a_{i'}$.  Finally, the functional 
equation can be used to related $a_{b-i'}$ and $a_{i'}$ in order to determine 
$\varepsilon$.
\end{rem}

\subsection{Computing characteristic polynomials}

In this section we address the precision loss when computing the reverse 
characteristic polynomial $\det(1 - t A)$ of a $b \times b$ matrix $A$ 
given to finite precision over $\mathbf{Q}_p$.

From the definition of the determinant, 
\begin{align*}
\det(1 - t A) & = \sum_{\sigma \in S_b} \sgn(\sigma) 
                    \prod_{i=1}^{b} (1 - t A)_{i,\sigma(i)} \\
              & = \sum_{\sigma \in S_b} \sgn(\sigma) 
                    \prod_{i=1}^{b} \bigl( \delta_{i,\sigma(i)} - t A_{i,\sigma(i)} \bigr)
\end{align*}
where $S_b$ is the permutation group on $\{1,\dotsc,b\}$ and $\delta_{i,j}$ 
is equal to $1$ or $0$ as $i = j$ or $i \neq j$, respectively.

\begin{prop} \label{prop:productval}
Let $x_1, \dotsc, x_{\ell} \in \mathbf{Q}_p$, where $\ell \geq 2$, 
be given as $x_i = p^{v_i} u_i$ with $v_i = \ord_p(x_i) \in \mathbf{Z}$ 
and $u_i \in \mathbf{Z}_p^{\times}$ whenever $x_i \neq 0$ and 
$u_i = v_i = 0$ otherwise.  Suppose that $N \in \mathbf{Z}$ is given 
such that $N > \sum_{j=1}^{\ell} v_j$ and, for all $i$, $N > \sum_{j \neq i} v_j$.

Let $\tilde{x}_1, \dotsc, \tilde{x}_{\ell}$ be $p$-adic approximations 
satisfying $\ord_p(x_i - \tilde{x}_i) \geq N - \sum_{j \neq i} v_j$ 
for all $i$.  Then 
\begin{equation*}
\ord_p(x_1 \dotsm x_{\ell} - \tilde{x}_1 \dotsm \tilde{x}_{\ell}) \geq N.
\end{equation*}
\end{prop}

\begin{proof}
First, note that $x_i \neq 0$ implies that $\tilde{x}_i \neq 0$. 
Otherwise, if $\tilde{x}_i = 0$ then
\begin{equation*}
N - \sum_{j \neq i} v_j \leq \ord_p(x_i - \tilde{x}_i) = \ord_p(x_i) = v_i
\end{equation*}
which implies that $N \leq \sum v_j$, a contradiction.

Represent $\tilde{x}_i$ as $p^{\tilde{v}_i} \tilde{u}_i$ with the 
same convention that $\tilde{u}_i \in \mathbf{Z}_p^{\times}$ unless 
$\tilde{x}_i = 0$, in which case $\tilde{u}_i = \tilde{v}_i = 0$.  
We first show that $v_i = \tilde{v}_i$ whenever $x_i \neq 0$.  Indeed, 
if $v_i \neq \tilde{v}_i$ then 
\begin{equation*}
\ord_p(x_i - \tilde{x}_i) = \min\{\ord_p(x_i), \ord_p(\tilde{x}_i)\} \leq \ord_p(x_i) = v_i
\end{equation*}
whereas from the assumptions we derive 
\begin{equation*}
\ord_p(x_i - \tilde{x}_i) \geq N - \sum_{j \neq i} v_j > v_i,
\end{equation*}
a contradiction.

Note that, for all $i$ such that $x_i \neq 0$, the assumption 
$\ord_p(x_i - \tilde{x}_i) \geq N - \sum_{j \neq i} v_j$ implies that 
$\ord_p(u_i - \tilde{u}_i) \geq N - \sum_j v_j$.  Specifically, 
\begin{equation*}
\ord_p(u_i - \tilde{u}_i) = \ord_p(x_i - \tilde{x}_i) - v_i \geq N - \sum_{j=1}^{\ell} v_j.
\end{equation*}

Considering the product $x_1 \dotsm x_{\ell}$, we distinguish two 
cases.  First, assume that $x_i \neq 0$ for all $i$.  We find that 
\begin{equation*}
\ord_p(x_1 \dotsm x_{\ell} - \tilde{x}_1 \dotsm \tilde{x}_{\ell}) = 
    \sum_{j} v_j + \ord_p(u_1 \dotsm u_{\ell} - \tilde{u}_1 \dotsm \tilde{u}_{\ell}) \geq N.
\end{equation*}
In the other case, suppose that for some $k < \ell$, $x_1, \dotsc, x_k$ 
are non-zero and that $x_{k+1}, \dotsc, x_{\ell}$ are all zero.  In 
particular, $v_{k+1} = \dotsb = v_{\ell} = 0$.  Then 
\begin{equation*}
\begin{split}
\ord_p(x_1 \dotsm x_{\ell} - \tilde{x}_1 \dotsm \tilde{x}_{\ell}) 
  & = \ord_p(\tilde{x}_1 \dotsm \tilde{x}_{\ell}) \\
  & \geq v_1 + \dotsb + v_k + N - \sum_{j \neq k+1} v_j + \dotsb + N - \sum_{j \neq \ell} v_j \\
  & = (\ell - k) N - (\ell - k - 1) (v_1 + \dotsb + v_k) \\
  & > N. \qedhere
\end{split}
\end{equation*}
\end{proof}

\begin{cor}
Let $A$ be a $b \times b$ matrix over $\mathbf{Q}_p$ 
where $b \geq 2$.  With the same notation as in the 
previous Proposition, suppose that $N \in \mathbf{Z}$ 
satisfies, for all $\sigma \in S_b$ and $i$, 
\begin{equation*}
N > \sum_{j=1}^{b} v_{j,\sigma(j)}, \quad N > \sum_{j \neq i} v_{j,\sigma(j)}.
\end{equation*}
Assume that $\tilde{A}$ is an approximation to $A$
satisying, for all $\sigma \in S_b$ and $i$, 
\begin{equation*}
\ord_p \bigl(A_{i,\sigma(i)} - \tilde{A}_{i,\sigma(i)}\bigr) \geq N - \sum_{j \neq i} v_{j,\sigma(j)}.
\end{equation*}
Then 
\begin{equation*}
\ord_p\bigl( \det(1 - t A) - \det(1 - t \tilde{A}) \bigr) \geq N.
\end{equation*}
\end{cor}

\begin{rem}
In particular, if $\ord_p(A) = \ord_p(\tilde{A}) = -v < 0$ and 
$\ord_p(A - \tilde{A}) \geq N + (b - 1) v$ with $N$ satisfying 
the assumptions of the previous Corollary, then 
$\ord_p\bigl(\det(1 - t A) - \det(1 - t \tilde{A})\bigr) \geq N$.
\end{rem}

However, the matrices that we are interested in are far from generic 
as they represent the action of Frobenius and much better bounds are 
available.  The following result applies in the most general situation 
that we consider:

\begin{lem} \label{lem:charpoly}
Let $A$ denote the matrix representing the action of $q^{-1} F_q$ 
on $H^n(U)$ with respect to the monomial basis.  Define two 
integers $r$ and $s$ via \mbox{$r = \ord_p((n-1)!)$} and 
\mbox{$s = (n + 1) \floorts{\log_p (n-1)}$} and suppose that 
$A$ and $\tilde{A}$ are $p$-adically close, $\ord_p(A-B) \geq N + (r + s)$.  
Then 
\begin{equation*}
\ord_p \bigl( \det(1 - t A) - \det(1 - t \tilde{A}) \bigr) \geq N.
\end{equation*}
\end{lem}

\begin{proof} 
See Gerkmann~\citep[Lemma~3.3, Lemma~3.4]{Gerkmann2007}.
\end{proof}

\begin{rem}
As a consequence of Lemma~\ref{lem:charpoly}, we can take 
$N_1 = N_0 + r + s$ in our description of the deformation 
algorithm.

In particular, note that in the case that the prime~$p$ 
satisfies $p > n - 1$ we can choose $r = s = 0$ and there 
is no precision loss in this step of the computation.
\end{rem}

\subsection{Computing the action of $q^{-1} F_q$ on $H(\mathfrak{U}_1)$}

Let $F_1$ and $F_{q,1}$ denote the matrices representing the 
actions of $p^{-1} F_p$ and $q^{-1} F_q$ on $H(\mathfrak{U}_1)$, 
respectively.  By Gerkmann~\citep[Lemma~3.3]{Gerkmann2007}, 
the $p$-adic valuation of both matrices is bounded below by $-(r+s)$.  
We aim to compute $F_{q,1}$ via 
\begin{equation*}
F_{q,1} = F_1 F_1^{\sigma} \dotsm F_1^{\sigma^{a-1}}.
\end{equation*}

Since there is no precision loss involved in applying $\sigma$ [[TODO: Reference]], 
we can indeed compute $F_{q,1} \bmod p^{N_1}$ from $F_1 \bmod p^{N_1}$ 
by computing the product exactly over $\mathbf{Q}_p$ and only reducing 
modulo $p^{N_1}$ at the end.

\subsection{Analytic continuation and evaluation}

The matrix~$F_t$ is computed as matrix of power series in the first place 
by finding the local expansion of the solution to the differential equation 
around~$t = 0$.  Wishing to evaluate its entries at a point different 
from~$0$, we are prompted to compute its analytic continuation, that is to say, 
express the entries as rational functions instead of power series.

[[TODO:  Mention previously existing bounds by Gerkmann.]]

[[TODO:  Something about Jan and Kiran's paper, giving details about 
provable choices of $m$ and $K_1$.]]

\begin{rem}
Typically, the contribution of the terms 
$\max_i \lambda_i - p \min_i \lambda_i$ in Equation~[[TODO: Ref]] is small 
compared to $p g(N_1)$.  Therefore, in a simplified heuristic implementation, 
one can choose $m = 1.1 \times p N_1$ and $K_1 = (\deg(r) + 1) m$.
\end{rem}

The matrix $F_1$ is defined as the matrix of $p^{-1} F_p$ 
on $H(\mathfrak{U}_1)$ and it is computed as 
\begin{equation*}
F_1 = r(\hat{z}_1)^{-m} G(\hat{z}_1) \pmod{p^{N_1}}.
\end{equation*}
We claim that it suffices to compute 
$\hat{z}_1$ and $G$ to precision $p^{N_1}$.  Indeed, by 
assumption $r(z_1)$ is non-zero in $\mathbf{F}_p$ and so 
$r(\hat{z}_1)$ has valuation~$0$.  Therefore, $r(\hat{z}_1)^{-m}$ 
has valuation~$0$ and can be computed modulo $p^{N_1}$ 
without precision loss.

The matrix $G$ is computed as $G = r^m F_t$ over $\mathbf{Q}_p[[t]]$ 
modulo~$(p^{N_1}, t^{K_1})$.

\subsection{Local expansion}

From the previously computed pieces of data, that is, the 
matrix~$F_0$ representing $p^{-1} F_p$ on $H(\mathfrak{U}_0)$ 
and the matrix~$C(t)$ giving the solution to the homogeneous 
matrix differential equation, we can compute the matrix~$F(t)$ 
as 
\begin{equation} \label{eq:genfrob}
F(t) = C(t) F_0 C(t^p)^{-1}.
\end{equation}
Note that we wish to compute $F(t)$ modulo $(p^{N_1}, t^{K_1})$.

Regarding the $t$-adic precision, it is immediate that we 
need to compute $C(t)$ modulo $t^{K_1}$.  The third factor 
$C(t^p)^{-1}$ can be computed as $C(t)^{-1} \vert_{t=t^p}$, 
which requires us to compute $C(t)^{-1}$ modulo $t^{\ceil{K_1 / p}}$.

In order to establish bounds on the $p$-adic precision 
that require for the three factors, we need estimates of 
their $p$-adic denominators:

From~\citep[Lemma~3.3]{Gerkmann2007}, we know that the $p$-adic 
valuation of $F_0$ is bounded below by $-(r+s)$.

There are two bounds available in the literature that are 
relevant for the other two factors.  We begin by writing 
$C(t) = \sum_{i=0}^{\infty} C_i t^i$ with $C_i$ a matrix 
over $\mathbf{Q}_p$.  The first result we present can be 
found in the work of Lauder~[[TODO: Add Lauder Reference]] 
and Gerkmann~\citep[Theorem~5.1]{Gerkmann2007}.

\begin{thm}
The $p$-adic valuation of the $b \times b$ matrix $C_t$ is 
bounded from below by 
\begin{equation*}
\ord_p(C_i) \geq - \bigl( (b-1) + \ord_p((b-1)!) + 
    \min\{b-1,\ord_p \prod_j \binom{b}{j}\} \floor{\log_p(i)}.
\end{equation*}
\end{thm}

An improvement to this can be found in the work of 
Kedlaya~\citep[Theorem~18.3.3]{Kedlaya2010}.

[[TODO:  Develop a useful bound from Kedlaya's book.  Note that 
the factor $(1 - n)$ in the following result is likely to be 
wrong and will have to include something like $-(r + s)$.]]

\begin{thm}
\begin{equation*}
\ord_p(C_i) \geq (1 - n) \floor{\log_p{i}}.
\end{equation*}
\end{thm}

Consequently, we can bound the $p$-adic valuation of $C(t) \bmod t^{K_1}$ 
using the bound for $C_{K_1}$.  Similarly, by considering the differential 
equation 
\begin{equation*}
\bigl(d/dt - M^t\bigr) (C^{-1})^t = 0
\end{equation*}
we can bound the $p$-adic valuation of $C^{-1}(t^p)$.
\begin{cor}
The valuations of $C(t) \bmod{t^K}$ and $C^{-1}(t^p) \bmod{t^K}$ 
\end{cor}

With these bounds on the valuations for the three factors in 
Equation~\eqref{eq:genfrob} we aim to determine the values for 
$N_2$ and $N_3$ in the deformation method.

We have an analogous result to Proposition~\ref{prop:productval} 
for products of matrices:

\begin{prop} \label{prop:matrixproductval}
Let $A_1, \dotsc, A_{\ell}$ be $b \times b$ matrices over $\mathbf{Q}_p$, 
where $\ell \geq 2$, given as $A_i = p^{v_i} U_i$ with 
$v_i = \ord_p(A_i) \in \mathbf{Z}$ and at least one entry of $U_i$ a 
$p$-adic unit $A_i \neq 0$ and $v_i = 0$ and $U_i$ the zero matrix 
otherwise.  Suppose that $N \in \mathbf{Z}$ is given such that 
$N > \sum_{j=1}^{\ell} v_j$ and, for all $i$, $N > \sum_{j \neq i} v_j$.

Let $\tilde{A}_1, \dotsc, \tilde{A}_{\ell}$ be $p$-adic approximations 
satisfying $\ord_p(A_i - \tilde{A}_i) \geq N - \sum_{j \neq i} v_j$ 
for all $i$.  Then 
\begin{equation*}
\ord_p(A_1 \dotsm A_{\ell} - \tilde{A}_1 \dotsm \tilde{A}_{\ell}) \geq N.
\end{equation*}
\end{prop}

\begin{proof}
We can follow the proof of Proposition~\ref{prop:productval}, 
observing that, for matrices $A$, $B$ over $\mathbf{Q}_p$, 
we have $\ord_p(A + B) \geq \min \{\ord_p(A), \ord_p(B)\}$.
\end{proof}

[[TODO:  Determine actual values for $N_2$ and $N_3$, once we have 
corrected the bounds for $\ord_p(C_i)$.  Suppose that 
$\ord_p(C) \geq x$, $\ord_p(F_0) \geq y$, and $\ord_p(C^{-1}) \geq z$.]]

\begin{cor}
In order to compute the product $F(t) = C(t) F(0) C^{-1}(t^p)$ to 
precision $N_1$, we need to have $C(t)$, $F(0)$, and $C^{-1}(t^p)$ 
to precisions 
\begin{align*}
N_2  & \geq N_1 - y - z, \\
N_3  & \geq N_1 - x - z, \\
N_2' & \geq N_1 - x - y,
\end{align*}
respectively.
\end{cor}

\begin{proof}
This follows from Proposition~\ref{prop:matrixproductval}, observing 
that the bounds on the valuations of $C(t)$, $F(0)$, and $C^{-1}(t^p)$ 
are non-positive.
\end{proof}

[[TODO:  Working precisions for the local solution to the differential equation.]]
